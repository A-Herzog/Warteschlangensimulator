\RequirePackage{amsmath}
\RequirePackage{fix-cm}
\documentclass{svmono}

\def\ColoredLinks{}
%%%%%%%%%%%%%%%%%%%%%%%%%%%%%%%%%%%%%%%%%%%
% Alexanders Standardmacros - Version 3.0 %
%%%%%%%%%%%%%%%%%%%%%%%%%%%%%%%%%%%%%%%%%%%





%%%%%%%%%%%%%%%%%%%%%%%%%%%%%%%%%%%%%%%%%%%%%%%%
%  Farben für Links definieren                 %
%%%%%%%%%%%%%%%%%%%%%%%%%%%%%%%%%%%%%%%%%%%%%%%%

\ifdefined\ColoredLinks
  \def\linkColorLinkR{0}
  \def\linkColorLinkG{0}
  \def\linkColorLinkB{0.55}
	\def\linkColorCiteR{0}
  \def\linkColorCiteG{0}
  \def\linkColorCiteB{0.55}
	\def\linkColorUrlR{0}
  \def\linkColorUrlG{0}
  \def\linkColorUrlB{0.55}
\else
  \def\linkColorLinkR{0}
  \def\linkColorLinkG{0}
  \def\linkColorLinkB{0}
	\def\linkColorCiteR{0.3}
  \def\linkColorCiteG{0.3}
  \def\linkColorCiteB{0.3}
	\def\linkColorUrlR{0}
  \def\linkColorUrlG{0}
  \def\linkColorUrlB{0}
\fi





%%%%%%%%%%%%%%%%%%%%%%%%%%%%%%%%%%%%%%%%%%%%%%%%
%  Packages einbinden                          %
%%%%%%%%%%%%%%%%%%%%%%%%%%%%%%%%%%%%%%%%%%%%%%%%

%\usepackage{makeidx} - unterschägt manchmal Einträge
\usepackage{imakeidx} % Speziellen Features eigentlicht nicht verwendet, aber hat die makeidx-Fehler nicht
\usepackage[utf8]{inputenc}
\usepackage{amssymb}
\usepackage[ngerman]{babel}
\usepackage{epsf}
\usepackage[dvips]{rotating}
\usepackage{amsmath,amsfonts}
\usepackage[amsmath,thmmarks,noconfig]{ntheorem}
\usepackage{nameref}
\usepackage{hycolor}
\usepackage{hyperxmp}
\usepackage{hyperref}
\hypersetup{pdfauthor={Alexander Herzog},
            pdftitle={Warteschlangensimulator - Kurzeinführung},
            %pdfsubject={},
            %pdfkeywords={},
            pdfproducer={LaTeX},
            pdfcreator={pdfLaTeX},
						pdfcopyright={Copyright Alexander Herzog},
						pdfcontactcity={Clausthal-Zellerfeld},
						pdfcontactpostcode={38678},
						pdfcontactcountry={Deutschland},
						pdfcontactemail={alexander.herzog@tu-clausthal.de},
						pdfcontacturl={https://www.simzentrum.de,https://www.tu-clausthal.de},
						pdflang={de},
						bookmarksnumbered,
						colorlinks=true,
						filecolor=[rgb]{0,0,0},  % \href-Links nicht hervorheben
						linkcolor=[rgb]{\linkColorLinkR,\linkColorLinkG,\linkColorLinkB},
						citecolor=[rgb]{\linkColorCiteR,\linkColorCiteG,\linkColorCiteB},
						urlcolor=[rgb]{\linkColorUrlR,\linkColorUrlG,\linkColorUrlB} %\url{}, nur für E-Mail-Link genutzt
					}
\expandafter\def\expandafter\UrlBreaks\expandafter{\UrlBreaks
  \do\a\do\b\do\c\do\d\do\e\do\f\do\g\do\h\do\i\do\j
  \do\k\do\l\do\m\do\n\do\o\do\p\do\q\do\r\do\s\do\t
  \do\u\do\v\do\w\do\x\do\y\do\z\do\A\do\B\do\C\do\D
  \do\E\do\F\do\G\do\H\do\I\do\J\do\K\do\L\do\M\do\N
  \do\O\do\P\do\Q\do\R\do\S\do\T\do\U\do\V\do\W\do\X
  \do\Y\do\Z}
\usepackage{float}
\usepackage{fancyvrb}
\restylefloat{figure}
\restylefloat{table}
\usepackage{sectsty}
%\allsectionsfont{\fontfamily{cmss}\selectfont} - das führt zu Type 3 Fonts
\allsectionsfont{\sffamily\selectfont}
\usepackage{framed}
\usepackage[toc,page]{appendix}
\renewcommand\appendixname{Anhang}
\let\appendixtocname\appendixname\let\appendixpagename\appendixname
\usepackage[T1]{fontenc} % aus der pdf kopierbare Umlaute
\usepackage{eurosym}
\usepackage{longtable}

\usepackage{graphicx}
\usepackage[export]{adjustbox}




%%%%%%%%%%%%%%%%%%%%%%%%%%%%%%%%%%%%%%%%%%%%%%%%
%  Rotierte Tabellenüberschriften              %
%%%%%%%%%%%%%%%%%%%%%%%%%%%%%%%%%%%%%%%%%%%%%%%%

\usepackage{adjustbox}
\usepackage{array}
\usepackage{booktabs}
\usepackage{multirow}

\newcolumntype{R}[2]{%
    >{\adjustbox{angle=#1,lap=\width-(#2)}\bgroup}%
    l%
    <{\egroup}%
}
\newcommand*\rot{\multicolumn{1}{|R{90}{1em}|}}





%%%%%%%%%%%%%%%%%%%%%%%%%%%%%%%%%%%%%%%%%%%%%%%%
%  Symbole definieren                          %
%%%%%%%%%%%%%%%%%%%%%%%%%%%%%%%%%%%%%%%%%%%%%%%%

\newcommand{\setH}{\mathbb{H}}
\newcommand{\setC}{\mathbb{C}}
\newcommand{\setR}{\mathbb{R}}
\newcommand{\setQ}{\mathbb{Q}}
\newcommand{\setZ}{\mathbb{Z}}
\newcommand{\setN}{\mathbb{N}}
\newcommand{\comp}{\complement}
\newcommand{\Umg}{{\cal U}}

\newcommand{\calA}{{\cal A}}
\newcommand{\calB}{{\cal B}}
\newcommand{\calC}{{\cal C}}
\newcommand{\calD}{{\cal D}}
\newcommand{\calE}{{\cal E}}
\newcommand{\calF}{{\cal F}}
\newcommand{\calG}{{\cal G}}
\newcommand{\calH}{{\cal H}}
\newcommand{\calI}{{\cal I}}
\newcommand{\calJ}{{\cal J}}
\newcommand{\calK}{{\cal K}}
\newcommand{\calL}{{\cal L}}
\newcommand{\calM}{{\cal M}}
\newcommand{\calN}{{\cal N}}
\newcommand{\calO}{{\cal O}}
\newcommand{\calP}{{\cal P}}
\newcommand{\calQ}{{\cal Q}}
\newcommand{\calR}{{\cal R}}
\newcommand{\calS}{{\cal S}}
\newcommand{\calT}{{\cal T}}
\newcommand{\calU}{{\cal U}}
\newcommand{\calV}{{\cal V}}
\newcommand{\calW}{{\cal W}}
\newcommand{\calX}{{\cal X}}
\newcommand{\calY}{{\cal Y}}
\newcommand{\calZ}{{\cal Z}}

\def\arccot{\mathop{\rm arccot}}
\def\Arcoth{\mathop{\rm Arcoth}}
\def\Arsinh{\mathop{\rm Arsinh}}
\def\Artanh{\mathop{\rm Artanh}}
\def\Arcosh{\mathop{\rm Arcosh}}

\def\d{{\rm d}}
\def\dx{\d x}
\def\dy{\d y}
\def\dz{\d z}
\def\dt{\d t}
\def\euler{\mathrm{e}}

\def\id{{\rm id}}
\def\Kern{{\rm Kern}}

\def\ra{\Rightarrow}

\def\oversym#1#2{\mathop{#1}\limits^{#2}}
\def\Inneres#1{\oversym{#1}\circ}


\def\TO#1{\oversym\longrightarrow{#1}}
\def\RA#1{\oversym\Longrightarrow{#1}}
\def\IFF#1{\oversym\iff{#1}}
\def\gleich#1{\oversym={#1}}

\def\oBdA{o.\,B.\,d.\,A.\ }

\renewcommand{\qed}{\begin{flushright} $\square$ \end{flushright}}

\def\Definition#1{{\bf #1}}

\def\E{{\bf E}}
\def\Var{{\bf Var}}
\def\Std{{\bf Std}}
\def\CV{{\bf CV}}
\def\SCV{{\bf SCV}}





%%%%%%%%%%%%%%%%%%%%%%%%%%%%%%%%%%%%%%%%%%%%%%%%
%  Umgebungen definieren                       %
%%%%%%%%%%%%%%%%%%%%%%%%%%%%%%%%%%%%%%%%%%%%%%%%

%\theoremstyle{changebreak}
%\theoremheaderfont{\normalfont\bfseries}
%\theoremseparator{:}
%\theoremsymbol{}
%\newtheorem{definition}{Definition}[chapter]
%\newtheorem{satz}[definition]{Satz}
%\newtheorem{hilfssatz}[definition]{Hilfssatz}
%\newtheorem{lemma}[definition]{Lemma}
%\newtheorem{beispiel}[definition]{Beispiel}
%\newtheorem{beispiele}[definition]{Beispiele}
%\newtheorem{bezeichnung}[definition]{Bezeichnung}
%\theorembodyfont{\rmfamily}
%\newtheorem{bemerkung}[definition]{Bemerkung}
%\newtheorem{vereinbarung}[definition]{Vereinbarung}
%\newtheorem{folgerung}[definition]{Folgerung}
%\theoremheaderfont{\normalfont\bfseries}
%\theoremstyle{nonumberchangebreak}
%\theorembodyfont{\rmfamily}
%\theoremsymbol{$\blacksquare$}
%\newtheorem{beweis}{Beweis}
%\theoremsymbol{}





%%%%%%%%%%%%%%%%%%%%%%%%%%%%%%%%%%%%%%%%%%%%%%%%
%  Standard epsilon durch schöneres ersetzen   %
%%%%%%%%%%%%%%%%%%%%%%%%%%%%%%%%%%%%%%%%%%%%%%%%

\def\epsilon{\varepsilon}





%%%%%%%%%%%%%%%%%%%%%%%%%%%%%%%%%%%%%%%%%%%%%%%%
%  Verbatim-Umgebungen für verschiedene Zwecke %
%%%%%%%%%%%%%%%%%%%%%%%%%%%%%%%%%%%%%%%%%%%%%%%%

\DefineVerbatimEnvironment{ExcelVerbatim}{Verbatim}{frame=single,label=Excel-Befehl,fontsize=\small}
\DefineVerbatimEnvironment{ExcelVerbatimWithMath}{Verbatim}{frame=single,label=Excel-Befehl,commandchars=\\\{\},codes={\catcode`$=3\catcode`^=7},fontsize=\small}
\DefineVerbatimEnvironment{ExcelVerbatimWithFormat}{Verbatim}{frame=single,label=Excel-Befehl,fontsize=\small,commandchars=\\\{\}}
\DefineVerbatimEnvironment{RVerbatim}{Verbatim}{frame=single,label=R-Code,fontsize=\small}
\DefineVerbatimEnvironment{ExcelMacroVerbatim}{Verbatim}{frame=single,label=Excel-Makro,fontsize=\small} % ,numbers=left
\DefineVerbatimEnvironment{ExcelMarcroVerbatimWithMath}{Verbatim}{frame=single,label=Excel-Makro,commandchars=\\\{\},codes={\catcode`$=3\catcode`^=7},fontsize=\small}
\DefineVerbatimEnvironment{ExcelMacroVerbatimWithFormat}{Verbatim}{frame=single,label=Excel-Makro,fontsize=\small,commandchars=\\\{\}}
\newcommand{\sub}[2]{{#1_#2}}

\newcommand{\newlinesymbol}{\rotatebox[origin=c]{270}{$\curvearrowright$}}





%%%%%%%%%%%%%%%%%%%%%%%%%%%%%%%%%%%%%%%%%%%%%%%%
%  Formatierungen                              %
%%%%%%%%%%%%%%%%%%%%%%%%%%%%%%%%%%%%%%%%%%%%%%%%

%\def\emphasis#1{\textbf{#1}}
%\def\emphasisBFOnly#1{\textbf{#1}}
\def\emphasis#1{\emph{#1}}
\def\emphasisBFOnly#1{#1}

\def\emphasisIT#1{\textit{#1}}
\def\emphasisEnglish#1{\textit{#1}}

%\font\deutschfont=suet14
%\def\deutsch#1{\hbox{\deutschfont #1}}

\parindent0pt
\parskip5pt

%\oddsidemargin4.6mm
%\evensidemargin-5.4mm
%\textwidth160mm
\usepackage{xcolor}
\usepackage{lmodern}
\usepackage{wrapfig}
\usepackage[english]{keystroke}
\textwidth160mm
\textheight220mm
\oddsidemargin0mm
\evensidemargin0mm
\topmargin0mm

\def\cmd#1{\textbf{"\texttt{#1}"}}
\def\cm#1{\textbf{\texttt{#1}}}

\begin{document}

\hypersetup{pageanchor=false}
\thispagestyle{empty}
\vskip5cm {\Huge\textbf{Hotkey reference for \vskip.1cm Warteschlangensimulator}}
\vskip.5cm \hrule
\vskip.5cm {\large \textsc{Alexander Herzog} (\href{mailto:alexander.herzog@tu-clausthal.de}{alexander.herzog@tu-clausthal.de})}
\IfFileExists{../../Warteschlangennetz-mittel.png}{
\vskip2cm \centerline{\fbox{\includegraphics[width=16cm]{../../Warteschlangennetz-mittel.jpg}}}
}{}
\vskip2cm {\color{gray}
This reference refers to version 4.6.0
 of Warteschlangensimulator.\\
Download address: \href{https://a-herzog.github.io/Warteschlangensimulator/}{https://a-herzog.github.io/Warteschlangensimulator/}.
}

\vfill
\pagebreak

\setcounter{page}{1}

\renewcommand{\arraystretch}{2}



\part{Hotkeys}

The hotkeys listed below refer to the drawing area of Warteschlangensimulator. In addition to these hotkeys, many menu items can be activated with a key combination. Which key combination triggers which menu item is displayed directly in the menu.

\vskip1em
\begin{tabular}{|p{4cm}|p{11.5cm}|}
\hline
\textbf{Hotkey}&\textbf{Action}\\
\hline\hline
\keystroke{Menu}&
Opens the context menu for the selected element or the selected elements or for the drawing area if no element is selected.\\
\hline
\Del&
Deletes the selected element or the selected elements.\\
\hline
\Shift + \Del&
Deletes the selected element element and trys to close the path with regard to the incoming and outgoing edges.\\
\hline
\Ctrl + \Enter&
Opens the edit dialog for the selected element.\\
\hline
\Ctrl + \Shift + \Enter&
Shows the simulation data during an animation for the selected element.\\
\hline
\Shift + \Enter&
Opens the sub-model edit dialog (if a sub-model element is selected).\\
\hline
\Alt + Cursor&
Moves the selected element on the drawing area.\\
\hline
\Alt + \Shift + Cursor&
Moves the selected element on the drawing area with pixel accuracy.\\
\hline
\Ctrl + \keystroke{C} or&
Copies the selected element or the selected elements to the clipboard.\\
\Ctrl + \Ins&~\\
\hline
\Ctrl + \keystroke{V} or&
Pastes the content of the clipboard to the drawing area.\\
\Shift + \Ins&~\\
\hline
\PgUp&
Moves the selected element on step forward.\\
\hline
\Ctrl + \PgUp&
Moves the selected element to the front.\\
\hline
\PgDown&
Moves the selected element on step backwards.\\
\hline
\Ctrl + \PgDown&
Moves the selected element to the back.\\
\hline
\Esc&
In insert edge mode: Cancels mode for inserting edges.
Else: Shows or hides the templates panel.\\
\hline
\end{tabular}

\begin{tabular}{|p{4cm}|p{11.5cm}|}
\hline
\textbf{Hotkey}&\textbf{Action}\\
\hline\hline
\keystroke{F2}&
Show or hide the templates panel.\\
\hline
\Ctrl + \keystroke{F2}&
Opens the model properties dialog.\\
\hline
\Ctrl + \keystroke{F3}&
Actives or deactivates the function for adding edges.\\
\hline
\keystroke{F12}&
Shows or hides the navigator panel.\\
\hline
\Ctrl + \keystroke{F12}&
Shows or hides the model overview.\\
\hline
\Ctrl + \keystroke{1}&
Show a quick fix context menu with suggestions for the selected element.\\
\hline
\Ctrl + \keystroke{3} oder&
Selects the quick access input field.\\
\Ctrl + \keystroke{E}&~\\
\hline
\Ctrl + \keystroke{F}&
Search for an element name or ID.\\
\hline
\Ctrl + \Shift + \keystroke{F}&
Search for any text.\\
\hline
\Ctrl + \keystroke{+} (Num-Pad)&
Increases the zoom factor.\\
\hline
\Ctrl + \keystroke{-} (Num-Pad)&
Decreases the zoom factor.\\
\hline
\Ctrl + \keystroke{*} (Num-Pad)&
Set the zoom factor to 100\,\%.\\
\hline
\Ctrl + \keystroke{/} (Num-Pad)&
Fly-out zoom.\\
\hline
\Ctrl + \keystroke{0} (Num-Pad)&
Centers the model on the drawing area.\\
\hline	
\Ctrl + \Home&
Scrolls to top left.\\	
\hline
\Ctrl +  \Shift + \keystroke{G}&
Select previous heatmap mode.\\
\hline
\Ctrl +  \Shift + \keystroke{H}&
Setup heatmap mode.\\
\hline
\Ctrl +  \Shift + \keystroke{I}&
Select next heatmap mode.\\
\hline
\end{tabular}



\part{Mouse interactions}

\begin{tabular}{|p{4.2cm}|p{2.3cm}|p{9cm}|}
\hline
\textbf{Mouse key}&\textbf{on}&\textbf{Action}\\
\hline\hline
Left&
Drawing area&
Unselects all elements.\\
\hline
Left&
Element&
Selects the element, all previous selections will be cleared.\\
\hline
\Shift + Left&
Element&
Adds the element to the selection.\\
\hline
Double click left&
Element&
Opens the edit dialog for the selected element.\\
\hline
Right&
Element&
Shows the context menu for the selected element.\\
\hline
Right&
Drawing area&
Shows the context menu for the drawing area.\\
\hline
Keep left pressed&
Element&
Moves the selected element or the selected elements on the drawing surface.\\
\hline
\Shift + Keep left pressed&
Element&
Moves the selected element or the selected elements with pixel accuracy on the drawing surface.\\
\hline
\Ctrl + Keep left pressed&
Element&
Place a copy of the selected element on the drawing area.\\
\hline
Keep left pressed&
Drawing area&
Drawing a frame for selecting elements.\\
\hline
Mouse wheel&
Drawing area&
Moves the drawing area vertically.\\
\hline
\Alt + Mouse wheel&
Drawing area&
Moves the drawing area horizontally.\\
\hline
\Ctrl + Mouse wheel&
Drawing area&
Changes the zoom factor.\\
\hline
Middle&
Drawing area&
Ends the mode for inserting connection edges.\\
\hline
Middle&
Element&
Starts the mode for inserting connection edges or selects an element for the connection.\\
\hline
\end{tabular}

\vskip1em
Mice where the mouse wheel can be pressed left and right are also supported. Pressing the mouse wheel to the left or right then moves the drawing area to the left or right (comparable to the mouse wheel action \Alt + mouse wheel).

\end{document}