
\section*{Abbruchrate}


Verfügen die Kunden an einer Bedienstation nur über ein begrenzte
Wartezeittoleranz, so gibt die Abbruchrate den Kehrwert der mittleren Wartezeittoleranz an.
Die Abbruchrate wird in der Warteschlangentheorie auch mit $\nu$ bezeichnet.

\section*{Abbruchzeit}


Hat ein Kunde an einer Bedienstation das Warten aufgegeben, da er
länger hätte warten müssen, als seine persönliche Wartezeittoleranz es zu ließ, so wird die Wartezeit
des Kunden in der Statistikansicht als seine Abbruchzeit ausgewiesen.

\section*{Abrüstzeit}


Der Begriff Abrüstzeit ist ein Pseudonym für die Nachbearbeitungszeit.
Die Nachbearbeitungszeit ist die Zeitspanne, die ein Bediener an einer Bedienstation
nach dem ab Abschluss einer Bedienung eines Kunden benötigt, bevor er wieder für weitere Bedienprozesse zur Verfügung steht.
Während der Nachbearbeitungszeit befindet sich der zuvor bediente Kunde bereits nicht mehr an der Bedienstation.

\section*{Allen-Cunneen-Näherungsformel}


Mit Hilfe der Allen-Cunneen-Formel können Näherungswerte für die Kenngrößen von Warteschlangenmodellen, in denen die
Zwischenankunftszeiten und die Bedienzeiten nicht notwendig exponentiell verteilt sind, berechnet werden. Während die
Erlang-Formeln, die exponentiell verteilte Zwischenankunftszeiten und die Bedienzeiten voraussetzen, exakte Ergebnisse
liefern, handelt es sich bei den Ergebnissen der Allen-Cunneen-Formel nur noch um Näherungswerte für die tatsächlichen
Kenngrößen. Die Eingangsverteilungen werden bei der Allen-Cunneen-Näherungsformel über die Erwartungswerte und die
Variationskoeffizienten charakterisiert, d.h. es muss der exakte Typ der Verteilungsfunktion nicht bekannt sein.

\section*{Ankunftsrate}


Die Ankunftsrate ist der Kehrwert der Zwischenankunftszeiten an einer Quelle.
In der Warteschlangentheorie wird die Ankunftsrate auch mit $\lambda$ bezeichnet.

\section*{Anzahl an Bedienern}


An jeder Bedienstation arbeiten ein oder mehrere Bediener,
die die eintreffenden Kunden bedienen. In der analytischen Warteschlangentheorie
wird die Anzahl an Bedienern mit c bezeichnet und in der Kendeall-Notation an dritter Stelle angegeben. Die Bediener
sind im analytischen Kontext fest einer Bedienstation zugeordnet. In einem Simulationsmodell muss eine solche
feste Zuordnung nicht bestehen. Hier können sich verschiedene Bedienstationen Bedienergruppen teilen.

\section*{Arbeitslast}


Die Arbeitslast ist der Quotient aus Ankunftsrate und Bedienrate. Rundet man die berechnete Arbeitslast zur
nächsten ganzen Zahl auf, so ergibt sich die minimale Anzahl an notwendigen Bedienern. Die Arbeitslast wird in der
Warteschlangentheorie auch mit a bezeichnet.

\section*{Auslastung}


Die in der Statistikansicht angezeigte Auslastung der Bedienergruppen gibt an,
zu welchem Anteil ihrer verfügbaren Zeit die Bediener jeweils mit der Bedienung von Kunden beschäftigt waren.
Die Auslastung ist folglich ein Wert zwischen 0 und 1. Die Auslastung wird in der
Warteschlangentheorie auch mit $\rho$ bezeichnet.

\section*{Autokorrelation}


Im Modelleigenschaften-Dialog kann eingestellt werden, dass während der
Simulation die Autokorrelation der Wartezeiten erfasst werden soll. Die Autokorrelation gibt an, wie stark
die Wartezeit eines Kunden von den Wartezeiten der unmittelbar vorherigen Kunden abhängt. Auf Basis der
Autokorrelation der Wartezeiten kann entschieden werden, wie viel Rest-Abhängigkeit für die Festlegung einer
Batch-Means Batch-Größe akzeptiert werden soll. Die Batch-Means-Methode ermöglicht es dann,
Konfidenzintervalle für die Kenngrößen in der Statistikansicht
auszuweisen.

\section*{Batch}


Treffen jeweils mehrere Kunden gleichzeitig an einer Quelle
ein oder aber werden mehrere Kunden gleichzeitig an einer Bedienstation
bedient, so spricht man von Batch-Ankünften bzw. Batch-Bedienungen. Kunden können auch explizit an einer
Zusammenfassen-Station zu einem Batch zusammengefasst werden.

\section*{Batch-Means-Methode}


Die Batch-Means-Methode ermöglicht es, Konfidenzintervalle zu den Kenngrößen in der
Statistikansicht auszuweisen. Dies ist nicht auf direktem
Wege möglich, da z.B. die Wartezeiten der unmittelbar aufeinander folgenden Kunden abhängig sind
und dies für die Konfidenzintervallberechnung nicht zulässig ist. Die zu verwendende Batch-Größe
für die Batch-Means-Methode kann über die Autokorrelations-Erfassung ermittelt werden. Beide
Verfahren können im Modelleigenschaften-Dialog konfiguriert werden.

\section*{Bedienrate}


Die Bedienrate ist der Kehrwert der mittleren Bedienzeit an einer Bedienstation.
Die Bedienrate wird in der Warteschlangentheorie auch mit $\mu$ bezeichnet.

\section*{Bedienregel}


Die Bedienregel gibt an, welcher von mehreren wartenden Kunden als nächstes an einer
Bedienstation bedient werden soll. Die konkreten
Bedienregeln werden an einer Bedienstation über die Prioritäten abgebildet.

\section*{Bedienzeit}


Über die Bedienzeit kann an einer Bedienstation eingestellt werden,
wie lange der Prozess der Bedienung eines Kunden (oder im Fall von Batch-Bedienungen eines Kunden-Batches)
durch einen Bediener dauern soll.
In der Statistikansicht werden die Bedienzeiten mit \textbf{S} bezeichnet.

\section*{(Umlauf-)Bestand}


Der Bestand, auch als Umlaufbestand oder als Workunits in process (WIP) bezeichnet,
gibt an, wie viele Kunden bzw. Werkstücke sich insgesamt im System befinden. Dies umfasst
also sowohl die wartenden als auch die in Bedienung befindlichen Werkstücke.
In der analytischen Warteschlangentheorie wird der
Bestand meist mit N bezeichnet. Der mittlere Bestand wird dann mit E[N] beschrieben.

\section*{Divergierende Fertigung}


Erfolgt die Produktion eines gemeinsamen Vorprodukts auf einer zentralen Fertigungslinie
und teilt sich diese später in Einzellinien für die verschiedenen Produkte auf, so spricht
man von einer divergierenden Fertigung. Divergierende Fertigungen sind aus
warteschlangentheoretischer Sicht gut beherrschbar: An dem Punkt, an dem die Ausdifferenzierung erfolgt, beginnen neue
Teilsystem mit jeweils einem eigenständigen Ankunftsstrom. Die Teilsysteme können unabhängig
voneinander analysiert und optimiert werden.

\section*{Durchlaufzeit}


Die Durchlaufzeit ist ein Pseudonym für die Verweilzeit.
Die in der Statistikansicht mit \textbf{V} bezeichnete Verweilzeit
gibt an, wie viel Zeit ein Kunde insgesamt im System oder an einer Station verbracht hat.

\section*{Economy of Scale}


Die Economy of Scale oder auch der positive Skaleneffekt beschreiben den bei vielen Produktionssystemen
vorliegenden Effekt, dass in einem größeren System die Produktionsstückkosten geringer sind als in
einem kleineren System. Die Ursache dafür kann darin liegen, dass sich die Fixkosten auf mehr Einheiten
verteilen, oder aber, dass sich Schwankungen bedingt durch stochastischen Zwischenankunfts- und Bedienzeiten
bei größeren System besser gegenseitig ausgleichen können.

\section*{Einschwingphase}


Bevor die Warte- und Bedienzeiten für die Statistik gezählt werden, können eine bestimmte Anzahl an Kunden
das System durchlaufen, die nicht für die Statistik gezählt werden. Im
Modelleigenschaften-Dialog kann eingestellt werden, wie lang diese
Einschwingphase sein soll.

\section*{Erlang-Formeln}


Mit Hilfe der Erlang-B- und Erlang-C-Formeln lassen sich einfache Warteschlangenmodelle exakt berechnen.
Im Warteschlangenrechner könne die Werte dieser Formeln direkt
berechnet werden.

\section*{Experimentierfähigkeit}


Ein Modell wird als experimentierfähig bezeichnet, wenn damit nicht nur der Ist-Zustand eines
realen Systems nachgestellt werden kann, sondern auch Was-wäre-wenn-Fragen zu bislang in der Realität
noch nicht aufgetretenen Szenarien untersucht  werden können. Während ein Modell zur Nachstellung
früherer Ist-Zustände mit historischen Detaildaten parametrisiert werden kann, ist
dies für ein experimentierfähiges Modell nicht zielführend. Hier muss ganz bewusst abstrahiert
werden, um globale Änderungen an dem Modell vornehmen zu können.

\section*{Exzess (Wölbung)}


Der Exzess ist ein Maß für die Wölbung einer Messreihe.
Ein Exzess von 0 bedeutet, dass die Wahrscheinlichkeitsverteilung bzw. die Messreihe genauso spitz wie eine Normalverteilung ist.
Je größer der Exzess-Wert ist, desto spitzer ist die Dichte.

\section*{FIFO (FCFS)}


Die Bedienreihenfolge FIFO (für First in first out) bzw. FCFS (für First come first serve) bedeutet,
dass die Kunden in Ankünftsreihenfolge bedient werden. Dies entspricht gedanklich einer klassischen Warteschlange.
Während die Bedienreihenfolge keine Auswirkungen auf die mittlere Wartezeit besitzt, verändert sich
die Streuung der Wartezeiten je nach gewählter Bedienreihenfolge. FIFO führt dabei zu einer minimalen Streuung.
Über eine formelbasierte Priorisierung der Kunden können neben den klassischen Bedienfolgen FIFO und LIFO auch
beliebige weitere Strategien abgebildet werden.

\section*{Flussgrad}


Der in der Statistikansicht ausgewiesene Flussgrad ist
das Verhältnis von Verweilzeit zu Bedienzeit. Der Flussgrad ist minimal 1. Ein Flussgrad
von 1 bedeutet, dass keine Wartezeiten angefallen sind. Ein Flussgrad von 2 bedeutet, dass
die Kunden genauso lange warten mussten, wie ihre Bedienungen gedauert haben usw.

\section*{Geschlossenes Warteschlangennetz}


In einem geschlossenen Warteschlangennetz zirkuliert eine endliche, feste Anzahl an Kunden.
Im Gegensatz zu einem offenen Warteschlangennetz, verfügt ein geschlossenes Warteschlangennetz
meist über keinen Ausgang. Da die Erfassung der Zeitdauern
auf Kundenbasis normalerweise dann erfolgt, wenn ein Kunde das System verlässt, ist es für die Erfassung
der Zeitdauern in einem geschlossenen Warteschlangennetz notwendig, im
Modelleigenschaften-Dialog anzugeben, dass auch Kunden, die am
Simulationsende das System noch nicht verlassen haben, erfasst werden sollen.

\section*{Jockeying}


Wenn Kunden, nach dem sie sich an der Warteschlangen an einer Bedienstation angestellt haben,
diese ggf. wieder verlassen, um sich an einer anderen Bedienstation, an der die Warteschlange
kürzer ist, erneut anzustellen, so spricht man davon, dass Jockeying zwischen den Warteschlangen auftritt.

\section*{Kampagnen}


Wird versucht, möglichst viele Kunden desselben Typs an einer Bedienstation
in unmittelbarer Folge zu bedienen, so spricht man von einer Kampagnenfertigung. Eine Kampagnenfertigung ist
besonders dann von Interesse, wenn bei dem Wechsel von einem Kundentyp zu einem anderen an der Bedienstation
Rüstzeiten aufteten und es diese zu vermeiden gilt. Bei der Kampagnenfertigung wird die Priorisierung der
wartenden Kunden so abgeändert, dass zunächst nur wartende Kunden des bisherigen Kundentyps für die nächste
Bedienung in Frage kommen. Erst wenn keine Kunden des aktuellen Typs mehr in der Warteschlange vorhanden sind,
wird die Suche auf alle Kundentypen ausgeweitet. Diese Veränderung in der Bedienreihenfolge führt zu einer
erhöhten Streuung der Wartezeiten verglichen mit einer über alle Kundentypen wirksamen FIFO-Bedienreihenfolge.

\section*{Kendall-Notation}


Zur Beschreibung der Eigenschaften von Warteschlangenmodellen wurde 1953 die Kendall-Notation eingeführt.
Diese beschreibt die Verteilungsfunktionen der Zwischenankunftszeiten und der Bedienzeiten, der Anzahl an
Bedienern, der Systemgröße sowie optional weiterer Eigenschaften. Der Ausdruck M/M/c steht z.B. für ein
Bediensystem mit jeweils exponentiell vertteilten Zwischenankunfts- und Bedienzeiten ("M") und c Bedienern.
Weitere übliche Bezeichnungen statt "M" sind "G" für allgemeine Verteilungen und "D" für deterministische
Zeitdauern. Ist nur ein Bediener im System vorhanden, so wird eine "1" statt des "c" geschrieben. Bei einer
begrentzen Systemgröße wird nach der Anzahl an Bedienern diese getrennt durch einen weiteren Schrägstrich
angegeben; fehlt sie, so wird von einem unbegrenzten System ausgegangen. Als Ergänzungen können Populationsgrößen,
Wartezeittoleranzen der Kunden usw. angegeben werden. Warteschlangennetze werden beschrieben, in dem mehrere
dieser Einzelnotationen durch Pfeile verbunden werden.

\section*{Konfidenzintervalle}


Die einzelnen Messwerte, die sich in der Simulation ergeben und auf
deren Basis z.B. Mittelwerte
berechnet werden, stellen nur einige der möglichen Realisierungen,
die sich in der Realität ergeben können dar. Daher wird häufig die
Frage gestellt, wie nah der per Simulation ermittelte Mittelwert
an dem tatsächlichen Mittelwert des Modells liegt. Hierüber geben
die Konfidenzintervalle Aufschluss.


Ein Konfidenzintervall ist der Zahlenbereich um einen gemessenen Mittelwert
herum, in dem sich der tatsächliche Mittelwert mit einer bestimmten
Wahrscheinlichkeit befindet. Je kleiner das Konfidenzintervall gewählt
werden soll, desto geringer ist die Wahrscheinlichkeit, dass der tatsächliche
Mittelwert in diesem Intervall liegt oder umso mehr Messwerte sind erforderlich,
um diese Bedienung zu erfüllen. Dasselbe gilt für die Wahrscheinlichkeit,
dass der tatsächliche Mittelwert in dem Intervall liegt (auch Konfidenzniveau genannt):
Je höher das Konfidenzniveau gewählt wird, desto größer fällt das zugehörige
Konfidenzintervall aus oder wiederum desto mehr Messwerte sind erforderlich.

\section*{Konvergierende Fertigung}


Werden mehrere Vorprodukte auf eigenen Fertigungslinien hergestellt und dann an einer Station
zusammengeführt, so spricht man von einer konvergierenden Fertigung. Konvergierende Fertigungen
sind aus warteschlangentheoretischer Sicht äußerst anspruchsvoll: Erfolgt keine Steuerung der
zusammenlaufenden Stränge, so kommt es - selbst wenn beide Stränge im Mittel gleich viele
Komponenten pro Zeiteinheit liefern können - an der Station, an der die Zusammenführung erfolgt,
immer zu sehr langen Warteschlangen. Eine konvergierende Fertigung setzt
(wenn diese längere Zeit ohne regelmäßige Stillstände wie Wochenenden) immer eine Steuerung z.B.
in Form einer Pull-Produktion voraus.

\section*{Kunden im System}


Die Anzahl an Kunden im System gibt an, wie viele Kunden sich insgesamt in dem Warteschlangenmodell
befinden. Die umfasst die Kunden, die momentan bedient werden, und die Kunden, die momentan warten.
Der Bestand ist ein Synonym für die Anzahl an Kunden im System.
In der analytischen Warteschlangentheorie wird die
Anzahl an Kunden im System meist mit N bezeichnet. Die mittlere Anzahl an Kunden im System
wird dann mit E[N] beschrieben.

\section*{LIFO (LCFS)}


Die Bedienreihenfolge LIFO (für Last in first out) bzw. FCFS (für Last come first serve) bedeutet,
dass die Kunden in umgekehrter Ankünftsreihenfolge bedient werden. Dies entspricht gedanklich einem Stapel
auf den oben aufgelegt wird und von dem auch von oben gezogen wird.
Während die Bedienreihenfolge keine Auswirkungen auf die mittlere Wartezeit besitzt, verändert sich
die Streuung der Wartezeiten je nach gewählter Bedienreihenfolge. LIFO führt dabei zu einer maximalen Streuung.
Über eine formelbasierte Priorisierung der Kunden können neben den klassischen Bedienfolgen FIFO und LIFO auch
beliebige weitere Strategien abgebildet werden.

\section*{Longest job first}


Ist diese Priorisierung gewählt, so werden Kunden mit langen Bediendauern bevorzugt bedient.
Kunden, deren Bedienung nur wenig Zeit in Anspruch nehmen wird, müssen entsprechend länger warten.
Die Priorisierung nach langer Bediendauer führt zu (im Vergleich zu einer FIFO-Priorisierung)
längeren mittleren Wartezeiten über alle Kunden und erhöht außerdem die Streuung der Wartezeiten
erheblich. Voraussetzung für die Anwendung dieser Priorisierung ist es, dass die Bediendauern
der Kunden bereits zu dem Zeitpunkt, an dem sich die Kunden an die Warteschlange anstellen, bekannt sind.

\section*{Median}


Im Zusammenhang mit dem Mittelwert
ist auch häufig vom Median die Rede. Während der Mittelwert die Vorteile
mit sich bringt, dass er leicht berechnet werden kann und in vielen weiteren
Formeln zur Berechnung statistischer Kennzahlen benötigt wird, besitzt er
den Nachteil, dass er anfällig für Ausreißer ist. Wird in der Messreihe
ein einziger Wert massiv erhöht, so hat dies große Auswirkungen auf den
Mittelwert - obwohl sich bis auf den einen einzigen Ausreißer der Rest
der Messreihe überhaupt nicht verändert hat. Der Median hingegen gibt
"den Wert in der Mitte" einer Messreihe an. Wird der Wert eines einzelnen
Ausreißers verändert, so hat dies keine Auswirkungen auf den Median.

\section*{Mittelwert}


Der Mittelwert ist die am häufigsten verwendete statistische Kenngröße.
Er gibt den Durchschnitt der betrachteten Werte an und wird gebildet, in
dem alle Messwerte aufsummiert werden und durch die Anzahl der Werte
dividiert wird.

\section*{Nacharbeit}


Erfüllt ein Produkt am Ende eines Produktionsprozesses nicht die Qualitätsanforderungen,
so wird es in vielen Fällen (wenn dies technisch möglich ist und das Produkt nicht notwendig
sofort zu Ausschuss wird) zur Nacharbeit erneut in die Produktion geleitet. Je nach Komplexität
des Produktes und des Fertigungsprozesses kann diese Nacharbeit auf separaten, nur dafür
vorgesehenen Anlagen erfolgen oder aber direkt innerhalb des regulären Produktionsprozesses.
Erfolgt die Nacharbeit an denselben Bedienstationen, die auch zur regulären Fertigung eingesetzt
werden, so erhöht die Nacharbeit die Auslastung an diesen Stationen. Daher muss der  Anteil an
Werkstücken, der im Mittel einer Nacharbeit bedarf, bei der Planung der Kapazität des
Produktionsprozesses berücksichtigt werden.

\section*{Nachbearbeitungszeit}


Die Nachbearbeitungszeit ist die Zeitspanne, die ein Bediener an einer Bedienstation
nach dem ab Abschluss einer Bedienung eines Kunden benötigt, bevor er wieder für weitere Bedienprozesse zur Verfügung steht.
Während der Nachbearbeitungszeit befindet sich der zuvor bediente Kunde bereits nicht mehr an der Bedienstation.

\section*{Offenes Warteschlangennetz}


Ein offenes Warteschlangennetz stellt den Normalfall für ein Modell im Warteschlangensimulator dar: Kunden
treffen an einer oder mehreren Quellen ein, werden an einer oder
mehreren Bedienstationen bedient und verlassen das System am Ende
über einen oder mehrere Ausgänge. Einen Gegenentwurf dazu stellen
geschlossene Warteschlangennetze dar, in denen eine endliche, feste Anzahl an Kunden fortwährend zirkuliert.

\section*{Priorisierung}


Über die Priorisierung der Kunden an einer Bedienstationen wird
festgelegt, welcher der wartenden Kunden als nächstes bedient wird, wenn ein Bediener verfügbar wird.
Es wird jeweils derjenige Kunde mit der höchsten Priorität als nächstes bedient.

\section*{Pull-Produktion}


Bei der Pull-Produktion führt eine Station nur dann einen Bedienvorgang aus, wenn die jeweilige
Folgestation einen Kunden bzw. ein Werkstück anfordert, d.h. die Kunden werden von den jeweiligen
Folgestationen durch die Produktion gezogen. Auf diese Weise können die Bestände an den einzelnen
Stationen begrenzt werden. Allerdings wird zum einen eine spezielle Steuerung erfordert und zum
anderen muss sichergestellt werden, dass dennoch genug Puffer vorhanden sind, so dass die Bedienstationen
auch in diesem Fall nicht unnötig in Leerlauf geraten.

\section*{Push-Produktion}


Die Push-Produktion stellt den Normalfall in einem Warteschlangenmodell dar. Ist eine Bedienstation
mit der Bedienung eines Kunden fertig, so schiebt sie den Kunden zur nächsten Station. Die Weiterleitung
des Kunden ist dabei unabhängig davon, wie viele Kunden an der Folgestation bereits warten.

\section*{Quantile}


Die Quantile geben Aufschluss darüber, wie sich die konkreten Werte einer
Messreihe über den Wertebereich verteilen und stellen damit eine Ergänzung
zur Standardabweichung aber
auch zum Mittelwert dar.


Quantile geben an, ein wie großer Anteil der Messwerte jeweils kleiner oder gleich
einem bestimmten Wert ist. Beträgt das 75%-Quantil der Wartezeiten z.B. 100 Sekunden,
so bedeutet dies, dass 75% der Kunden höchstens 100 Sekunden warten mussten.

\section*{Rüstzeit}


Rüstzeiten können an Bedienstationenen auftreten, wenn beim Wechsel
von einem Kundentyp zu einem anderen die Station als solches umkonfiguriert werden muss. Die Rüstzeit ist
der jeweiligen Bedienzeit eines Kunden vorgelagert.

\section*{Schiebende Fertigung}


Die schiebende Fertigung, auch Push-Produktion genannt, stellt den Normalfall in einem Warteschlangenmodell dar.
Ist eine Bedienstation mit der Bedienung eines Kunden fertig, so schiebt sie den Kunden zur nächsten Station.
Die Weiterleitung des Kunden ist dabei unabhängig davon, wie viele Kunden an der Folgestation bereits warten.

\section*{Schiefe}


Die Schiefe ist ein Maß für die Asymmetrie einer Messreihe.
Eine Schiefe von 0 bedeutet, dass die Wahrscheinlichkeitsverteilung bzw. die Messreihe symmetrisch ist.
Messreihen mit einer Schiefe kleiner als 0 werden als \textbf{linksschief} bezeichnet,
Messreihen mit einer Schiefe größer als 0 als \textbf{rechtsschief}.

\section*{Service-Level}


Der Service-Level ist eine zur mittleren Wartezeit alternative Methode der Erfassung der Wartezeiten,
die besonders im Kundenserviceumfeld verwendet wird. Der Service-Level gibt den Anteil der Kunden,
der höchstens eine vorab festgelegte Zeitdauer warten musste, an.

\section*{Shortest job first}


Ist diese Priorisierung gewählt, so werden Kunden mit kurzen Bediendauern bevorzugt bedient.
Kunden, deren Bedienung länger dauern wird, müssen entsprechend länger warten.
Die Priorisierung nach kurzer Bediendauer führt zu kürzeren mittleren Wartezeiten über alle Kunden,
erhöht allerdings die Streuung der Wartezeiten erheblich.
Voraussetzung für die Anwendung dieser Priorisierung ist es,
dass die Bediendauern der Kunden bereits zu dem Zeitpunkt,
an dem sich die Kunden an die Warteschlange anstellen, bekannt sind.

\section*{SIRO}


Die Bedienreihenfolge SIRO (für Service in random order) bedeutet, dass die Kunden in zufälliger Reihenfolge
der Warteschlange entnommen werden. Während die Bedienreihenfolge keine Auswirkungen auf die mittlere Wartezeit besitzt,
verändert sich die Streuung der Wartezeiten je nach gewählter Bedienreihenfolge. Die Streuung liegt im Falle von SIRO
zwischen den Werten für FIFO und LIFO.

\section*{Skaleneffekt}


Die Economy of Scale oder auch der positive Skaleneffekt beschreiben den bei vielen Produktionssystemen
vorliegenden Effekt, dass in einem größeren System die Produktionsstückkosten geringer sind als in
einem kleineren System. Die Ursache dafür kann darin liegen, dass sich die Fixkosten auf mehr Einheiten
verteilen, oder aber, dass sich Schwankungen bedingt durch stochastischen Zwischenankunfts- und Bedienzeiten
bei größeren System besser gegenseitig ausgleichen können.

\section*{Standardabweichung}


Neben dem Mittelwert sind
Standardabweichung und Varianz zusammen die zweite wichtige Kenngröße
einer Messreihe. Die beiden Werte geben an, wie weit die konkreten
Werte im Durchschnitt vom Mittelwert abweichen. Eine Standardabweichung
von 0 bedeutet, dass die Messreihe nur aus einem Wert besteht. Negative
Standardabweichungen sind nicht möglich. Je größer die Standardabweichung
ist, desto stärker variieren die Messwerte einer Messreihe.

\section*{Ungeduld}


Sind die Kunden an einer Bedienstationen nur bereit, begrenzt lange auf
ihre Bedienung zu warten, so spricht man von der Ungedulg der Kunden. Wie lange die Kunden in diesem Fall bereit
sind zu warten, bevor sie das Warten aufgeben und die Station unbedient verlassen, wird über die Wartezeittoleranzen
der Kunden festgelegt.

\section*{Varianz}


Bei der Varianz handelt es sich um die quadrierte Standardabweichung.

\section*{Variationskoeffizienten}


Der Variationskoeffizient stellt eine normierte Fassung der
Standardabweichung
dar. Genauso wie die Standardabweichung nimmt der Variationskoeffizient
nie Werte kleiner als 0 an und ein Wert von 0 bedeutet, dass die
Messreihe nur aus einem Wert besteht.

\section*{Verweilzeit}


Die in der Statistikansicht mit \textbf{V} bezeichnete Verweilzeit
gibt an, wie viel Zeit ein Kunde insgesamt im System oder an einer Station verbracht hat.

\section*{Warm-up-Phase}


Bevor die Warte- und Bedienzeiten für die Statistik gezählt werden, können eine bestimmte Anzahl an Kunden
das System durchlaufen, die nicht für die Statistik gezählt werden. Im
Modelleigenschaften-Dialog kann eingestellt werden, wie lang diese
Warm-up-Phase, die auch Einschwingphase genannt wird, sein soll.

\section*{Warteabbruch}


Besitzt ein Kunde nur eine begrenzte Wartezeittoleranz und überschreitet seine bisherige Wartezeit
seine individuelle Wartezeittoleranz, so bricht er das warten ab, ohne bedient worden zu sein.

\section*{Warteraum}


Ein Warteraum ist integraler Bestandteil einer jeder Bedienstation.
Im Warteraum wartet ein Kunde, bevor ein Bediener verfügbar, um ihn zu bedienen.

\section*{Warteschlange}


Wenn ein Kunde an einer Bedienstation eintrifft und
nicht sofort bedient werden kann, weil momentan alle Bediener belegt sind, muss der Kunde zunächst warten.
Die wartenden Kunden bilden die Warteschlange vor dem Bedienschalter. Wird ein Bediener frei, so wird
der wartende Kunde mit der höchsten Priorität der Warteschlange entnommen und als nächstes bedient.
In der analytischen Warteschlangentheorie wird die
Warteschlangenlänge meist mit NQ bezeichnet.

\section*{Wartezeit}


Wartezeiten der Kunden entstehen insbesondere an Bedienstation.
Ein Kunde muss immer dann warten, wenn momentan kein Bediener verfügbar ist, der ihn bedienen könnte.
In der Statistikansicht werden die Wartezeiten mit \textbf{W} bezeichnet.

\section*{Wartezeittoleranz}


Die Wartezeittoleranz gibt an, wie lange die Kunden an einer Bedienstation
bereit sind auf ihre Bedienung zu warten. Es muss keine Wartezeittoleranz definiert werden; in diesem
Fall sind die Kunden bereit, beliebig lange zu warten. Die eine endliche Wartezeittoleranz definiert und überschreitet
ein Kunde diese Wartezeittoleranz, so bricht er den Wartevorgang ab und verlässt die Bedientation, ohne bedient
worden zu sein.

\section*{Wiederholer}


Wenn Kunden das Warten an einer Bedienstation aufgrund von
zu langen Wartezeiten aufgeben und später einen neuen Anlauf tätigen, bedient zu werden,
so spricht vom von Wiederholern. Wiederholer treten insbesondere bei einer hohen Auslastung einer
Bedienstation auf und erhöhen in diesem Fall die Last noch weiter. Diese Effekte lassen sich besonders
in Systemen, in denen Menschen bedient werden (wie z.B. in Callcentern) beobachten.

\section*{Ziehende Fertigung}


Bei der ziehenden Fertigung, auch der Pull-Produktion genannt, führt eine Station nur dann einen Bedienvorgang aus, wenn die
jeweilige Folgestation einen Kunden bzw. ein Werkstück anfordert, d.h. die Kunden werden von den jeweiligen
Folgestationen durch die Produktion gezogen. Auf diese Weise können die Bestände an den einzelnen
Stationen begrenzt werden. Allerdings wird zum einen eine spezielle Steuerung erfordert und zum
anderen muss sichergestellt werden, dass dennoch genug Puffer vorhanden sind, so dass die Bedienstationen
auch in diesem Fall nicht unnötig in Leerlauf geraten.

\section*{Zwischenankunftszeiten}


Die Zwischenankunftszeiten geben an einer Quelle
die Zeitspannen zwischen den Ankünften zweier aufeinander folgender Kunden an. Je länger die
Zwischenankunftszeiten sind, desto weniger Kunden treffen pro Zeiteinheit in dem System ein.