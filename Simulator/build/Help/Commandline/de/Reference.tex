\section{\texttt{Benchmark}}

\textbf{Führt einen Geschwindigkeitstest des Rechners aus.}

Dieser Befehl erwartet einen oder keine weiteren Parameter.\\
Wird ein Parameter übergeben, so gibt dieser entweder die maximale Thread-Anzahl an,\\
die verwendet werden sollen, oder den Dateinamen des zu verwendenden Modells.

\section{\texttt{BuildCLIReference}}

\textbf{Generiert aus der Kommandozeilebefehle-Hilfe LaTeX-Code.}

Der Befehl erwartet zwei Parameter:\\
1. Sprache "`de"' oder "`en"'\\
2. Ausgabeverzeichnis für die LaTeX-Dateien

\section{\texttt{BuildElementGroupImages}}

\textbf{Erzeugt pro Elementgruppe ein Bild im Desktop-Ordner.}

Der Befehl erwartet einen Parameter:\\
1. Sprache "`de"' oder "`en"'

\section{\texttt{BuildLaTeXReference}}

\textbf{Generiert aus der HTML-Hilfe zu den Stationen LaTeX-Code.}

Der Befehl erwartet zwei Parameter:\\
1. Sprache "`de"' oder "`en"'\\
2. Ausgabeverzeichnis für die LaTeX-Dateien und Bilder

\section{\texttt{Filter}}

\textbf{Datei mit Statistikergebnissen filtern.}

Dieser Befehl erwartet genau drei weitere Parameter:\\
1. Name der Eingabe-Statistik-Datei\\
2. Name der Eingabe-Filterkonfigurationsdatei\\
3. Name der Ausgabedatei, die die Filterergebnisse aufnehmen soll\\
Wenn die Ausgabedatei bereits existiert, werden die neuen Ergebnisse angehängt.\\
Ansonsten wird die Datei neu angelegt.

\section{\texttt{GC}}

\textbf{Listet die aktiven Garbage Collectors auf.}

Dieser Befehl erwartet keine weiteren Parameter.

\section{\texttt{Hilfe}}

\textbf{Zeigt diese Hilfe an.}

Dieser Befehl erwartet einen oder keine weiteren Parameter.\\
Wird ein Befehl als zusätzlicher Parameter angegeben, so wird die Hilfe zu diesem Befehl angezeigt.\\
Ansonsten wird die Hilfe zu allen Befehlen angezeigt.

\section{\texttt{Optimierung}}

\textbf{Führt eine Modell-Optimierung durch.}

Dieser Befehl erwartet genau zwei weitere Parameter:\\
1. Eingabe-Modell-Datei\\
2. Optimiererkonfigurations-Datei\\
Beide Eingabedateien müssen existieren.

\section{\texttt{Parameterreihe}}

\textbf{Führt eine Parameterreihen-Simulation durch.}

Dieser Befehl erwartet genau zwei weitere Parameter:\\
1. Eingabe-Parameterreihen-Datei\\
2. Ausgabe-Parameterreihen-Datei\\
Die Eingabedatei muss existieren.\\
Die Ausgabedatei darf nicht existieren.

\section{\texttt{ParameterreiheTabelle}}

\textbf{Exportiert die Ergebnistabelle einer Parameterreihe.}

Dieser Befehl erwartet genau zwei weitere Parameter:\\
1. Eingabe-Parameterreihen-Datei\\
2. Ausgabe-Tabellendatei\\
Die Eingabedatei muss existieren.\\
Die Ausgabedatei darf nicht existieren.

\section{\texttt{ParameterreiheVarianzanalyse}}

\textbf{Erstellt eine Parameterreihen-Konfiguration für eine Varianzanalyse}

Dieser Befehl erwartet drei oder vier weitere Parameter:\\
1. Eingabe-Modell-Datei\\
2. Ausgabe-Parameterreihen-Datei\\
3. Anzahl an Wiederholungen des Modells\\
4. Neuer Wert für die Anzahl an Ankünften (optional)\\
Die Eingabedatei muss existieren.\\
Die Ausgabedatei darf nicht existieren.

\section{\texttt{Reset}}

\textbf{Konfiguration zurücksetzen.}

Setzt die Konfiguration des Simulators auf den Auslieferungszustand zurück.\\
Lediglich die eingetragenen Lizenzschlüssel bleiben erhalten.

\section{\texttt{Server}}

\textbf{Simulator als Rechenserver starten.}

Dieser Befehl erwartet 0 bis 2 weitere Parameter.\\
Werden entsprechend viele Parameter angegeben, so haben diese folgende Bedeutungen:\\
1. Zu verwendender Port\\
2. Passwort für die verschlüsselte Datenübertragung\\
Wird kein Parameter übergeben, so wird der Port per Kommandozeile abgefragt.

\section{\texttt{ServerLimited}}

\textbf{Simulator als Rechenserver starten.}

Dieser Befehl erwartet 0 bis 2 weitere Parameter.\\
Werden entsprechend viele Parameter angegeben, so haben diese folgende Bedeutungen:\\
1. Zu verwendender Port\\
2. Passwort für die verschlüsselte Datenübertragung\\
Wird kein Parameter übergeben, so wird der Port per Kommandozeile abgefragt.\\
Der Server begrenzt die Anzahl an gleichzeitigen Anfragen basierend auf der Anzahl\\
an verfügbaren CPU-Kernen.

\section{\texttt{ServerWeb}}

\textbf{Simulator als webbasierten Rechenserver starten.}

Dieser Befehl erwartet einen weiteren Parameter:\\
den zu verwendenden Port.

\section{\texttt{SetMaxThreads}}

\textbf{Maximalanzahl an Threads festlegen}

Die Funktion erwartet als Parameter die erlaubte Maximalanzahl an Threads.\\
Werte kleiner oder gleich 0 werden als unlimitiert interpretiert.

\section{\texttt{SetNUMAMode}}

\textbf{NUMA-Modus festlegen}

Diese Funktion erwartet eine "`1"' oder eine "`0"' als Parameter.

\section{\texttt{Simulation}}

\textbf{Führt einen einzelnen Simulationslauf durch.}

Dieser Befehl erwartet genau zwei weitere Parameter:\\
1. Eingabe-Modell-Datei\\
2. Ausgabe-Statstik-Datei\\
Die Eingabedatei muss existieren, die Ausgabedatei darf nicht existieren.

\section{\texttt{Version}}

\textbf{Gibt die aktuelle Versionsnummer aus.}

Gibt lediglich die Versionsnummer des Warteschlangensimulators aus.\\
Führt keine weiteren Verarbeitungen durch.

\section{\texttt{Zusammenfassung}}

\textbf{Exportiert einen Teil oder die gesamten Simulationsergebnisse für ein Modell.}

Dieser Befehl erwartet genau drei weitere Parameter:\\
1. "`Inline"`, "`Einzeldateien"`, "`Liste"`, "`Text"`, "`PDF"' oder ein Listeneintrag je nach dem, ob\\
a) ein HTML-Report mit eingebetteten Bildern,\\
b) ein HTML-Report mit Bildern in separaten Dateien,\\
c) eine Übersicht über alle verfügbaren Einzeldokumente\\
d) ein DOCX-Report,\\
e) ein PDF-Report oder\\
f) ein bestimmtes Einzeldokument ausgegeben werden soll.\\
2. Dateiname der Eingabedatei\\
3. Dateiname der Ausgabedatei

