\section{\texttt{Ausgabetabelle}}

\textbf{Verarbeitet eine an einer Speichern+Ausgang-Station erzeugte Tabelle.}

Dieser Befehl erwartet zwei weitere Parameter:\\
1. Eingabe-Tabellendatei\\
2. Eingabe-Tabelledatei\\
Die Eingabedatei muss existieren, die Ausgabedatei darf nicht existieren.

\section{\texttt{Export}}

\textbf{Exportiert ein Modell.}

Dieser Befehl erwartet genau zwei weitere Parameter:\\
1. Name der Eingabe-Datei\\
2. Name der Ausgabedatei, die die Exportergebnisse aufnehmen soll\\
Die Eingabedatei muss existieren, die Ausgabedatei darf nicht existieren.

\section{\texttt{Filter}}

\textbf{Datei mit Statistikergebnissen filtern.}

Dieser Befehl erwartet genau drei weitere Parameter:\\
1. Name der Eingabe-Statistik-Datei\\
2. Name der Eingabe-Filterkonfigurationsdatei\\
3. Name der Ausgabedatei, die die Filterergebnisse aufnehmen soll\\
Wenn die Ausgabedatei bereits existiert, werden die neuen Ergebnisse angehängt.\\
Ansonsten wird die Datei neu angelegt.

\section{\texttt{Hilfe}}

\textbf{Zeigt diese Hilfe an.}

Dieser Befehl erwartet einen oder keine weiteren Parameter.\\
Wird ein Befehl als zusätzlicher Parameter angegeben, so wird die Hilfe zu diesem Befehl angezeigt.\\
Ansonsten wird die Hilfe zu allen Befehlen angezeigt.

\section{\texttt{Interaktiv}}

\textbf{Startet den interaktiven Modus.}

Dieser Befehl erwartet keine weiteren Parameter.

\section{\texttt{Optimierung}}

\textbf{Führt eine Modell-Optimierung durch.}

Dieser Befehl erwartet genau zwei weitere Parameter:\\
1. Eingabe-Modell-Datei\\
2. Optimiererkonfigurations-Datei\\
Beide Eingabedateien müssen existieren.

\section{\texttt{Parameterreihe}}

\textbf{Führt eine Parameterreihen-Simulation durch.}

Dieser Befehl erwartet genau zwei weitere Parameter:\\
1. Eingabe-Parameterreihen-Datei\\
2. Ausgabe-Parameterreihen-Datei\\
Die Eingabedatei muss existieren.\\
Die Ausgabedatei darf nicht existieren.

\section{\texttt{ParameterreiheTabelle}}

\textbf{Exportiert die Ergebnistabelle einer Parameterreihe.}

Dieser Befehl erwartet genau zwei weitere Parameter:\\
1. Eingabe-Parameterreihen-Datei\\
2. Ausgabe-Tabellendatei\\
Die Eingabedatei muss existieren.\\
Die Ausgabedatei darf nicht existieren.

\section{\texttt{ParameterreiheVarianzanalyse}}

\textbf{Erstellt eine Parameterreihen-Konfiguration für eine Varianzanalyse}

Dieser Befehl erwartet drei oder vier weitere Parameter:\\
1. Eingabe-Modell-Datei\\
2. Ausgabe-Parameterreihen-Datei\\
3. Anzahl an Wiederholungen des Modells\\
4. Neuer Wert für die Anzahl an Ankünften (optional)\\
Die Eingabedatei muss existieren.\\
Die Ausgabedatei darf nicht existieren.

\section{\texttt{Server}}

\textbf{Simulator als Rechenserver starten.}

Dieser Befehl erwartet 0 bis 2 weitere Parameter.\\
Werden entsprechend viele Parameter angegeben, so haben diese folgende Bedeutungen:\\
1. Zu verwendender Port\\
2. Passwort für die verschlüsselte Datenübertragung\\
Wird kein Parameter übergeben, so wird der Port per Kommandozeile abgefragt.

\section{\texttt{ServerLimited}}

\textbf{Simulator als Rechenserver starten.}

Dieser Befehl erwartet 0 bis 2 weitere Parameter.\\
Werden entsprechend viele Parameter angegeben, so haben diese folgende Bedeutungen:\\
1. Zu verwendender Port\\
2. Passwort für die verschlüsselte Datenübertragung\\
Wird kein Parameter übergeben, so wird der Port per Kommandozeile abgefragt.\\
Der Server begrenzt die Anzahl an gleichzeitigen Anfragen basierend auf der Anzahl\\
an verfügbaren CPU-Kernen.

\section{\texttt{ServerMQTT}}

\textbf{Simulator als MQTT-basierten Rechenserver starten.}

Dieser Befehl erwartet zwei bis vier weiteren Parameter:\\
die Adresse des zu verwendenden MQTT-Brokers, den Namen\\
des MQTT-Topic und optional den Namen des\\
Statusmeldungen-Topics und optional Nutzername und\\
Passwort getrennt durch "`:"`.

\section{\texttt{ServerMQTTFixed}}

\textbf{Simulator als MQTT-basierten Rechenserver für ein festes Modell starten starten.}

Dieser Befehl erwartet drei bis fünf weiteren Parameter:\\
die Adresse des zu verwendenden MQTT-Brokers,\\
den Namen des MQTT-Topic,\\
optional den Namen des Statusmeldungen-Topics und \\
optional Nutzername und Passwort getrennt durch "`:"`.

\section{\texttt{ServerMQTTTest}}

\textbf{Simulator als MQTT-basierten Echo-Testserver starten.}

Dieser Befehl erwartet zwei oder drei weiteren Parameter:\\
die Adresse des zu verwendenden MQTT-Brokers, den Namen des MQTT-Topic optional Nutzername und Passwort getrennt durch "`:"`.

\section{\texttt{ServerSocket}}

\textbf{Simulator als Socket-basierten Rechenserver starten.}

Dieser Befehl erwartet einen oder zwei weiteren Parameter:\\
den zu verwendenden Port und optional einen Timeout-Wert (in Sekunden) für einzelne Simulationen.

\section{\texttt{ServerWeb}}

\textbf{Simulator als webbasierten Rechenserver starten.}

Dieser Befehl erwartet einen oder zwei weiteren Parameter:\\
den zu verwendenden Port und optional Nutzername und Passwort getrennt durch "`:"`.

\section{\texttt{ServerWebFixed}}

\textbf{Simulator als webbasierten Rechenserver für ein festes Modell starten.}

Dieser Befehl erwartet zwei oder drei weitere Parameter:\\
den zu verwendenden Port, den Dateinamen\\
der Modelldatei\\
und optional Nutzername und Passwort getrennt durch "`:"`.

\section{\texttt{Simulation}}

\textbf{Führt einen einzelnen Simulationslauf durch.}

Dieser Befehl erwartet zwei oder drei weitere Parameter:\\
1. Eingabe-Modell-Datei\\
2. Eingabe-Datentabelle (optional)\\
3. Ausgabe-Statistik-Datei\\
Die Eingabedatei und die Eingabe-Datentabelle (sofern angegeben) müssen existieren, die Ausgabedatei darf nicht existieren.

\section{\texttt{SimulationTimeout}}

\textbf{Führt einen einzelnen Simulationslauf (mit Timeout) durch.}

Dieser Befehl erwartet drei oder vier weitere Parameter:\\
1. Eingabe-Modell-Datei\\
2. Eingabe-Datentabelle (optional)\\
3. Ausgabe-Statistik-Datei\\
4. Timeout-Wert (in Sekunden)\\
Die Eingabedatei und die Eingabe-Datentabelle (sofern angegeben) müssen existieren, die Ausgabedatei darf nicht existieren.

\section{\texttt{Verzeichnis}}

\textbf{Simuliert alle Modelle und Parameterreihen in einem Verzeichnis.}

Dieser Befehl erwartet als Parameter den Namen des zu verarbeitenden Verzeichnisses.

\section{\texttt{VerzeichnisFilter}}

\textbf{Alle Statistikdateien in einem Verzeichnis filtern.}

Dieser Befehl erwartet genau drei weitere Parameter:\\
1. Name des Verzeichnisses\\
2. Name der Eingabe-Filterkonfigurationsdatei\\
3. Name der Ausgabedatei, die die Filterergebnisse aufnehmen soll\\
Wenn die Ausgabedatei bereits existiert, werden die neuen Ergebnisse angehängt.\\
Ansonsten wird die Datei neu angelegt.

\section{\texttt{Zusammenfassung}}

\textbf{Exportiert einen Teil oder die gesamten Simulationsergebnisse für ein Modell.}

Dieser Befehl erwartet genau drei weitere Parameter:\\
1. "`Inline"`, "`Einzeldateien"`, "`Liste"`, "`Text"`, "`PDF"' , "`LaTeX"`, "`HTMLApp"' oder ein Listeneintrag je nach dem, ob\\
a) ein HTML-Report mit eingebetteten Bildern,\\
b) ein HTML-Report mit Bildern in separaten Dateien,\\
c) eine Übersicht über alle verfügbaren Einzeldokumente,\\
d) ein DOCX-Report,\\
e) ein PDF-Report,\\
f) ein LaTeX-Report,\\
g) ein HTML-Web-App-Report oder\\
h) ein bestimmtes Einzeldokument ausgegeben werden soll.\\
2. Dateiname der Eingabedatei\\
3. Dateiname der Ausgabedatei

