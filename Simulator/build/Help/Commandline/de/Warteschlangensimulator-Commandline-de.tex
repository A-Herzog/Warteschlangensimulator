\RequirePackage{amsmath}
\RequirePackage{fix-cm}
\documentclass[deutsch]{svmono}

\def\ColoredLinks{}
%%%%%%%%%%%%%%%%%%%%%%%%%%%%%%%%%%%%%%%%%%%
% Alexanders Standardmacros - Version 3.0 %
%%%%%%%%%%%%%%%%%%%%%%%%%%%%%%%%%%%%%%%%%%%





%%%%%%%%%%%%%%%%%%%%%%%%%%%%%%%%%%%%%%%%%%%%%%%%
%  Farben für Links definieren                 %
%%%%%%%%%%%%%%%%%%%%%%%%%%%%%%%%%%%%%%%%%%%%%%%%

\ifdefined\ColoredLinks
  \def\linkColorLinkR{0}
  \def\linkColorLinkG{0}
  \def\linkColorLinkB{0.55}
	\def\linkColorCiteR{0}
  \def\linkColorCiteG{0}
  \def\linkColorCiteB{0.55}
	\def\linkColorUrlR{0}
  \def\linkColorUrlG{0}
  \def\linkColorUrlB{0.55}
\else
  \def\linkColorLinkR{0}
  \def\linkColorLinkG{0}
  \def\linkColorLinkB{0}
	\def\linkColorCiteR{0.3}
  \def\linkColorCiteG{0.3}
  \def\linkColorCiteB{0.3}
	\def\linkColorUrlR{0}
  \def\linkColorUrlG{0}
  \def\linkColorUrlB{0}
\fi





%%%%%%%%%%%%%%%%%%%%%%%%%%%%%%%%%%%%%%%%%%%%%%%%
%  Packages einbinden                          %
%%%%%%%%%%%%%%%%%%%%%%%%%%%%%%%%%%%%%%%%%%%%%%%%

%\usepackage{makeidx} - unterschägt manchmal Einträge
\usepackage{imakeidx} % Speziellen Features eigentlicht nicht verwendet, aber hat die makeidx-Fehler nicht
\usepackage[utf8]{inputenc}
\usepackage{amssymb}
%\usepackage[ngerman]{babel}
\usepackage{epsf}
\usepackage[dvips]{rotating}
\usepackage{amsmath,amsfonts}
\usepackage[amsmath,thmmarks,noconfig]{ntheorem}
\usepackage{nameref}
\usepackage{hycolor}
\usepackage{hyperxmp}
\usepackage{hyperref}
\hypersetup{pdfauthor={Alexander Herzog},
            pdftitle={Warteschlangensimulator - Hotkey reference},
            %pdfsubject={},
            %pdfkeywords={},
            pdfproducer={LaTeX},
            pdfcreator={pdfLaTeX},
						pdfcopyright={Copyright Alexander Herzog},
						pdfcontactcity={Clausthal-Zellerfeld},
						pdfcontactpostcode={38678},
						pdfcontactcountry={Deutschland},
						pdfcontactemail={alexander.herzog@tu-clausthal.de},
						pdfcontacturl={https://www.simzentrum.de,https://www.tu-clausthal.de},
						pdflang={de},
						bookmarksnumbered,
						colorlinks=true,
						filecolor=[rgb]{0,0,0},  % \href-Links nicht hervorheben
						linkcolor=[rgb]{\linkColorLinkR,\linkColorLinkG,\linkColorLinkB},
						citecolor=[rgb]{\linkColorCiteR,\linkColorCiteG,\linkColorCiteB},
						urlcolor=[rgb]{\linkColorUrlR,\linkColorUrlG,\linkColorUrlB} %\url{}, nur für E-Mail-Link genutzt
					}
\expandafter\def\expandafter\UrlBreaks\expandafter{\UrlBreaks
  \do\a\do\b\do\c\do\d\do\e\do\f\do\g\do\h\do\i\do\j
  \do\k\do\l\do\m\do\n\do\o\do\p\do\q\do\r\do\s\do\t
  \do\u\do\v\do\w\do\x\do\y\do\z\do\A\do\B\do\C\do\D
  \do\E\do\F\do\G\do\H\do\I\do\J\do\K\do\L\do\M\do\N
  \do\O\do\P\do\Q\do\R\do\S\do\T\do\U\do\V\do\W\do\X
  \do\Y\do\Z}
\usepackage{float}
\usepackage{fancyvrb}
\restylefloat{figure}
\restylefloat{table}
\usepackage{sectsty}
%\allsectionsfont{\fontfamily{cmss}\selectfont} - das führt zu Type 3 Fonts
\allsectionsfont{\sffamily\selectfont}
\usepackage{framed}
\usepackage[toc,page]{appendix}
\renewcommand\appendixname{Anhang}
\let\appendixtocname\appendixname\let\appendixpagename\appendixname
\usepackage[T1]{fontenc} % aus der pdf kopierbare Umlaute
\usepackage{eurosym}
\usepackage{longtable}

\usepackage{graphicx}
\usepackage[export]{adjustbox}




%%%%%%%%%%%%%%%%%%%%%%%%%%%%%%%%%%%%%%%%%%%%%%%%
%  Rotierte Tabellenüberschriften              %
%%%%%%%%%%%%%%%%%%%%%%%%%%%%%%%%%%%%%%%%%%%%%%%%

\usepackage{adjustbox}
\usepackage{array}
\usepackage{booktabs}
\usepackage{multirow}

\newcolumntype{R}[2]{%
    >{\adjustbox{angle=#1,lap=\width-(#2)}\bgroup}%
    l%
    <{\egroup}%
}
\newcommand*\rot{\multicolumn{1}{|R{90}{1em}|}}





%%%%%%%%%%%%%%%%%%%%%%%%%%%%%%%%%%%%%%%%%%%%%%%%
%  Symbole definieren                          %
%%%%%%%%%%%%%%%%%%%%%%%%%%%%%%%%%%%%%%%%%%%%%%%%

\newcommand{\setH}{\mathbb{H}}
\newcommand{\setC}{\mathbb{C}}
\newcommand{\setR}{\mathbb{R}}
\newcommand{\setQ}{\mathbb{Q}}
\newcommand{\setZ}{\mathbb{Z}}
\newcommand{\setN}{\mathbb{N}}
\newcommand{\comp}{\complement}
\newcommand{\Umg}{{\cal U}}

\newcommand{\calA}{{\cal A}}
\newcommand{\calB}{{\cal B}}
\newcommand{\calC}{{\cal C}}
\newcommand{\calD}{{\cal D}}
\newcommand{\calE}{{\cal E}}
\newcommand{\calF}{{\cal F}}
\newcommand{\calG}{{\cal G}}
\newcommand{\calH}{{\cal H}}
\newcommand{\calI}{{\cal I}}
\newcommand{\calJ}{{\cal J}}
\newcommand{\calK}{{\cal K}}
\newcommand{\calL}{{\cal L}}
\newcommand{\calM}{{\cal M}}
\newcommand{\calN}{{\cal N}}
\newcommand{\calO}{{\cal O}}
\newcommand{\calP}{{\cal P}}
\newcommand{\calQ}{{\cal Q}}
\newcommand{\calR}{{\cal R}}
\newcommand{\calS}{{\cal S}}
\newcommand{\calT}{{\cal T}}
\newcommand{\calU}{{\cal U}}
\newcommand{\calV}{{\cal V}}
\newcommand{\calW}{{\cal W}}
\newcommand{\calX}{{\cal X}}
\newcommand{\calY}{{\cal Y}}
\newcommand{\calZ}{{\cal Z}}

\def\arccot{\mathop{\rm arccot}}
\def\Arcoth{\mathop{\rm Arcoth}}
\def\Arsinh{\mathop{\rm Arsinh}}
\def\Artanh{\mathop{\rm Artanh}}
\def\Arcosh{\mathop{\rm Arcosh}}

\def\d{{\rm d}}
\def\dx{\d x}
\def\dy{\d y}
\def\dz{\d z}
\def\dt{\d t}
\def\euler{\mathrm{e}}

\def\id{{\rm id}}
\def\Kern{{\rm Kern}}

\def\ra{\Rightarrow}

\def\oversym#1#2{\mathop{#1}\limits^{#2}}
\def\Inneres#1{\oversym{#1}\circ}


\def\TO#1{\oversym\longrightarrow{#1}}
\def\RA#1{\oversym\Longrightarrow{#1}}
\def\IFF#1{\oversym\iff{#1}}
\def\gleich#1{\oversym={#1}}

\def\oBdA{o.\,B.\,d.\,A.\ }

\renewcommand{\qed}{\begin{flushright} $\square$ \end{flushright}}

\def\Definition#1{{\bf #1}}

\def\E{{\bf E}}
\def\Var{{\bf Var}}
\def\Std{{\bf Std}}
\def\CV{{\bf CV}}
\def\SCV{{\bf SCV}}





%%%%%%%%%%%%%%%%%%%%%%%%%%%%%%%%%%%%%%%%%%%%%%%%
%  Umgebungen definieren                       %
%%%%%%%%%%%%%%%%%%%%%%%%%%%%%%%%%%%%%%%%%%%%%%%%

%\theoremstyle{changebreak}
%\theoremheaderfont{\normalfont\bfseries}
%\theoremseparator{:}
%\theoremsymbol{}
%\newtheorem{definition}{Definition}[chapter]
%\newtheorem{satz}[definition]{Satz}
%\newtheorem{hilfssatz}[definition]{Hilfssatz}
%\newtheorem{lemma}[definition]{Lemma}
%\newtheorem{beispiel}[definition]{Beispiel}
%\newtheorem{beispiele}[definition]{Beispiele}
%\newtheorem{bezeichnung}[definition]{Bezeichnung}
%\theorembodyfont{\rmfamily}
%\newtheorem{bemerkung}[definition]{Bemerkung}
%\newtheorem{vereinbarung}[definition]{Vereinbarung}
%\newtheorem{folgerung}[definition]{Folgerung}
%\theoremheaderfont{\normalfont\bfseries}
%\theoremstyle{nonumberchangebreak}
%\theorembodyfont{\rmfamily}
%\theoremsymbol{$\blacksquare$}
%\newtheorem{beweis}{Beweis}
%\theoremsymbol{}





%%%%%%%%%%%%%%%%%%%%%%%%%%%%%%%%%%%%%%%%%%%%%%%%
%  Standard epsilon durch schöneres ersetzen   %
%%%%%%%%%%%%%%%%%%%%%%%%%%%%%%%%%%%%%%%%%%%%%%%%

\def\epsilon{\varepsilon}





%%%%%%%%%%%%%%%%%%%%%%%%%%%%%%%%%%%%%%%%%%%%%%%%
%  Verbatim-Umgebungen für verschiedene Zwecke %
%%%%%%%%%%%%%%%%%%%%%%%%%%%%%%%%%%%%%%%%%%%%%%%%

\DefineVerbatimEnvironment{ExcelVerbatim}{Verbatim}{frame=single,label=Excel-Befehl,fontsize=\small}
\DefineVerbatimEnvironment{ExcelVerbatimWithMath}{Verbatim}{frame=single,label=Excel-Befehl,commandchars=\\\{\},codes={\catcode`$=3\catcode`^=7},fontsize=\small}
\DefineVerbatimEnvironment{ExcelVerbatimWithFormat}{Verbatim}{frame=single,label=Excel-Befehl,fontsize=\small,commandchars=\\\{\}}
\DefineVerbatimEnvironment{RVerbatim}{Verbatim}{frame=single,label=R-Code,fontsize=\small}
\DefineVerbatimEnvironment{ExcelMacroVerbatim}{Verbatim}{frame=single,label=Excel-Makro,fontsize=\small} % ,numbers=left
\DefineVerbatimEnvironment{ExcelMarcroVerbatimWithMath}{Verbatim}{frame=single,label=Excel-Makro,commandchars=\\\{\},codes={\catcode`$=3\catcode`^=7},fontsize=\small}
\DefineVerbatimEnvironment{ExcelMacroVerbatimWithFormat}{Verbatim}{frame=single,label=Excel-Makro,fontsize=\small,commandchars=\\\{\}}
\newcommand{\sub}[2]{{#1_#2}}

\newcommand{\newlinesymbol}{\rotatebox[origin=c]{270}{$\curvearrowright$}}





%%%%%%%%%%%%%%%%%%%%%%%%%%%%%%%%%%%%%%%%%%%%%%%%
%  Formatierungen                              %
%%%%%%%%%%%%%%%%%%%%%%%%%%%%%%%%%%%%%%%%%%%%%%%%

%\def\emphasis#1{\textbf{#1}}
%\def\emphasisBFOnly#1{\textbf{#1}}
\def\emphasis#1{\emph{#1}}
\def\emphasisBFOnly#1{#1}

\def\emphasisIT#1{\textit{#1}}
\def\emphasisEnglish#1{\textit{#1}}

%\font\deutschfont=suet14
%\def\deutsch#1{\hbox{\deutschfont #1}}

\parindent0pt
\parskip5pt

%\oddsidemargin4.6mm
%\evensidemargin-5.4mm
%\textwidth160mm
\usepackage{xcolor}
\usepackage{lmodern}
\usepackage{wrapfig}
\textwidth160mm
\textheight220mm
\oddsidemargin0mm
\evensidemargin0mm
\topmargin0mm

\def\cmd#1{\textbf{,,\texttt{#1}''}}
\def\cm#1{\textbf{\texttt{#1}}}

\begin{document}

\thispagestyle{empty}
\vskip5cm {\Huge\textbf{Kommandozeilenbefehlsreferenz \vskip.1cm Warteschlangensimulator}}
\vskip.5cm \hrule
\vskip.5cm {\large \textsc{Alexander Herzog} (\href{mailto:alexander.herzog@tu-clausthal.de}{alexander.herzog@tu-clausthal.de})}
\IfFileExists{../../Warteschlangennetz-mittel.png}{
\vskip2cm \centerline{\fbox{\includegraphics[width=16cm]{../../Warteschlangennetz-mittel.png}}}
}{}
\vskip2cm {\color{gray}
Diese Referenz bezieht sich auf die Version \input{../../Version.tex} des Warteschlangensimulators.\\
Download-Adresse: \href{https://github.com/A-Herzog/Warteschlangensimulator/}{https://github.com/A-Herzog/Warteschlangensimulator/}.
}
\renewcommand{\thepage}{\roman{page}}

{
\clearpage
\pdfbookmark{Inhaltsverzeichnis}{Inhaltsverzeichnis}
\begingroup \let\cleardoublepage\relax \tableofcontents \endgroup
}

\renewcommand{\thepage}{\arabic{page}}
\setcounter{page}{1}

\chapter{Allgemeines}

Bei dem Warteschlangensimulator handelt es sich um ein Java-Programm, d.\,h.\ zur Ausführung wird eine
Java-Laufzeitumgebung benötigt. Ist diese vorhanden, so kann der Simulator über folgenden Befehl gestartet
werden:

\texttt{java -jar Simulator.jar}

Wird an diese Befehlszeile durch ein Leerzeichen abgetrennt der Dateiname einer Model-, Statistik-,
Parameterreihen- oder Optimiererdatei angehängt, so wird diese unmittelbar nach dem Programmstart geladen.

\chapter{Liste der Befehle}

Alternativ kann der Warteschlangensimulator mit einem der in den folgenden Abschnitten beschriebenen Befehle
aufgerufen werden. In diesem Fall wird kein Programmfenster angezeigt, sondern alle Ausgaben erfolgen direkt
auf der Kommandozeile. Das Aufrufschema der Befehle lautet dabei:

\texttt{java -jar Simulator.jar Befehl OptionenFürBefehl}

Alle im Folgenden beschriebenen Befehle können auch direkt über die grafische Programmoberfläche über den
Befehl "`Kommandozeilenbefehl ausführen..."' im "`Extras"'-Menü ausprobiert werden.

\section{\texttt{Benchmark}}

\textbf{Performs a speed test of the computer.}

The command expected zero or one additional parameters.\\
If a parameter is passed, this is either the maximum number of threads\\
to be used, or the file name of the model to be used.

\section{\texttt{BuildCLIReference}}

\textbf{Generates LaTeX code from the command line help.}

The command expects two additional parameters:\\
1. Language "`en"' or "`de"'\\
2. Output folder for the LaTeX

\section{\texttt{BuildElementGroupImages}}

\textbf{Generates per element group an image in the desktop folder.}

The command expects one additional parameter:\\
1. Language "`en"' or "`de"'

\section{\texttt{BuildLaTeXReference}}

\textbf{Generates LaTeX code from the html-based station help.}

The command expects two additional parameters:\\
1. Language "`en"' or "`de"'\\
2. Output folder for the LaTeX and image files

\section{\texttt{Filter}}

\textbf{Filter statistic file.}

The command expects exactly three additional parameters:\\
1. Name of the input statistic file\\
2. Name of the input filter configuration file\\
3. Name of the output file which will contain the filtered results\\
If the output file already exists, the new results will be added.\\
Otherwise the file will be created.

\section{\texttt{GC}}

\textbf{Lists the currently active garbage collectors.}

This command expects no additional parameters.

\section{\texttt{Help}}

\textbf{Shows this help.}

This command expects one or no additional parameters.\\
If a command is enters as additional parameter, the help information for this command is displayed.\\
Otherwise a list with all available commands is displayed.

\section{\texttt{Optimizer}}

\textbf{Starts a model optimization.}

This command expects exactly two additional parameters:\\
1. Input model file\\
2. Optimizer configuration file\\
Both input files have to exist.

\section{\texttt{ParameterSeriesVarianceAnalysis}}

\textbf{Creates a parameter series configuration for a variance analysis}

This command expects three or four additional parameters:\\
1. Input model file\\
2. Output parameter series file\\
3. Number of repetitions or the model\\
4. New value for the number of arrivals (optional)\\
The input file has to exist.\\
The output has to be non existent.

\section{\texttt{Parameterseries}}

\textbf{Run a parameter series simulation.}

This command expects exactly two additional parameters:\\
1. Input parameter series file\\
2. Output parameter series file\\
The input file has to exist.\\
The output has to be non existent.

\section{\texttt{ParameterseriesTable}}

\textbf{Exports the results table of a parameter series.}

This command expects exactly two additional parameters:\\
1. Input parameter series file\\
2. Output table file\\
The input file has to exist.\\
The output has to be non existent.

\section{\texttt{Report}}

\textbf{Exports the whole report or a part of it.}

This command expects exactly three additionall parameters:\\
1. "`Inline"`, "`SingleFiles"`, "`List"`, "`Text"`, "`PDF"' or a list entry, depending if\\
a) a html report with inline images,\\
b) a html teport with separate images,\\
c) an overview of all available individual documents,\\
d) a docx report\\
e) a pdf report or\\
f) an individual document is to be reported.\\
2. File name of the input file\\
3. File name of the output file

\section{\texttt{Reset}}

\textbf{Resets the program configuration.}

Resets the configuration of the simulator.\\
Only the entered license keys will be kept.

\section{\texttt{Server}}

\textbf{Starts the program as simulation server.}

This command expects 0 to 2 additional parameters.\\
If these parameters exist they have the following meaning:\\
1. Port to be used\\
2. Password for encrypted data transfer\\
If no parameters are specified, the simulator will ask for the port via the console.

\section{\texttt{ServerLimited}}

\textbf{Starts the program as simulation server.}

This command expects 0 to 2 additional parameters.\\
If these parameters exist they have the following meaning:\\
1. Port to be used\\
2. Password for encrypted data transfer\\
If no parameters are specified, the simulator will ask for the port via the console.\\
The server will limit the number of simultaneous requests depending on the\\
number of available CPU cores.

\section{\texttt{ServerWeb}}

\textbf{Starts the program as web-based simulation server.}

This command expects one additional parameter:\\
the port to be used.

\section{\texttt{SetMaxThreads}}

\textbf{Set the maximum number of threads}

The function expects the maximum number of threads to be used as parameter.\\
Values less or equal 0 will be interpreted as unlimited.

\section{\texttt{SetNUMAMode}}

\textbf{Set NUMA mode}

The function expects a "`1"' or a "`0"' as parameter.

\section{\texttt{Simulation}}

\textbf{Start a simulation run.}

This command expects exactly two additional parameters:\\
1. Input model file\\
2. Output statistic file\\
The input file has to be existent, the output has to be non existent.

\section{\texttt{Version}}

\textbf{Displays the current version number.}

Outputs only the version number of the Warteschlangensimulators.\\
No further processing will be done.



\end{document}