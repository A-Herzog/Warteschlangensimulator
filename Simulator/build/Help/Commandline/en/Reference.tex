\section{\texttt{Filter}}

\textbf{Filter statistic file.}

The command expects exactly three additional parameters:\\
1. Name of the input statistic file\\
2. Name of the input filter configuration file\\
3. Name of the output file which will contain the filtered results\\
If the output file already exists, the new results will be added.\\
Otherwise the file will be created.

\section{\texttt{Folder}}

\textbf{Simulates all models und parameter series in a folder.}

The command expects the name of the folder to process as additional parameter.

\section{\texttt{FolderFilter}}

\textbf{Filter all statistic files in a folder.}

The command expects exactly three additional parameters:\\
1. Name of the input folder\\
2. Name of the input filter configuration file\\
3. Name of the output file which will contain the filtered results\\
If the output file already exists, the new results will be added.\\
Otherwise the file will be created.

\section{\texttt{Help}}

\textbf{Shows this help.}

This command expects one or no additional parameters.\\
If a command is enters as additional parameter, the help information for this command is displayed.\\
Otherwise a list with all available commands is displayed.

\section{\texttt{Interactive}}

\textbf{Starts the interactive mode.}

This command expects no additional parameters.

\section{\texttt{Optimizer}}

\textbf{Starts a model optimization.}

This command expects exactly two additional parameters:\\
1. Input model file\\
2. Optimizer configuration file\\
Both input files have to exist.

\section{\texttt{ParameterSeriesVarianceAnalysis}}

\textbf{Creates a parameter series configuration for a variance analysis}

This command expects three or four additional parameters:\\
1. Input model file\\
2. Output parameter series file\\
3. Number of repetitions or the model\\
4. New value for the number of arrivals (optional)\\
The input file has to exist.\\
The output has to be non existent.

\section{\texttt{Parameterseries}}

\textbf{Run a parameter series simulation.}

This command expects exactly two additional parameters:\\
1. Input parameter series file\\
2. Output parameter series file\\
The input file has to exist.\\
The output has to be non existent.

\section{\texttt{ParameterseriesTable}}

\textbf{Exports the results table of a parameter series.}

This command expects exactly two additional parameters:\\
1. Input parameter series file\\
2. Output table file\\
The input file has to exist.\\
The output has to be non existent.

\section{\texttt{Report}}

\textbf{Exports the whole report or a part of it.}

This command expects exactly three additionall parameters:\\
1. "`Inline"`, "`SingleFiles"`, "`List"`, "`Text"`, "`PDF"`, "`LaTeX"`, "`HTMLApp"' or a list entry, depending if\\
a) a html report with inline images,\\
b) a html teport with separate images,\\
c) an overview of all available individual documents,\\
d) a docx report,\\
e) a pdf report,\\
f) a LaTeX report\\
g) a html web app report or\\
h) an individual document is to be reported.\\
2. File name of the input file\\
3. File name of the output file

\section{\texttt{Server}}

\textbf{Starts the program as simulation server.}

This command expects 0 to 2 additional parameters.\\
If these parameters exist they have the following meaning:\\
1. Port to be used\\
2. Password for encrypted data transfer\\
If no parameters are specified, the simulator will ask for the port via the console.

\section{\texttt{ServerLimited}}

\textbf{Starts the program as simulation server.}

This command expects 0 to 2 additional parameters.\\
If these parameters exist they have the following meaning:\\
1. Port to be used\\
2. Password for encrypted data transfer\\
If no parameters are specified, the simulator will ask for the port via the console.\\
The server will limit the number of simultaneous requests depending on the\\
number of available CPU cores.

\section{\texttt{ServerMQTT}}

\textbf{Starts the program as MQTT-based simulation server.}

This command expects two to four additional parameter:\\
the address of the MQTT broker, the MQTT topic optionally\\
the name of a status information topic and optionally\\
the user name and the password separated by "`:"`.

\section{\texttt{ServerMQTTFixed}}

\textbf{Starts the program as MQTT-based simulation server for a fixed model.}

This command three to five additional parameter:\\
the address of the MQTT broker, the MQTT topic, the file name\\
of the model file to use, optionally the name of a status\\
information topic and optionally\\
the user name and the password separated by "`:"`.

\section{\texttt{ServerMQTTTest}}

\textbf{Starts the program as MQTT-based echo test server.}

This command expects two or three additional parameter:\\
the address of the MQTT broker, the MQTT topic and optionally the user name and the password separated by "`:"`.

\section{\texttt{ServerSocket}}

\textbf{Starts the program as socket-based simulation server.}

This command expects one additional parameter:\\
the port to be used.

\section{\texttt{ServerWeb}}

\textbf{Starts the program as web-based simulation server.}

This command expects one or two additional parameter:\\
the port to be used and optionally the user name and the password separated by "`:"`.

\section{\texttt{ServerWebFixed}}

\textbf{Starts the program as web-based simulation server for a fixed model.}

This command expects two or three additional parameters:\\
the port to be used, the file name of the\\
model to be loaded and optionally the user name and the password separated by "`:"`.

\section{\texttt{Simulation}}

\textbf{Starts a simulation run.}

This command expects two or three additional parameters:\\
1. Input model file\\
2. Input data table file (optional)\\
3. Output statistic file\\
The input model file and the input data table (if specified) has to be existent, the output has to be non existent.

\section{\texttt{SimulationTimeout}}

\textbf{Starts a simulation run (with timeout).}

This command expects three or four additional parameters:\\
1. Input model file\\
2. Input data table file (optional)\\
3. Output statistic file\\
4. Timeout value (in seconds)\\
The input model file and the input data table (if specified) has to be existent, the output has to be non existent.

