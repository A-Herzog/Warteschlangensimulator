\part{Referenz der Rechenbefehle}\label{part:Rechenbefehle}

Bei der Verwendung von Rechenbefehlen im Warteschlangensimulator ist zwischen
\textbf{Ausdrücken} und \textbf{Vergleichen} zu unterscheiden. Ausdrücke dienen
dazu, einen Zahlenwert zu berechnen, der dann z.B.\ als Zeitdauer verwendet wird.
Vergleiche liefern eine ja/nein Entscheidung (z.B.\ ob ein Kunde in eine bestimmte
Richtung geleitet werden soll). Im Gegensatz zu Ausdrücken besitzen Vergleiche immer
mindestens einen Vergleichsoperator.

Alle im Folgenden vorgestellten Befehle werden jeweils in jeder beliebigen
\textbf{Groß- und Kleinschreibung} erkannt, d.h.\ es wird nicht zwischen
verschiedenen Groß-/Kleinschreibweisen unterschieden.



\chapter{Konstanten}

Folgende Konstanten stehen in den allen Rechenbefehlen zur Verfügung:

\begin{itemize}

\item
\cmd{e}: Liefert die Basis der Exponentialfunktion $e^x$. Es gilt
$e\approx 2,718281828459$.

\item
\cmd{pi}: Liefert den Wert der Kreiskonstante $\pi$. Es gilt
$\pi\approx 3,1415926535898$.

\end{itemize}



\chapter{Variablen}

Werden Ausdrücke im Kontext eines Kunden berechnet, so stehen die Variablen
\begin{itemize}
\item
\cmd{w} für die \textbf{bisherige Wartezeit} des Kunden,
\item
\cmd{t} für die \textbf{bisherige Transferzeit} des Kunden und
\item
\cmd{p} für die \textbf{bisherige Bedienzeit} des Kunden zur Verfügung.
\end{itemize}

Bei der Berechnung von Score-Werten wird \cmd{w} abweichend nicht mit der bisherigen
gesamten Wartezeit des Kunden belegt, sondern mit der bisherigen Wartezeit an der aktuellen Station.

Des weiteren stehen stets alle Variablen, die über ein Zuweisungselement definiert werden, zur Verfügung.
Vor der ersten Zuweisung eines Wertes an eine Variable hat diese den Wert 0.



\chapter{Grundrechenarten}

Es werden die üblichen Grundrechenarten unterstützt:
\begin{itemize}
\item Addition: \cmd{$+$}
\item Subtraktion: \cmd{$-$}
\item Multiplikation: \cmd{$*$}
\item Division: \cmd{$/$}
\item Potenzieren: \cmd{$\hat~$}
\end{itemize}

Die Regel \textbf{Punkt- vor Strichrechnung} wird dabei berücksichtigt.
Um davon abweichende Auswertungen zu erzwingen, können \textbf{Klammern}
gesetzt werden.



\chapter{Nachgestellte Anweisungen}

Folgende Ausdrücke können unmittelbar hinter eine Zahl geschrieben werden:

\begin{itemize}
\item
\cmd{\%}:
Der Zahlenwert vor diesem Symbol wird als Prozentwert interpretiert,
z.B.\ $30\%=0{,}3$.

\item
\cmd{$^2$}:
Potenziert die Zahl vor diesem Symbol mit 2.

\item
\cmd{$^3$}:
Potenziert die Zahl vor diesem Symbol mit 3.

\item
\cmd{!}:
Berechnet die Fakultät der Zahl vor diesem Ausdruck, z.B.\ $4!=1\cdot2\cdot3\cdot4=24$.

\item
\cmd{$^{\circ}$}:
Rechnet die Zahl vor diesem Symbol von Grad nach Bogenmaß um, z.B.\ $180^{\circ}=3{,}1415\ldots$.\\
(Siehe hierzu auch Abschnitt \ref{sec:Winkelfunktionen} in dem die vom
Warteschlangensimulator unterstützten Winkelfunktionen vorgestellt werden.)
\end{itemize}



\chapter{Allgemeine Funktionen}

\begin{itemize}

\item
\cmd{abs(x)}:
Absolutbetrag, z.B.\ \cm{abs(-5)=5}.

\item
\cmd{ceil(x)}:
Aufrunden, z.B.\ \cm{ceil(2{,}1)=3}

\item
\cmd{exp(x)}:
Exponentialfunktion $e^x$.

\item
\cmd{factorial(x)}:
Fakultät, z.B.\ $4!=1\cdot2\cdot3\cdot4=24$.

\item
\cmd{floor(x)}:
Abrunden, z.B.\ \cm{floor(2{,}9)=2}

\item
\cmd{frac(x)}:
Nachkommaanteil, z.B.\ \cm{frac(1{,}3)=0,3}

\item
\cmd{gamma(x)}:
Gamma-Funktion, z.B.\ \cm{gamma(5)=4!=24}

\item
\cmd{int(x)}:
Ganzzahlanteil, z.B.\ \cm{int(2{,}9)=2}

\item
\cmd{log(x)}:
Logarithmus zur Basis $e$.

\item
\cmd{log(x;b)}:
Logarithmus zur Basis $b$.

\item
\cmd{ld(x)}:
Logarithmus zur Basis $2$, z.B.\ \cm{ld(256)=8}.

\item
\cmd{lg(x)}:
Logarithmus zur Basis $10$, z.B.\ \cm{lg(100)=2}.

\item
\cmd{ln(x)}:
Logarithmus zur Basis $e$.

\item
\cmd{modulo(a;b)} oder \cmd{mod(a;b)}:
Divisionsrest bei Division a/b

\item
\cmd{pow(x;y)}:
Potenzieren $x^y$.

\item
\cmd{round(x)}:
Runden, z.B.\ \cm{round(4{,}4)=4} und \cm{round(4{,}5)=5}.

\item
\cmd{sign(x)}:
Vorzeichen einer Zahl, z.B.\ \cm{sign(3)=1} und \cm{sign(-3)=-1}.

\item
\cmd{sqrt(x)}:
Quadratwurzel, z.B.\ \cm{sqrt{81}=9}.

\item
\cmd{sqr(x)}:
Zahl quadrieren, z.B.\ \cm{sqr(4)=16}.

\end{itemize}



\section{Zufallszahlen}

Mit den folgenden Befehlen können Zufallszahlen, die in einem bestimmten
Bereich \textbf{gleichverteilt} sind erzeugt werden. Im Abschnitt
\ref{sec:Wahrscheinlichkeitsverteilungen} werden weitere Funktionen zur
Erzeugung von Zufallszahlen gemäß bestimmten Verteilungsfunktionen vorgestellt.

\begin{itemize}

\item
\cmd{random()}:
Liefert eine Zufallszahl zwischen 0 (einschließlich) und 1 (ausschließlich).

\item
\cmd{random(x)}:
Liefert eine Zufallszahl zwischen 0 (einschließlich) und x (ausschließlich).

\end{itemize}





\chapter{Winkelfunktionen}\label{sec:Winkelfunktionen}

Die Winkelfunktionen beziehen sich immer auf $2\pi$ als Vollkreis (Bogenmaß, ,,Rad'').
Wenn Winkel in Grad ($360^\circ$ für den Vollkreis) angegeben werden sollen, so müssen
diese bei der Verwendung der elementaren Winkelfunktionen in Bodenmaß umgerechnet werden,
z.B.\ \cm{sin($90^\circ$)=1}.



\section{Elementare Winkelfunktionen}

\begin{itemize}

\item
\cmd{sin(x)}:
Sinus

\item
\cmd{cos(x)}:
Cosinus

\item
\cmd{tan(x)}:
Tangens

\item
\cmd{cot(x)}:
Cotangens

\end{itemize}



\section{Hyperbolische Winkelfunktionen}

\begin{itemize}

\item
\cmd{sinh(x)}:
Sinus hyperbolicus

\item
\cmd{cosh(x)}:
Cosinus hyperbolicus

\item
\cmd{tanh(x)}:
Tangens hyperbolicus

\item
\cmd{coth(x)}:
Cotangens hyperbolicus

\end{itemize}



\section{Umkehrfunktionen der elementaren Winkelfunktionen}

\begin{itemize}

\item
\cmd{arcsin(x)}:
Arcus-Sinus

\item
\cmd{arccos(x)}:
Arcus-Cosinus

\item
\cmd{arctan(x)}:
Arcus-Tangens

\item
\cmd{arccot(x)}:
Arcus-Cotangens

\end{itemize}



\section{Umkehrfunktionen der hyperbolischen Winkelfunktionen}

\begin{itemize}

\item
\cmd{arcsinh(x)}:
Arcus-Sinus hyperbolicus

\item
\cmd{arccosh(x)}:
Arcus-Cosinus hyperbolicus

\item
\cmd{arctanh(x)}:
Arcus-Tangens hyperbolicus

\item
\cmd{arccoth(x)}:
Arcus-Cotangens hyperbolicus

\end{itemize}



\chapter{Funktionen mit mehreren Parametern}

Die folgenden Funktionen können beliebig viele Parameter entgegennehmen.
Die einzelnen Parameter müssen dabei durch Semikolon ,,;'' getrennt
angegeben werden.

\begin{itemize}

\item
\cmd{Min(a;b;c;...)}:
Berechnet das Minimum der übergebenen Zahlen.

\item
\cmd{Max(a;b;c;...)}:
Berechnet das Maximum der übergebenen Zahlen.

\item
\cmd{Sum(a;b;c;...)}:
Berechnet die Summe der übergebenen Zahlen.

\item
\cmd{Mean(a;b;c;...)}:
Berechnet das arithmetische Mittel der übergebenen Zahlen.

\item
\cmd{Median(a;b;c;...)}:
Berechnet den Median der übergebenen Zahlen.

\item
\cmd{Var(a;b;c;...)}:
Berechnet die korrigierte Stichprobenvarianz der übergebenen Zahlen.

\item
\cmd{SD(a;b;c;...)}:
Berechnet die korrigierte Stichprobenstandardabweichung der übergebenen Zahlen.

\item
\cmd{SCV(a;b;c;...)}:
Berechnet den quadrierten Variationskoeffizient der übergebenen Zahlen.

\item
\cmd{CV(a;b;c;...)}:
Berechnet den Variationskoeffizient der übergebenen Zahlen.

\end{itemize}



\chapter{Wahrscheinlichkeitsverteilungen}\label{sec:Wahrscheinlichkeitsverteilungen}

Mit Hilfe der folgenden Befehle können sowohl Werte der Dichte und der Verteilungsfunktion
der folgenden Wahrscheinlichkeitsverteilungen berechnet werden als auch Zufallszahlen
auf Basis einer dieser Wahrscheinlichkeitsverteilungen berechnet werden:



\section{Exponentialverteilung mit Mittelwert \texorpdfstring{$a$}{a}}

\begin{itemize}

\item
\cmd{ExpDist(x;a;0)}:
Berechnet die Dichte an der Stelle $x$.

\item
\cmd{ExpDist(x;a;1)}:
Berechnet den Wert der Verteilungsfunktion an der Stelle $x$.

\item
\cmd{ExpDist(a)}:
Erzeugt eine Zufallszahl gemäß der Verteilung.

\end{itemize}



\section{Gleichverteilung über das Intervall \texorpdfstring{$[a;b]$}{[a;b]}}

\begin{itemize}

\item
\cmd{UniformDist(x;a;b;0)}:
Berechnet die Dichte an der Stelle $x$.

\item
\cmd{UniformDist(x;a;b;1)}:
Berechnet den Wert der Verteilungsfunktion an der Stelle $x$.

\item
\cmd{UniformDist(a;b)}:
Erzeugt eine Zufallszahl gemäß der Verteilung.

\end{itemize}



\section{Normalverteilung mit Mittelwert \texorpdfstring{$a$}{a} und Standardabweichung \texorpdfstring{$b$}{b}}

\begin{itemize}

\item
\cmd{NormalDist(x;a;b;0)}:
Berechnet die Dichte an der Stelle $x$.

\item
\cmd{NormalDist(x;a;b;1)}:
Berechnet den Wert der Verteilungsfunktion an der Stelle $x$.

\item
\cmd{NormalDist(a;b)}:
Erzeugt eine Zufallszahl gemäß der Verteilung.

\end{itemize}



\section{Lognormalverteilung mit Mittelwert \texorpdfstring{$a$}{a} und Standardabweichung \texorpdfstring{$b$}{b}}

\begin{itemize}

\item
\cmd{LogNormalDist(x;a;b;0)}:
Berechnet die Dichte an der Stelle $x$.

\item
\cmd{LogNormalDist(x;a;b;1)}:
Berechnet den Wert der Verteilungsfunktion an der Stelle $x$.

\item
\cmd{LogNormalDist(a;b)}:
Erzeugt eine Zufallszahl gemäß der Verteilung.

\end{itemize}



\section{Gamma-Verteilung mit Parametern \texorpdfstring{$\alpha=a$}{a} und \texorpdfstring{$\beta=b$}{b}}

\begin{itemize}

\item
\cmd{GammaDist(x;a;b;0)}:
Berechnet die Dichte an der Stelle $x$.

\item
\cmd{GammaDist(x;a;b;1)}:
Berechnet den Wert der Verteilungsfunktion an der Stelle $x$.

\item
\cmd{GammaDist(a;b)}:
Erzeugt eine Zufallszahl gemäß der Verteilung.
\end{itemize}



\section{Erlang-Verteilung mit Parametern \texorpdfstring{$n$}{n} und \texorpdfstring{$\lambda=l$}{l}}

\begin{itemize}

\item
\cmd{ErlangDist(x;n;l;0)}:
Berechnet die Dichte an der Stelle $x$.

\item
\cmd{ErlangDist(x;n;l;1)}:
Berechnet den Wert der Verteilungsfunktion an der Stelle $x$.

\item
\cmd{ErlangDist(n;b)}:
Erzeugt eine Zufallszahl gemäß der Verteilung.

\end{itemize}



\section{Beta-Verteilung in dem Intervall \texorpdfstring{$[a;b]$}{[a;b]} und mit Parametern \texorpdfstring{$\alpha=c$}{c} und \texorpdfstring{$\beta=d$}{d}}

\begin{itemize}

\item
\cmd{BetaDist(x;a;b;c;d;0)}:
Berechnet die Dichte an der Stelle $x$.

\item
\cmd{BetaDist(x;a;b;c;d;1)}:
Berechnet den Wert der Verteilungsfunktion an der Stelle $x$.

\item
\cmd{BetaDist(a;b;c;d)}:
Erzeugt eine Zufallszahl gemäß der Verteilung.

\end{itemize}



\section{Weibull-Verteilung mit Parametern Scale=\texorpdfstring{$a$}{a} und Form=\texorpdfstring{$b$}{b}}

\begin{itemize}

\item
\cmd{WeibullDist(x;a;b;0)}:
Berechnet die Dichte an der Stelle $x$.

\item
\cmd{WeibullDist(x;a;b;1)}:
Berechnet den Wert der Verteilungsfunktion an der Stelle $x$.

\item
\cmd{WeibullDist(a;b)}:
Erzeugt eine Zufallszahl gemäß der Verteilung.

\end{itemize}



\section{Cauchy-Verteilung mit Mittelwert \texorpdfstring{$a$}{a} und Scale=\texorpdfstring{$b$}{b}}

\begin{itemize}

\item
\cmd{CauchyDist(x;a;b;0)}:
Berechnet die Dichte an der Stelle $x$.

\item
\cmd{CauchyDist(x;a;b;1)}:
Berechnet den Wert der Verteilungsfunktion an der Stelle $x$.

\item
\cmd{CauchyDist(a;b)}:
Erzeugt eine Zufallszahl gemäß der Verteilung.

\end{itemize}



\section{\texorpdfstring{Chi$^2$}{Chi2}-Verteilung mit \texorpdfstring{$n$}{n} Freiheitsgraden}

\begin{itemize}

\item
\cmd{ChiSquareDist(x;n;0)}:
Berechnet die Dichte an der Stelle $x$.

\item
\cmd{ChiSquareDist(x;n;1)}:
Chi$^2$-Verteilung mit $n$ Freiheitsgraden.

\item
\cmd{ChiSquareDist(n)}:
Erzeugt eine Zufallszahl gemäß der Verteilung.

\end{itemize}



\section{Chi-Verteilung mit \texorpdfstring{$n$}{n} Freiheitsgraden}

\begin{itemize}

\item
\cmd{ChiDist(x;n;0)}:
Berechnet die Dichte an der Stelle $x$.

\item
\cmd{ChiDist(x;n;1)}:
Chi-Verteilung mit $n$ Freiheitsgraden.

\item
\cmd{ChiDist(n)}:
Erzeugt eine Zufallszahl gemäß der Verteilung.

\end{itemize}



\section{F-Verteilung mit \texorpdfstring{$a$}{a} Freiheitsgraden im Zähler und \texorpdfstring{$b$}{b} Freiheitsgraden im Nenner}

\begin{itemize}

\item
\cmd{FDist(x;a;b;0)}:
Berechnet die Dichte an der Stelle $x$.

\item
\cmd{FDist(x;a;b;1)}:
Berechnet den Wert der Verteilungsfunktion an der Stelle $x$.

\item
\cmd{FDist(a;b)}:
Erzeugt eine Zufallszahl gemäß der Verteilung.

\end{itemize}



\section{Johnson-SU-Verteilung mit den Parametern \texorpdfstring{$\gamma=a$}{a}, \texorpdfstring{$\xi=b$}{b}, \texorpdfstring{$\delta=c$}{c} und \texorpdfstring{$\lambda=d$}{d}}

\begin{itemize}

\item
\cmd{JohnsonSUDist(x;a;b;c;d;0)}:
Berechnet die Dichte an der Stelle $x$.

\item
\cmd{JohnsonSUDist(x;a;b;c;d;1)}:
Berechnet den Wert der Verteilungsfunktion an der Stelle $x$.

\item
\cmd{JohnsonSUDist(a;b;c;d)}:
Erzeugt eine Zufallszahl gemäß der Verteilung.  

\end{itemize}



\section{Dreiecksverteilung über \texorpdfstring{$[a;c]$}{[a;c]} mit der höchsten Wahrscheinlichkeitsdichte bei \texorpdfstring{$b$}{b}}

\begin{itemize}

\item
\cmd{TriangularDist(x;a;b;c;0)}:
Berechnet die Dichte an der Stelle $x$.

\item
\cmd{TriangularDist(x;a;b;c;1)}:
Berechnet den Wert der Verteilungsfunktion an der Stelle $x$.

\item
\cmd{TriangularDist(a;b;c)}:
Erzeugt eine Zufallszahl gemäß der Verteilung.

\end{itemize}



\section{Laplace-Verteilung mit Mittelwert \texorpdfstring{$mu$}{mu} und Skalierungsfaktor \texorpdfstring{$b$}{b}}

\begin{itemize}

\item
\cmd{LaplaceDist(x;mu;b;0)}:
Berechnet die Dichte an der Stelle $x$.

\item
\cmd{LaplaceDist(x;mu;b;1)}:
Berechnet den Wert der Verteilungsfunktion an der Stelle $x$.

\item
\cmd{LaplaceDist(mu;b)}:
Erzeugt eine Zufallszahl gemäß der Verteilung.

\end{itemize}



\section{Pareto-Verteilung mit Skalierungsparameter \texorpdfstring{$x_{\rm min}=xmin$}{xmin} und Formparameter \texorpdfstring{$\alpha=a$}{a}}

\begin{itemize}

\item
\cmd{ParetoDist(x;xmin;a;0)}:
Berechnet die Dichte an der Stelle $x$.

\item
\cmd{ParetoDist(x;xmin;a;1)}:
Berechnet den Wert der Verteilungsfunktion an der Stelle $x$.

\item
\cmd{ParetoDist(xmin;a)}:
Erzeugt eine Zufallszahl gemäß der Verteilung.

\end{itemize}



\section{Logistische Verteilung mit Mittelwert \texorpdfstring{$\mu=mu$}{mu} und Skalierungsparameter \texorpdfstring{$s$}{s}}

\begin{itemize}

\item
\cmd{LogisticDist(x;mu;s;0)}:
Berechnet die Dichte an der Stelle $x$.

\item
\cmd{LogisticDist(x;mu;s;1)}:
Berechnet den Wert der Verteilungsfunktion an der Stelle $x$.

\item
\cmd{LogisticDist(mu;s)}:
Erzeugt eine Zufallszahl gemäß der Verteilung.

\end{itemize}


	
\section{Inverse Gauß-Verteilung mit \texorpdfstring{$\lambda=l$}{l} und Mittelwert \texorpdfstring{$mu$}{mu}}

\begin{itemize}

\item
\cmd{InverseGaussianDist(x;l;mu;0)}:
Berechnet die Dichte an der Stelle $x$.

\item
\cmd{InverseGaussianDist(x;l;mu;1)}:
Berechnet den Wert der Verteilungsfunktion an der Stelle $x$.

\item
\cmd{InverseGaussianDist(l;mu)}:
Erzeugt eine Zufallszahl gemäß der Verteilung.

\end{itemize}



\section{Rayleigh-Verteilung mit Mittelwert \texorpdfstring{$mu$}{mu}}

\begin{itemize}

\item
\cmd{RayleighDist(x;mu;0)}:
Berechnet die Dichte an der Stelle $x$.

\item
\cmd{RayleighDist(x;mu;1)}:
Berechnet den Wert der Verteilungsfunktion an der Stelle $x$.

\item
\cmd{RayleighDist(mu)}:
Erzeugt eine Zufallszahl gemäß der Verteilung.

\end{itemize}



\section{Log-Logistische Verteilung mit \texorpdfstring{$\alpha$}{alpha} und Mittelwert \texorpdfstring{$\beta$}{beta}}

\begin{itemize}

\item
\cmd{LogLogisticDist(x;alpha;beta;0)}:
Berechnet die Dichte an der Stelle $x$.

\item
\cmd{LogLogisticDist(x;alpha;beta;1)}:
Berechnet den Wert der Verteilungsfunktion an der Stelle $x$.

\item
\cmd{LogLogisticDist(alpha;beta)}:
Erzeugt eine Zufallszahl gemäß der Verteilung.

\end{itemize}



\section{Potenzverteilung auf dem Bereich \texorpdfstring{$[a;b]$}{[a;b]} mit Exponent \texorpdfstring{$c$}{c}}

\begin{itemize}

\item
\cmd{PowerDist(x;a;b;c;0)}:
Berechnet die Dichte an der Stelle $x$.

\item
\cmd{PowerDist(x;a;b;c;1)}:
Berechnet den Wert der Verteilungsfunktion an der Stelle $x$.

\item
\cmd{PowerDist(a;b;c)}:
Erzeugt eine Zufallszahl gemäß der Verteilung.

\end{itemize}



\section{Gumbel-Verteilung mit Erwartungswert \texorpdfstring{$a$}{a} und Standardabweichung \texorpdfstring{$b$}{b}}

\begin{itemize}

\item
\cmd{GumbelDist(x;a;b;0)}:
Berechnet die Dichte an der Stelle $x$.

\item
\cmd{GumbelDist(x;a;b;1)}:
Berechnet den Wert der Verteilungsfunktion an der Stelle $x$.

\item
\cmd{GumbelDist(a;b)}:
Erzeugt eine Zufallszahl gemäß der Verteilung.

\end{itemize}



\section{Fatigue-Life-Verteilung mit Lageparameter \texorpdfstring{$\mu$}{mu}, Skalierungsparameter \texorpdfstring{$\beta$}{beta} und Formparameter \texorpdfstring{$\gamma$}{gamma}}

\begin{itemize}

\item
\cmd{FatigueLifeDist(x;mu;beta;gamma;0)}:
Berechnet die Dichte an der Stelle $x$.

\item
\cmd{FatigueLifeDist(x;mu;beta;gamma;1)}:
Berechnet den Wert der Verteilungsfunktion an der Stelle $x$.

\item
\cmd{FatigueLifeDist(mu;beta;gamma)}:
Erzeugt eine Zufallszahl gemäß der Verteilung.

\end{itemize}



\section{Frechet-Verteilung mit Lageparameter \texorpdfstring{$\delta$}{delta}, Skalierungsparameter \texorpdfstring{$\beta$}{beta} und Formparameter \texorpdfstring{$\alpha$}{alpha}}

\begin{itemize}

\item
\cmd{FrechetDist(x;delta;beta;alpha;0)}:
Berechnet die Dichte an der Stelle $x$.

\item
\cmd{FrechetDist(x;delta;beta;alpha;1)}:
Berechnet den Wert der Verteilungsfunktion an der Stelle $x$.

\item
\cmd{FrechetDist(delta;beta;alpha)}:
Erzeugt eine Zufallszahl gemäß der Verteilung.

\end{itemize}



\section{Hyperbolische Sekanten-Verteilung mit Mittelwert \texorpdfstring{$a$}{a} und Standardabweichung \texorpdfstring{$b$}{b}}

\begin{itemize}

\item
\cmd{HyperbolicSecantDist(x;a;b;0)}:
Berechnet die Dichte an der Stelle $x$.

\item
\cmd{HyperbolicSecantDist(x;a;b;1)}:
Berechnet den Wert der Verteilungsfunktion an der Stelle $x$.

\item
\cmd{HyperbolicSecantDist(a;b)}:
Erzeugt eine Zufallszahl gemäß der Verteilung.

\end{itemize}



\section{Verteilung aus empirischen Daten}

\begin{itemize}

\item
\cmd{EmpirischeDichte(x;wert1;wert2;wert3;...;max)}:\\
Berechnet die Dichte an der Stelle $x$.
Die angegebenen Werte werden dabei auf den Bereich von 0 bis $max$ verteilt.

\item
\cmd{EmpirischeVerteilung(x;wert1;wert2;wert3;...;max)}:\\
Berechnet den Wert der Verteilungsfunktion an der Stelle $x$.
Die angegebenen Werte werden dabei auf den Bereich von 0 bis $max$ verteilt.

\item
\cmd{EmpirischeZufallszahl(wert1;wert2;wert3;...;max)}:\\
Erzeugt eine Zufallszahl gemäß der Verteilung.
Die angegebenen Werte werden dabei auf den Bereich von 0 bis $max$ verteilt.

\item
\cmd{EmpirischeVerteilungMittelwert(wert1;wert2;wert3;...;max)}:\\
Liefert den Erwartungswert der Verteilung.

\item
\cmd{EmpirischeVerteilungMedian(wert1;wert2;wert3;...;max)}:\\
Liefert den Median der Verteilung.

\item
\cmd{EmpirischeVerteilungQuantil(wert1;wert2;wert3;...;max;p)}:\\
Liefert das Quantil zur Wahrscheinlichkeit p der Verteilung.

\item
\cmd{EmpirischeVerteilungSD(wert1;wert2;wert3;...;max)}:\\
Liefert die Standardabweichung der Verteilung.

\item
\cmd{EmpirischeVerteilungVar(wert1;wert2;wert3;...;max)}:\\
Liefert die Varianz der Verteilung.

\item
\cmd{EmpirischeVerteilungCV(wert1;wert2;wert3;...;max)}:\\
Liefert den Variationskoeffizient der Verteilung.

\end{itemize}



\chapter{Erlang-C-Rechner}

Mit Hilfe des folgenden Befehls können verschiedene Kenngrößen auf Basis der
erweiterten Erlang-C-Formel berechnet werden:

\begin{itemize}

\item
\cmd{ErlangC(lambda;mu;nu;c;K;-1)}:\\
Berechnet die mittlere Warteschlangenlänge $\E[N_Q]$. 

\item
\cmd{ErlangC(lambda;mu;nu;c;K;-2)}:\\
Berechnet die mittlere Anzahl an Kunden im System $\E[N]$.

\item
\cmd{ErlangC(lambda;mu;nu;c;K;-3)}:\\
Berechnet die mittlere Wartezeit $\E[W]$.

\item
\cmd{ErlangC(lambda;mu;nu;c;K;-4)}:\\
Berechnet die mittlere Verweilzeit $\E[V]$.

\item
\cmd{ErlangC(lambda;mu;nu;c;K;-5)}:\\
Berechnet die mittlere Erreichbarkeit $1-P(A)$.

\item
\cmd{ErlangC(lambda;mu;nu;c;K;t)}:\\
Berechnet die Wahrscheinlichkeit für den Service-Level an der $t$-Sekunden-Schranke $P(W\le t)$.

\end{itemize}

Die Parameter haben dabei folgende Bedeutungen:
\begin{itemize}
\item
\cm{lambda}:\\
Ankunftsrate $\lambda$ (in Kunden pro Zeiteinheit), d.h.\ Kehrwert der mittleren Zwischenankunftszeit.
\item
\cm{mu}:\\
Bedienrate $\mu$ (in Kunden pro Zeiteinheit), d.h.\ Kehrwert der mittleren Bediendauer.
\item
\cm{nu}:\\
Abbruchrate $\nu$ (in Kunden pro Zeiteinheit), d.h.\ Kehrwert der mittleren Wartezeittoleranz.
\item
\cm{c}:\\
Anzahl an verfügbaren parallel arbeitenden Bedienern.
\item
\cm{K}:\\
Anzahl an verfügbaren Plätzen im System (Warte- und Bedienplätze zusammen, d.h.\ es gilt $K\ge c$).
\end{itemize}



\chapter{Allen-Cunneen-Approximationsformel}

Mit Hilfe des folgenden Befehls können verschiedene Kenngrößen auf Basis der
Allen-Cunneen-Approxi\-ma\-ti\-ons\-for\-mel berechnet werden:

\begin{itemize}

\item
\cmd{AllenCunneen(lambda;mu;cvI;cvS;c;-1)}:\\
Berechnet die mittlere Warteschlangenlänge $E[N_Q]$. 

\item
\cmd{AllenCunneen(lambda;mu;cvI;cvS;c;-2)}:\\
Berechnet die mittlere Anzahl an Kunden im System $\E[N]$.

\item
\cmd{AllenCunneen(lambda;mu;cvI;cvS;c;-3)}:\\
Berechnet die mittlere Wartezeit $\E[W]$.

\item
\cmd{AllenCunneen(lambda;mu;cvI;cvS;c;-4)}:\\
Berechnet die mittlere Verweilzeit $\E[V]$.
\end{itemize}

Die Parameter haben dabei folgende Bedeutungen:
\begin{itemize}
\item
\cm{lambda}:\\
Ankunftsrate $\lambda$ (in Kunden pro Zeiteinheit), d.h.\ Kehrwert der mittleren Zwischenankunftszeit.
\item
\cm{mu}:\\
Bedienrate $\mu$ (in Kunden pro Zeiteinheit), d.h.\ Kehrwert der mittleren Bediendauer.
\item
\cm{cvI}:\\
Variationskoeffizient der Zwischenankunftszeiten $\CV[I]$ (kleine Werte bedeuten, dass die Ankünfte sehr gleichmäßig erfolgen).
\item
\cm{cvS}:\\
Variationskoeffizient der Bedienzeiten $\CV[S]$ (kleine Werte bedeuten, dass die Bedienung sehr gleichmäßig erfolgen).
\item
\cm{c}:\\
Anzahl an verfügbaren parallel arbeitenden Bedienern.
\end{itemize}