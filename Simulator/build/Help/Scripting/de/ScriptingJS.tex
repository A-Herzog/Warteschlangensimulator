\part{Referenz der Javascript-Befehle}

An verschiedenen Stellen im Simulator können Skripte verwendet werden.
Als Skriptsprachen wird dabei entweder \textbf{Javascript} oder \textbf{Java} verwendet.

In diesem Abschnitt werden die zusätzlichen \textbf{Javascript}-Befehle, die den
Zugriff auf die Simulations- oder Statistikdaten ermöglichen und zur Ausgabe
der gefilterten Daten zur Verfügung stehen vorgestellt.



\chapter{\texttt{Statistics}-Objekt}

Das \cmd{Statistics}-Objekt ermöglicht den Lesezugriff auf alle Elemente der XML-Datei, die den Statistikdaten
zu Grunde liegt. Es ist nur verfügbar, wenn das Skript zum Filtern von Statistikdaten verwendet wird oder innerhalb
der Umgebung zur Ausführung von Parameterreihen-Javascripten verwendet wird. Während der
Simulation steht das \cmd{Statistics}-Objekt nicht zur Verfügung. Das Objekt stellt folgende Methoden zur Verfügung:

\section{Definition des Ausgabeformats}

\begin{itemize}

\item
\cmd{Statistics.setFormat("{}Format")}:\\
Über diesen Befehl kann das Format, in dem \cmd{Statistics.xml} Zahlen zur Ausgabe als Zeichenkette
formatiert, eingestellt werden. Es kann dabei für Zahlenwerte die lokaler Notation (im deutschsprachigen
Raum mit einem Dezimalkomma) oder die System-Notation mit einem Dezimalpunkt ausgegeben werden. Außerdem
kann angegeben werden, ob Zahlenwerte als Prozentangabe ausgegeben werden sollen. In diesem Fall wird
der Wert mit 100 multipliziert und ein ,,\%''-Zeichen an die Zahl angefügt. Voreingestellt ist stets die
Ausgabe in lokaler Notation und die Ausgabe als normale Fließkommazahl (also nicht als Prozentwert).

Folgende Parameter können \cmd{Statistics.setFormat} übergeben werden:
\begin{itemize}
\item
\cmd{System}:
Wahl der System-Notation für Zahlen und Prozentwerte.
\item
\cmd{Local}:
Wahl der lokalen Notation für Zahlen und Prozentwerte.
\item
\cmd{Fraction}:
Wahl der Ausgabe als normale Zahl (z.B.\ $0{,}357$ oder $0.375$).
\item
\cmd{Percent}:
Wahl der Ausgabe als Prozentwert (z.B.\ $35{,}7\%$ oder $35.7\%$).
\item
\cmd{Time}:
Ausgabe der Zahlenwerte als Zeitangaben (z.B.\ $00{:}03{:}25{,}87$).
\item
\cmd{Number}:
Ausgabe der Zahlenwerte als normale Zahlen (Ausgabe gemäß Angabe \cm{Percent} oder \cm{Fraction}).
\end{itemize}
	
\item
\cmd{Statistics.setSeparator("{}Format")}:\\
Über diesen Befehl kann eingestellt werden, durch welches Zeichen die einzelnen Einträge
einer Verteilung getrennt werden soll, wenn diese über \cmd{Statistics.xml} ausgegeben wird.
Vorgabe ist die Trennung durch ein Semikolon.

Folgende Parameter können \cmd{Statistics.setSeparator} übergeben werden:
\begin{itemize}
\item
\cmd{Semicolon}:
Semikolons als Trenner
\item
\cmd{Line}:
Zeilenumbrüche als Trenner
\item
\cmd{Tabs}:
Tabulatoren als Trenner
\end{itemize}

\end{itemize}

\section{Zugriff auf die Statistik-XML-Daten}

\begin{itemize}
	
\item
\cmd{Statistics.xml("{}Pfad")}:\\
Lädt das XML-Datenfeld, dessen Pfad als Parameter angegeben wurde und gibt dieses gemäß den Vorgaben,
die per \cmd{Statistics.setFormat} und \cmd{Statistics.setSeparator} eingestellt wurden,
als formatierte Zeichenkette zurück.
	
Beispiel:\\
\cm{var name=Statistics.xml("{}Modell->ModellName")}
  
\item
\cmd{Statistics.xmlNumber("{}Pfad")}:\\
Lädt das XML-Datenfeld, dessen Pfad als Parameter angegeben wurde und gibt den Inhalt als Zahl zurück.
Konnte das Feld nicht als Zahlenwert interpretiert werden, so wird eine Zeichenkette mit einer Fehlermeldung
zurückgegeben.  
	
\item
\cmd{Statistics.xmlArray("{}Pfad")}:\\
Lädt das XML-Datenfeld, dessen Pfad als Parameter angegeben wurde, interpretiert dieses als Verteilung
und gibt die Werte als Array aus Zahlenwerten zurück.
Konnte das Feld nicht als Verteilung interpretiert werden, so wird eine Zeichenkette mit einer Fehlermeldung
zurückgegeben.
	
Beispiel:\\
\cm{Statistics.xmlArray("{}StatistikBedienzeitenKunden->Kundentyp[Typ=$\backslash$"{}KundenA$\backslash$"]->\\{}[Verteilung]")}
  
\item
\cmd{Statistics.xmlSum("{}Pfad")}:\\
Lädt das XML-Datenfeld, dessen Pfad als Parameter angegeben wurde, interpretiert dieses als Verteilung,
summiert die Werte auf und liefert die Summe als Zahl zurück.
Konnte das Feld nicht als Verteilung interpretiert werden, so wird eine Zeichenkette mit einer Fehlermeldung
zurückgegeben.
	
Beispiel:\\
\cm{Statistics.xmlSum("{}StatistikBedienzeitenKunden->Kundentyp[Typ=$\backslash$"{}KundenA$\backslash$"]->\\{}[Verteilung]")}

\item
\cmd{Statistics.xmlMean("{}Pfad")}:\\
Lädt das XML-Datenfeld, dessen Pfad als Parameter angegeben wurde, interpretiert dieses als Verteilung,
bildet den Mittelwert der Werte und liefert diesen als Zahl zurück.
Konnte das Feld nicht als Verteilung interpretiert werden, so wird eine Zeichenkette mit einer Fehlermeldung
zurückgegeben.
	
Beispiel:\\
\cm{Statistics.xmlMean("{}StatistikBedienzeitenKunden->Kundentyp[Typ=$\backslash$"{}KundenA$\backslash$"]->\\{}[Verteilung]")}
  
\item
\cmd{Statistics.xmlSD("{}Pfad")}:\\
Lädt das XML-Datenfeld, dessen Pfad als Parameter angegeben wurde, interpretiert dieses als Verteilung,
bildet die Standardabweichung der Werte und liefert diesen als Zahl zurück.
Konnte das Feld nicht als Verteilung interpretiert werden, so wird eine Zeichenkette mit einer Fehlermeldung
zurückgegeben.

Beispiel:\\
\cm{Statistics.xmlSD("{}StatistikBedienzeitenKunden->Kundentyp[Typ=$\backslash$"{}KundenA$\backslash$"]->\\{}[Verteilung]")}

\item
\cmd{Statistics.xmlCV("{}Pfad")}:\\
Lädt das XML-Datenfeld, dessen Pfad als Parameter angegeben wurde, interpretiert dieses als Verteilung,
bildet den Variationskoeffizienten der Werte und liefert diesen als Zahl zurück.
Konnte das Feld nicht als Verteilung interpretiert werden, so wird eine Zeichenkette mit einer Fehlermeldung
zurückgegeben.
	
Beispiel:\\
\cm{Statistics.xmlCV("{}StatistikBedienzeitenKunden->Kundentyp[Typ=$\backslash$"{}KundenA$\backslash$"]->\\{}[Verteilung]")}

\item
\cmd{Statistics.translate("{}en")}:\\
Übersetzt die Statistikdaten ins Deutsche ("{}de") oder ins Englische ("{}en"), so dass jeweils die gewünschten
xml-Bezeichner verwendet werden können, auch wenn die Statistikdaten evtl.\ mit einer anderen Spracheinstellung
erstellt wurden.

\end{itemize}

\section{Speichern der Statistikdaten in Dateien}

\begin{itemize}

\item
\cmd{Statistics.save("{}Dateiname")}:\\
Speichert die kompletten Statistikdaten in der angegebenen Datei.\\
Diese Funktion steht nur in der Funktion zur Ausführung von Skripten zur Verfügung.

\item
\cmd{Statistics.saveNext("{}Pfad")}:\\
Speichert die kompletten Statistikdaten unter dem nächsten verfügbaren Dateinamen in dem angegebenen Verzeichnis.\\
Diese Funktion steht nur in der Funktion zur Ausführung von Skripten zur Verfügung.

\item
\cmd{Statistics.filter("{}Dateiname")}:\\
Wendet das angegebene Skript auf die Statistikdaten an und gibt das Ergebnis zurück.\\
Diese Funktion steht nur in der Funktion zur Ausführung von Skripten zur Verfügung.

\item
\cmd{Statistics.cancel()}:\\
Setzt den Abbruch-Status. (Nach einem Abbruch werden Dateiausgaben nicht mehr ausgeführt.)

\end{itemize}

\section{Zugriff auf das Modell}

\begin{itemize}

\item
\cmd{Statistics.getStationID("{}StationName")}:\\
Liefert die ID einer Station basierend auf dem Namen der Station.
Existiert keine Station mit dem passenden Namen, so liefert die Funktion -1.

\end{itemize}

\section{Abfrage der zugehörigen Statistikdatei}

\begin{itemize}

\item
\cmd{Statistics.getStatisticsFile()}:\\
Liefert den vollständigen Pfad- und Dateinamen der Statistikdatei, aus der die Daten stammen.
Stammen die Statistikdaten nicht aus einer Datei, so wird eine leere Zeichenkette zurück geliefert.

\item
\cmd{Statistics.getStatisticsFileName()}:\\
Liefert den Dateinamen der Statistikdatei, aus der die Daten stammen.
Stammen die Statistikdaten nicht aus einer Datei, so wird eine leere Zeichenkette zurück geliefert.

\end{itemize}



\chapter{\texttt{System}-Objekt}

Das \cmd{System}-Objekt ermöglicht den Zugriff auf einige allgemeine Programmfunktionen.
Es ist nur verfügbar, wenn das Skript zum Filtern von Statistikdaten verwendet wird oder
die Funktion zur Ausführung von Parameterreihen-Javascripten aktiv ist. Während der
Simulation steht das \cmd{System}-Objekt nicht zur Verfügung. Das Objekt stellt folgende Methoden zur Verfügung:

\begin{itemize}

\item
\cmd{System.calc("{}Ausdruck")}:\\
Berechnet den als Zeichenkette übergebenen Ausdruck mit Hilfe der Termauswertungsfunktion, die
auch an verschiedenen anderen Stellen im Programm zur Anwendung kommt (siehe Teil \ref{part:Rechenbefehle})	
und liefert das Ergebnis   als Zahl zurück. Konnte der Ausdruck nicht berechnet werden, so wird eine Fehlermeldung als
Zeichenkette zurückgeliefert. Die Termauswertung ermöglicht den Zugriff auf alle bekannten
Wahrscheinlichkeitsverteilungen, den Erlang-C-Rechner usw.

\item
\cmd{System.time()}:\\
Liefert die aktuelle Systemzeit als Millisekunden-Wert zurück. Diese Funktion kann zur Messung der
Laufzeit des Skriptes verwendet werden.

\item
\cmd{System.getInput("http://Adresse",-1)}:\\
Lädt einen Zahlenwert über die angegebene Adresse und liefert diesen zurück.
Wenn kein Wert geladen werden konnte, wird der im zweiten Parameter angegebene
Fehlerwert zurückgeliefert.

\item
\cmd{System.execute("program.exe")}:\\
Führt ein externes Programm aus und kehrt sofort zurück. Liefert true, wenn das Programm gestartet werden konnte.
Die Ausführung externer Programme durch Skripte ist standardmäßig deaktiviert
und muss zunächst im Einstellungen-Dialog aktiviert.

\item
\cmd{System.executeAndReturnOutput("program.exe")}:\\
Führt ein externes Programm aus und liefert die Ausgaben des Programms zurück.
Die Ausführung externer Programme durch Skripte ist standardmäßig deaktiviert
und muss zunächst im Einstellungen-Dialog aktiviert.

\item
\cmd{System.executeAndWait("program.exe")}:\\
Führt ein externes Programm aus und liefert den Rückgabecode des Programms zurück.
Im Fehlerfall wird -1 zurückgeliefert.
Die Ausführung externer Programme durch Skripte ist standardmäßig deaktiviert
und muss zunächst im Einstellungen-Dialog aktiviert.
	
\end{itemize}



\chapter{\texttt{Simulation}-Objekt}

Das \cmd{Simulation}-Objekt ermöglicht den Zugriff auf die aktuellen Simulationsdaten während
der Laufzeit der Simulation. Es ist nur verfügbar während die Simulation läuft und kann bei der
späteren Filterung der Ergebnisse nicht verwendet werden. Das Objekt stellt folgende Methoden zur Verfügung:

\section{Basisfunktionen}

\begin{itemize}

\item
\cmd{Simulation.time()}:\\
Liefert die aktuelle Zeit in der Simulation als Sekunden-Zahlenwert.

\item
\cmd{Simulation.calc("{}Ausdruck")}:\\
Berechnet den als Zeichenkette übergebenen Ausdruck mit Hilfe der Termauswertungsfunktion, die
auch an verschiedenen anderen Stellen im Programm zur Anwendung kommt (siehe Teil \ref{part:Rechenbefehle})	
und liefert das Ergebnis als Zahl zurück. Konnte der Ausdruck nicht berechnet werden, so wird eine Fehlermeldung als
Zeichenkette zurückgeliefert. Die Termauswertung ermöglicht den Zugriff auf alle bekannten
Wahrscheinlichkeitsverteilungen, den Erlang-C-Rechner usw.

\item
\cmd{Simulation.getInput("http://Adresse",-1)}:\\
Lädt einen Zahlenwert über die angegebene Adresse und liefert diesen zurück.
Wenn kein Wert geladen werden konnte, wird der im zweiten Parameter angegebene
Fehlerwert zurückgeliefert.

\item
\cmd{Simulation.execute("program.exe")}:\\
Führt ein externes Programm aus und kehrt sofort zurück. Liefert true, wenn das Programm gestartet werden konnte.
Die Ausführung externer Programme durch Skripte ist standardmäßig deaktiviert
und muss zunächst im Einstellungen-Dialog aktiviert.

\item
\cmd{Simulation.executeAndReturnOutput("program.exe")}:\\
Führt ein externes Programm aus und liefert die Ausgaben des Programms zurück.
Die Ausführung externer Programme durch Skripte ist standardmäßig deaktiviert
und muss zunächst im Einstellungen-Dialog aktiviert.

\item
\cmd{Simulation.executeAndWait("program.exe")}:\\
Führt ein externes Programm aus und liefert den Rückgabecode des Programms zurück.
Im Fehlerfall wird -1 zurückgeliefert.
Die Ausführung externer Programme durch Skripte ist standardmäßig deaktiviert
und muss zunächst im Einstellungen-Dialog aktiviert.

\item
\cmd{Simulation.isWarmUp()}:\\
Liefert wahr oder falsch zurück in Abhängigkeit, ob sich die Simulation noch in der Einschwingphase befindet.

\item
\cmd{Simulation.getMapLocal()}:\\
Liefert eine stations-lokale Zuordnung, in die Werte geschrieben und aus der Werte gelesen werden können.
Die hier gespeicherten Werte bleiben über die Ausführung des aktuellen Skriptes hinaus erhalten.

\item
\cmd{Simulation.getMapGlobal()}:\\
Liefert eine modellweite Zuordnung, in die Werte geschrieben und aus der Werte gelesen werden können.
Die hier gespeicherten Werte bleiben über die Ausführung des aktuellen Skriptes hinaus erhalten.

\end{itemize}

\section{Zugriff auf kundenspezifische Daten}

\begin{itemize}

\item
\cmd{Simulation.clientTypeName()}:\\
Liefert den Namen des Typs des Kunden, der die Verarbeitung des Skripts ausgelöst hat.\\
(Sofern das Ereignis direkt durch einen Kunden ausgelöst wurde.)

\item
\cmd{Simulation.isWarmUpClient()}:\\
Liefert wahr oder falsch zurück in Abhängigkeit, ob der Kunde während der Einschwingphase generiert wurde und
daher nicht in der Statistik erfasst werden soll.\\
(Sofern das Ereignis direkt durch einen Kunden ausgelöst wurde.)
	
\item
\cmd{Simulation.isClientInStatistics()}:\\
Liefert wahr oder falsch zurück in Abhängigkeit, ob der Kunde in der Statistik erfasst werden soll.
Diese Einstellung ist unabhängig von der Einschwingphase. Ein Kunde wird nur erfasst, wenn er außerhalb
der Einschwingphase generiert wurde und hier nicht falsch zurückgeliefert wird.\\
(Sofern das Ereignis direkt durch einen Kunden ausgelöst wurde.)
  
\item
\cmd{Simulation.setClientInStatistics(inStatistics)}:\\
Stellt ein, ob ein Kunde in der Statistik erfasst werden soll.
Diese Einstellung ist unabhängig von der Einschwingphase. Ein Kunde wird nur erfasst, wenn er außerhalb
der Einschwingphase generiert wurde und hier nicht falsch eingestellt wurde.\\
(Sofern das Ereignis direkt durch einen Kunden ausgelöst wurde.)
  
\item
\cmd{Simulation.clientNumber()}:\\
Liefert die bei 1 beginnende, fortlaufende Nummer des aktuellen Kunden.
Werden mehrere Simulationsthreads verwendet, so ist dieser Wert thread-lokal.\\
(Sofern das Ereignis direkt durch einen Kunden ausgelöst wurde.)

\item
\cmd{Simulation.clientWaitingSeconds()}:\\
Liefert die bisherige Wartezeit des Kunden, der die Verarbeitung des Skripts ausgelöst hat, als Sekunden-Zahlenwert zurück.\\
(Sofern das Ereignis direkt durch einen Kunden ausgelöst wurde.)

\item
\cmd{Simulation.clientWaitingTime()}:\\
Liefert die bisherige Wartezeit des Kunden, der die Verarbeitung des Skripts ausgelöst hat, als formatierte Zeitangabe als String zurück.\\
(Sofern das Ereignis direkt durch einen Kunden ausgelöst wurde.)

\item
\cmd{Simulation.clientWaitingSecondsSet(seconds)}:\\
Stellt die bisherige Wartezeit des Kunden, der die Verarbeitung des Skripts ausgelöst hat, ein.\\
(Sofern das Ereignis direkt durch einen Kunden ausgelöst wurde.)

\item
\cmd{Simulation.clientTransferSeconds()}:\\
Liefert die bisherige Transferzeit des Kunden, der die Verarbeitung des Skripts ausgelöst hat, als Sekunden-Zahlenwert zurück.\\
(Sofern das Ereignis direkt durch einen Kunden ausgelöst wurde.)

\item
\cmd{Simulation.clientTransferTime()}:\\
Liefert die bisherige Transferzeit des Kunden, der die Verarbeitung des Skripts ausgelöst hat, als formatierte Zeitangabe als String zurück.\\
(Sofern das Ereignis direkt durch einen Kunden ausgelöst wurde.)

\item
\cmd{Simulation.clientTransferSecondsSet(seconds)}:\\
Stellt die bisherige Transferzeit des Kunden, der die Verarbeitung des Skripts ausgelöst hat, ein.\\
(Sofern das Ereignis direkt durch einen Kunden ausgelöst wurde.)

\item
\cmd{Simulation.clientProcessSeconds()}:\\
Liefert die bisherige Bedienzeit des Kunden, der die Verarbeitung des Skripts ausgelöst hat, als Sekunden-Zahlenwert zurück.\\
(Sofern das Ereignis direkt durch einen Kunden ausgelöst wurde.)

\item
\cmd{Simulation.clientProcessTime()}:\\
Liefert die bisherige Bedienzeit des Kunden, der die Verarbeitung des Skripts ausgelöst hat, als formatierte Zeitangabe als String zurück.\\
(Sofern das Ereignis direkt durch einen Kunden ausgelöst wurde.)

\item
\cmd{Simulation.clientProcessSecondsSet(seconds)}:\\
Stellt die bisherige Bedienzeit des Kunden, der die Verarbeitung des Skripts ausgelöst hat, ein.\\
(Sofern das Ereignis direkt durch einen Kunden ausgelöst wurde.)

\item
\cmd{Simulation.clientResidenceSeconds()}:\\
Liefert die bisherige Verweilzeit des Kunden, der die Verarbeitung des Skripts ausgelöst hat, als Sekunden-Zahlenwert zurück.\\
(Sofern das Ereignis direkt durch einen Kunden ausgelöst wurde.)

\item
\cmd{Simulation.clientResidenceTime()}:\\
Liefert die bisherige Verweilzeit des Kunden, der die Verarbeitung des Skripts ausgelöst hat, als formatierte Zeitangabe als String zurück.\\
(Sofern das Ereignis direkt durch einen Kunden ausgelöst wurde.)

\item
\cmd{Simulation.clientResidenceSecondsSet(seconds)}:\\
Stellt die bisherige Verweilzeit des Kunden, der die Verarbeitung des Skripts ausgelöst hat, ein.\\
(Sofern das Ereignis direkt durch einen Kunden ausgelöst wurde.)

\item
\cmd{Simulation.getClientValue(Index)}:\\
Liefert den zu dem \cm{Index} für den aktuellen Kunden hinterlegten Zahlenwert.\\
(Sofern das Ereignis direkt durch einen Kunden ausgelöst wurde.)
  
\item
\cmd{Simulation.setClientValue(Index,Wert)}:\\
Stellt für den aktuellen Kunden für \cm{Index} den Zahlenwert \cm{Wert} ein.\\
(Sofern das Ereignis direkt durch einen Kunden ausgelöst wurde.)

\item
\cmd{Simulation.getClientText("{}Schlüssel")}:\\
Liefert die zu \cm{Schlüssel} für den aktuellen Kunden hinterlegte Zeichenkette.\\
(Sofern das Ereignis direkt durch einen Kunden ausgelöst wurde.)
  
\item
\cmd{Simulation.setClientText("{}Schlüssel","{}Wert")}:\\
Stellt für den aktuellen Kunden für \cm{Schlüssel} die Zeichenkette \cm{Wert} ein.\\
(Sofern das Ereignis direkt durch einen Kunden ausgelöst wurde.)

\end{itemize}

\section{Temporäre Batche}

Handelt es sich bei dem aktuellen Kunden um einen temporären Batch, so kann auf die
Eigenschaften der in ihm enthaltenen inneren Kunden lesend zugegriffen werden:

\begin{itemize}
\item
\cmd{Simulation.batchSize()}:\\
Liefert die Anzahl an Kunden, die sich in dem temporären Batch befinden.
Ist der aktuelle Kunden keine temporärer Batch, so liefert die Funktion 0.

\item
\cmd{Simulation.getBatchTypeName(batchIndex)}:\\
Liefert den Namen eines der Kunden in dem aktuellen Batch.
Der übergebene Index ist 0-basierend und muss im Bereich von 0 bis \cm{batchSize()-1} liegen.

\item
\cmd{Simulation.getBatchWaitingSeconds(batchIndex)}:\\
Liefert die bisherige Wartezeit eines der Kunden in dem aktuellen Batch in Sekunden als Zahlenwert.
Der übergebene Index ist 0-basierend und muss im Bereich von 0 bis \cm{batchSize()-1} liegen.

\item
\cmd{Simulation.getBatchWaitingTime(batchIndex)}:\\
Liefert die bisherige Wartezeit eines der Kunden in dem aktuellen Batch in formatierter Form als Zeichenkette.
Der übergebene Index ist 0-basierend und muss im Bereich von 0 bis \cm{batchSize()-1} liegen.

\item
\cmd{Simulation.getBatchTransferSeconds(batchIndex)}:\\
Liefert die bisherige Transferzeit eines der Kunden in dem aktuellen Batch in Sekunden als Zahlenwert.
Der übergebene Index ist 0-basierend und muss im Bereich von 0 bis \cm{batchSize()-1} liegen.

\item
\cmd{Simulation.getBatchTransferTime(batchIndex)}:\\
Liefert die bisherige Transferzeit eines der Kunden in dem aktuellen Batch in formatierter Form als Zeichenkette.
Der übergebene Index ist 0-basierend und muss im Bereich von 0 bis \cm{batchSize()-1} liegen.

\item
\cmd{Simulation.getBatchProcessSeconds(batchIndex)}:\\
Liefert die bisherige Bedienzeit eines der Kunden in dem aktuellen Batch in Sekunden als Zahlenwert.
Der übergebene Index ist 0-basierend und muss im Bereich von 0 bis \cm{batchSize()-1} liegen.

\item
\cmd{Simulation.getBatchProcessTime(batchIndex)}:\\
Liefert die bisherige Bedienzeit eines der Kunden in dem aktuellen Batch in formatierter Form als Zeichenkette.
Der übergebene Index ist 0-basierend und muss im Bereich von 0 bis \cm{batchSize()-1} liegen.
	 
\item
\cmd{Simulation.getBatchResidenceSeconds(batchIndex)}:\\
Liefert die bisherige Verweilzeit eines der Kunden in dem aktuellen Batch in Sekunden als Zahlenwert.
Der übergebene Index ist 0-basierend und muss im Bereich von 0 bis \cm{batchSize()-1} liegen.

\item
\cmd{Simulation.getBatchResidenceTime(batchIndex)}:\\
Liefert die bisherige Verweilzeit eines der Kunden in dem aktuellen Batch in formatierter Form als Zeichenkette.
Der übergebene Index ist 0-basierend und muss im Bereich von 0 bis \cm{batchSize()-1} liegen.

\item
\cmd{Simulation.getBatchValue(batchIndex, index)}:\\
Liefert einen zu einem der Kunden in dem aktuellen Batch einen gespeicherten Zahlenwert.
Der übergebene Batch-Index ist 0-basierend und muss im Bereich von 0 bis \cm{batchSize()-1} liegen.

\item
\cmd{Simulation.getBatchText(batchIndex, key)}:\\
Liefert zu einem der Kunden in dem aktuellen Batch einen gespeicherten Textwert.
Der übergebene Batch-Index ist 0-basierend und muss im Bereich von 0 bis \cm{batchSize()-1} liegen.

\end{itemize}

\section{Zugriff auf Parameter des Simulationsmodells}

\begin{itemize}

\item
\cmd{Simulation.set("{}Name",Wert)}:\\
Setzt die Simulationsvariable, deren Name im ersten Parameter angegeben wurde auf den im zweiten Parameter angegebenen Wert.
\cm{Wert} kann dabei eine Zahl oder eine Zeichenkette sein. Im Falle einer Zahl erfolgt eine direkte Zuweisung.
Zeichenketten werden gemäß \cm{Simulation.calc} interpretiert und das Ergebnis an die Variable zugewiesen. Bei \cm{Name}
muss es sich um entweder eine an anderer Stelle definierte Simulationsvariable handeln oder um ein Kundendaten-Feld der Form
\cm{ClientData(index)} mit $index\ge0$.

\item
\cmd{Simulation.setValue(id,Wert)}:\\
Stellt den Wert an dem ,,Analoger Wert''- oder ,,Tank''-Element mit der angegebenen \cm{id} ein.
  
\item
\cmd{Simulation.setRate(id,Wert)}:\\
Stellt die Änderungsrate (pro Sekunde) an dem ,,Analoger Wert''-Element mit der angegebenen \cm{id} ein.
  
\item
\cmd{Simulation.setValveMaxFlow(id,VentilNr,Wert)}:\\
Stellt den maximalen Durchfluss (pro Sekunde) an dem angegebenen Ventil (1-basierend) des ,,Tank''-Elements mit der angegebenen \cm{id} ein.
Der maximale Durchfluss muss dabei eine nichtnegative Zahl sein. 

\item
\cmd{Simulation.getWIP(id)}:\\
Liefert die aktuelle Anzahl an Kunden an der Station mit der angegebenen Id.
  
\item
\cmd{Simulation.getNQ(id)}:\\
Liefert die aktuelle Anzahl an Kunden in der Warteschlange an der Station mit der angegebenen Id.

\item
\cmd{Simulation.getWIP()}:\\
Liefert die aktuelle Anzahl an Kunden im System.
  
\item
\cmd{Simulation.getNQ()}:\\
Liefert die aktuelle Anzahl an im System wartenden Kunden.

\end{itemize}

\section{Zugriff auf den aktuellen Eingabewert}

\begin{itemize}

\item
\cmd{Simulation.getInput()}:\\
Erfolgt die Javascript-Verarbeitung innerhalb eines ,,Eingabe (Skript)''-Elements,
so kann über diese Funktion der aktuelle Eingabewert abgerufen werden. Andernfalls liefert diese Funktion lediglich den Wert 0. 

\end{itemize}

\section{Anzahl an Bedienern in einer Ressource}

\begin{itemize}

\item
\cmd{Simulation.getAllResourceCount()}:\\
Liefert die aktuelle Anzahl an Bedienern in allen Ressourcen zusammen.
  
\item
\cmd{Simulation.getResourceCount(id)}:\\
Liefert die aktuelle Anzahl an Bedienern in der Ressource mit der angegebenen Id.
  
\item
\cmd{Simulation.setResourceCount(id,count)}:\\
Stellt die Anzahl an Bedienern an der Ressource mit der angegebenen Id ein.
Damit die Anzahl an Bedienern in der Ressource zur Laufzeit verändert werden kann,
muss initial eine feste Anzahl an Bedienern (nicht unendliche viele und nicht über einen Zeitplan)
in der Ressource definiert sein. Außerdem dürfen keine Ausfälle für die Ressource eingestellt sein.
Die Funktion liefert \cm{true} zurück, wenn die Anzahl an Bedienern erfolgreich geändert
werden konnte. Wenn der neue Wert geringer als der bisherige Wert ist, so ist der neue Wert
evtl.\ nicht sofort im Simulationssystem ersichtlich, da eigentlich nicht mehr vorhandene Bediener
zunächst aktuelle Bedienungen zu Ende führen, bevor diese entfernt werden. 

\item
\cmd{Simulation.getAllResourceDown()}:\\
Liefert die aktuelle Anzahl an Bedienern in Ausfallzeit über alle Ressourcen.

\item
\cmd{Simulation.getResourceDown(id)}:\\
Liefert die aktuelle Anzahl an Bedienern in Ausfallzeit in der Ressource mit der angegebenen Id.

\end{itemize}

\section{Signale auslösen}

\begin{itemize}

\item
\cmd{Simulation.signal(name)}:\\
Löst das Signal mit dem angegebenen Namen aus.

\end{itemize}

\section{Meldung in Logging ausgeben}

\begin{itemize}

\item
\cmd{Simulation.log(message)}:\\
Gibt die übergebene Meldung im Logging-System aus (sofern eine Logaufzeichnung aktiviert ist).

\end{itemize}

\section{Kunden an Verzögerungen-Stationen freigeben}

Wurde an einer Verzögerung-Station eingestellt, dass eine Liste der aktuell dort befindlichen Kunden
mitgeführt werden soll, so kann diese Liste über die folgende Funktion abgefragt werden und es können
selektiv einzelne Kunden vor Ablauf ihrer angegebenen Verzögerungszeit freigegeben werden.

\begin{itemize}

\item
\cmd{getDelayStationData(id)}:\\
Liefert ein Objekt, welches die unter \textbf{Zugriff auf kundenspezifische Daten} angegebenen Methoden
zur Abfrage der Liste der Kunden an der Verzögerungs-Station \cm{id} bereitstellt. Ist die ID ungültig
so wird \cm{null} zurückgeliefert.

\end{itemize}



\chapter{\texttt{Clients}-Objekt}

Das \cmd{Clients}-Objekt steht nur innerhalb des Skript-Bedingung-Elements zur Verfügung
und hält alle Informationen zu den wartenden Kunden vor. Des Weiteren ermöglicht es,
einzelne Kunden freizugeben.

\begin{itemize}

\item
\cmd{Clients.count()}:\\
Liefert die Anzahl an wartenden Kunden. Bei den anderen Methode kann
über den Index-Parameter (Wert 0 bis \cm{count()}-1)auf einen bestimmten
Kunden zugegriffen werden.

\item
\cmd{Clients.clientTypeName(index)}:\\
Liefert den Namen des Typs des Kunden.

\item
\cmd{Clients.clientWaitingSeconds(index)}:\\
Liefert die bisherige Wartezeit des Kunden als Sekunden-Zahlenwert zurück.

\item
\cmd{Clients.clientWaitingTime(index)}:\\
Liefert die bisherige Wartezeit des Kunden als formatierte Zeitangabe als String zurück.

\item
\cmd{Clients.clientTransferSeconds(index)}:\\
Liefert die bisherige Transferzeit des Kunden als Sekunden-Zahlenwert zurück.

\item
\cmd{Clients.clientTransferTime(index)}:\\
Liefert die bisherige Transferzeit des Kunden als formatierte Zeitangabe als String zurück.

\item
\cmd{Clients.clientProcessSeconds(index)}:\\
Liefert die bisherige Bedienzeit des Kunden als Sekunden-Zahlenwert zurück.

\item
\cmd{Clients.clientProcessTime(index)}:\\
Liefert die bisherige Bedienzeit des Kunden als formatierte Zeitangabe als String zurück.

\item
\cmd{Clients.clientResidenceSeconds(index)}:\\
Liefert die bisherige Verweilzeit des Kunden als Sekunden-Zahlenwert zurück.

\item
\cmd{Clients.clientResidenceTime(index)}:\\
Liefert die bisherige Verweilzeit des Kunden als formatierte Zeitangabe als String zurück.

\item
\cmd{Clients.clientData(index,data)}:\\
Liefert das über den zweiten Parameter adressierte Datum des angegebenen Kunden zurück.

\item
\cmd{Clients.clientData(index,data,value)}:\\
Stellt den als dritten Parameter übergebenen Zahlenwert an der als zweiten Parameter adressierten Kundendatenposition des angegebenen Kunden ein.

\item
\cmd{Clients.clientTextData(index,key)}:\\
Liefert den Wert es über den zweiten Parameter adressierte Schlüssels des angegebenen Kunden zurück.

\item
\cmd{Clients.clientTextData(index,key,value)}:\\
Stellt den als dritten Parameter übergebenen Wert für den als zweiten Parameter adressierten Schlüssel des angegebenen Kunden ein.

\item
\cmd{Clients.release(index)}:\\
Veranlasst die Weiterleitung des angegebenen Kunden.
	
\end{itemize}



\chapter{\texttt{Output}-Objekt}

Das \cmd{Output}-Objekt stellt Funktionen zur Ausgabe der gefilterten Ergebnisse zur Verfügung:

\begin{itemize}

\item
\cmd{Output.setFormat("{}Format")}:\\
Über diesen Befehl kann das Format, in dem \cmd{Output.print} und \cmd{Output.println}
Zahlen formatieren, eingestellt werden. Es kann dabei für Zahlenwerte die lokaler Notation (im deutschsprachigen
Raum mit einem Dezimalkomma) oder die System-Notation mit einem Dezimalpunkt ausgegeben werden. Außerdem
kann angegeben werden, ob Zahlenwerte als Prozentangabe ausgegeben werden sollen. In diesem Fall wird
der Wert mit 100 multipliziert und ein "\%"-Zeichen an die Zahl angefügt. Voreingestellt ist stets die
Ausgabe in lokaler Notation und die Ausgabe als normale Fließkommazahl (also nicht als Prozentwert).

Folgende Parameter können \cm{Output.setFormat} übergeben werden:
\begin{itemize}
\item
\cmd{System}:\\
Wahl der System-Notation für Zahlen und Prozentwerte.
\item
\cmd{Local}:\\
Wahl der lokalen Notation für Zahlen und Prozentwerte.
\item
\cmd{Fraction}:\\
Wahl der Ausgabe als normale Zahl (z.B.\ $0{,}357$ oder $0.375$).
\item
\cmd{Percent}:\\
Wahl der Ausgabe als Prozentwert (z.B.\ $35,7\%$ oder $35.7\%$).
\item
\cmd{Number}:\\
Interpretation von Zahlenwerten als normale Zahlen (Dezimalwert oder Prozentwert).
\item
\cmd{Time}:\\
Interpretation von Zahlenwerten als Zeitangaben.

\end{itemize}

\item
\cmd{Output.setSeparator("{}Format")}:\\
Über diesen Befehl kann eingestellt werden, durch welches Zeichen die einzelnen Einträge
eines Arrays getrennt werden soll, wenn diese über \cmd{Output.print} oder
\cmd{Output.println} ausgegeben werden.
Vorgabe ist die Trennung durch ein Semikolon.

Folgende Parameter können \cmd{Output.setSeparator} übergeben werden:
\begin{itemize}
\item
\cmd{Semicolon}:\\
Semikolons als Trenner.
\item
\cmd{Line}:\\
Zeilenumbrüche als Trenner.
\item
\cmd{Tabs}:\\
Tabulatoren als Trenner.
\end{itemize}

\item
\cmd{Output.setDigits(digits)}:\\
Über diesen Befehl kann eingestellt werden, wie viele Nachkommastellen bei der
Ausgabe von Zahlen gemäß lokaler Notation ausgegeben werden sollen. Ein
negativer Wert bedeutet, dass alle verfügbaren Nachkommastellen ausgegeben.
(Bei Verwendung der System-Notation werden stets alle verfügbaren Nachkommastellen ausgegeben.)

\item
\cmd{Output.print("{}Ausdruck")}:\\
Gibt den übergebenen Ausdruck aus.
Zeichenketten werden direkt ausgegeben. Zahlenwerte werden gemäß den per
\cm{Output.setFormat} vorgenommenen Einstellungen formatiert.

\item
\cmd{Output.println("{}Ausdruck")}:\\
Gibt den übergebenen Ausdruck aus und fügt dabei einen Zeilenumbruch an.
Zeichenketten werden direkt ausgegeben. Zahlenwerte werden gemäß den per
\cm{Output.setFormat} vorgenommenen Einstellungen formatiert.

\item
\cmd{Output.newLine()}:\\
Gibt einen Zeilenumbruch aus. Diese Funktion ist gleichwertig zu dem Aufruf von\\
\cmd{Output.println("{}"{})}.

\item
\cmd{Output.tab()}:\\
Gibt einen Tabulator aus.

\item
\cmd{Output.cancel()}:\\
Setzt den Abbruch-Status. (Nach einem Abbruch werden Dateiausgaben nicht mehr ausgeführt.)

\item
\cmd{Output.printlnDDE("{}Arbeitsmappe","{}Tabelle","{}Zelle","{}Ausdruck")}:\\
Dieser  Befehl steht nur zur Verfügung, wenn DDE verfügbar ist, d.h. unter Windows.
Er gibt den übergebenen Ausdruck per DDE in der angegebenen Tabelle in Excel aus.
Zahlenwerte werden dabei gemäß den per \cm{Output.setFormat} vorgenommenen Einstellungen formatiert.

\end{itemize}



\chapter{\texttt{FileOutput}-Objekt}

Das \cmd{FileOutput}-Objekt stellt alle Funktionen, die auch das
\cmd{Output}-Objekt anbietet, zur Verfügung und ist nur während
der Parameterreihen-Skript-Ausführung verfügbar. Im Unterschied
zum \cmd{Output}-Objekt werden die Ausgaben nicht auf die Standardausgabe
geleitet, sondern es muss zunächst per \cmd{FileOutput.setFile("{}Dateiname")}
eine Ausgabedatei definiert werden. Alle Ausgaben werden dann an diese
Datei angehängt.



\chapter{\texttt{Model}-Objekt}

Das \cmd{Model}-Objekt steht nur während der Parameterreihen-Javascript-Ausführung
zur verfügbar und bietet Funktionen, um auf Modelleigenschaften zuzugreifen und
Simulationen zu initiieren.

\begin{itemize}

\item
\cmd{Model.reset()}:\\
Stellt das Modell auf den Ausgangszustand zurück.

\item
\cmd{Model.run()}:\\
Simuliert das aktuelle Modell.
Auf die Ergebnisse kann im Folgenden über das \cmd{Statistics}-Objekt zugegriffen werden.

\item
\cmd{Model.setDistributionParameter("{}Pfad",Index,Zahl)}:\\
Stellt einen Verteilungsparameter \cm{Index} (zwischen 1 und 4) der über \cm{Pfad}
angegebenen Wahrscheinlichkeitsverteilung ein.

\item
\cmd{Model.setMean("{}Pfad",Zahl)}:\\
Stellt den Mittelwert der über \cm{"{}Pfad"} angegebenen Wahrscheinlichkeitsverteilung auf den angegebenen Wert.

\item
\cmd{Model.setSD("{}Pfad",Zahl)}:\\
Stellt die Standardabweichung der über \cm{Pfad} angegebenen Wahrscheinlichkeitsverteilung auf den angegebenen Wert.

\item
\cmd{Model.setString("{}Pfad","{}Text")}:\\
Schreibt an die über \cm{Pfad} angegebene Stelle im Modell die angegebene Zeichenkette.

\item
\cmd{Model.setValue("{}Pfad",Zahl)}:\\
Schreibt an die über \cm{Pfad} angegebene Stelle im Modell den angegebenen Wert.

\item
\cmd{Model.xml("{}Pfad")}:\\
Liefert den über \cm{Pfad} erreichbaren Wert.\\
Diese Funktion ist das Äquivalent zu \cmd{Statistics.xml("{}Pfad")} für Modelldaten.

\item
\cmd{Model.getResourceCount("{}RessourcenName")}:\\
Liefert die Anzahl an Bedienern in der Ressource mit Namen \cm{RessourcenName}.
Existiert die Ressource nicht oder ist in ihr die Bedieneranzahl nicht als Zahlenwert
hinterlegt, so liefert die Funktion -1. Ansonsten die Anzahl an Bedienern in der
Ressource.

\item
\cmd{Model.setResourceCount("{}RessourcenName",Anzahl)}:\\
Stellt die Anzahl an Bedienern in der Ressource mit Namen \cm{RessourcenName} ein.

\item
\cmd{Model.getGlobalVariableInitialValue("{}VariablenName")}:\\
Liefert den Ausdruck zur Bestimmung des initialen Wertes der globalen Variable
\cm{VariablenName}. Existiert die globale Variable nicht, so wird
eine leere Zeichenkette geliefert.

\item
\cmd{Model.setGlobalVariableInitialValue("{}VariablenName","{}Ausdruck")}:\\
Stellt den Ausdruck zur Bestimmung des initialen Wertes der globalen Variable
\cm{VariablenName} ein.

\item
\cmd{Model.cancel()}:\\
Setzt den Abbruch-Status. (Nach einem Abbruch werden keine Simulationen mehr ausgeführt.)

\item
\cmd{Model.getStationID("{}StationName")}:\\
Liefert die ID einer Station basierend auf dem Namen der Station.
Existiert keine Station mit dem passenden Namen, so liefert die Funktion -1.

\end{itemize}



\chapter{XML-Auswahlbefehle}

Über die Parameter der Funktionen des \cmd{Statistics}-Objektes kann der Inhalt eines XML-Elements oder der Wert eines
Attributes eines XML-Elements ausgelesen werden. Die Selektion eines XML-Elements erfolgt dabei mehrstufig
getrennt durch "\texttt{->}"-Zeichen. Zwischen den "\texttt{->}"-Zeichen stehen jeweils die Namen von XML-Elementen.
Zusätzlich können in eckigen Klammern Namen und Werte von Attributen angegeben werden, nach denen gefiltert werden soll.

Beispiele:

\begin{itemize}

\item
\cmd{Statistics.xml("{}Modell->ModellName")}:\\
Liefert den Inhalt des Elements \cm{ModellName}, welches ein Unterelement von \cm{Modell} ist.	

\item
\cmd{Statistics.xml("{}StatistikZwischenankunftszeitenKunden->\\{}Station[Typ=$\backslash$"{}Quelle id=1$\backslash$"]->[Mittelwert]")}:\\
Selektiert das \cm{Station}-Unterelement des \cm{StatistikZwischenankunftszeitenKunden}-Elements, bei
dem das \cm{Typ}-Attribut auf den Wert \cm{Quelle id=1} gesetzt ist. Und liefert dann den Inhalt des Attributs
\cm{Mittelwert}.

\end{itemize}