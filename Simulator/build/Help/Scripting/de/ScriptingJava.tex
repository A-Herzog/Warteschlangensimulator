\part{Referenz der Java-Befehle}

An verschiedenen Stellen im Simulator können Skripte verwendet werden.
Als Skriptsprachen wird dabei entweder \textbf{Javascript} oder \textbf{Java} verwendet.

In diesem Abschnitt werden die zusätzlichen \textbf{Java}-Befehle, die den
Zugriff auf die Simulations- oder Statistikdaten ermöglichen und zur Ausgabe
der gefilterten Daten zur Verfügung stehen vorgestellt.

Der \textbf{Java}-Code muss in eine Methode der Form
\begin{verbatim}
void function(SimulationInterface sim) {

}
\end{verbatim}
eingebettet werden. Neben den Standardsprachbefehlen kann
abhängig vom Kontext, in dem das Skript ausgeführt wird,
über die Methoden des übergebenen \texttt{SimulationInterface}
weitere Interfaces, die ihrerseits weitere Methoden mitbringen,
auf die Simulations- oder Statistikdaten zugegriffen werden:

\chapter{\texttt{StatisticsInterface} abrufbar über \texttt{sim.getStatistics()}}

Das über \texttt{sim.getStatistics()} gelieferte \texttt{StatisticsInterface}-Interface
ermöglicht den Lesezugriff auf alle Elemente der XML-Datei, die den Statistikdaten
zu Grunde liegt. Es ist nur verfügbar, wenn das Skript zum Filtern von Statistikdaten verwendet wird oder innerhalb
der Umgebung zur Ausführung von Parameterreihen-Skripten verwendet wird. Während der
Simulation liefert \texttt{sim.getStatistics()} lediglich \texttt{null}.
Das Interface stellt folgende Methoden zur Verfügung:



\section{Definition des Ausgabeformats}

\begin{itemize}

\item
\cmd{void setFormat(final String format)}:\\
Über diesen Befehl kann das Format, in dem \cmd{Statistics.xml} Zahlen zur Ausgabe als Zeichenkette
formatiert, eingestellt werden. Es kann dabei für Zahlenwerte die lokaler Notation (im deutschsprachigen
Raum mit einem Dezimalkomma) oder die System-Notation mit einem Dezimalpunkt ausgegeben werden. Außerdem
kann angegeben werden, ob Zahlenwerte als Prozentangabe ausgegeben werden sollen. In diesem Fall wird
der Wert mit 100 multipliziert und ein ,,\%''-Zeichen an die Zahl angefügt. Voreingestellt ist stets die
Ausgabe in lokaler Notation und die Ausgabe als normale Fließkommazahl (also nicht als Prozentwert).
Folgende Parameter können \cmd{Statistics.setFormat} übergeben werden:
\begin{itemize}
\item
\cmd{System}:
Wahl der System-Notation für Zahlen und Prozentwerte.
\item
\cmd{Local}:
Wahl der lokalen Notation für Zahlen und Prozentwerte.
\item
\cmd{Fraction}:
Wahl der Ausgabe als normale Zahl (z.B.\ $0{,}357$ oder $0.375$).
\item
\cmd{Percent}:
Wahl der Ausgabe als Prozentwert (z.B.\ $35{,}7\%$ oder $35.7\%$).
\item
\cmd{Time}:
Ausgabe der Zahlenwerte als Zeitangaben (z.B.\ $00{:}03{:}25{,}87$).
\item
\cmd{Number}:
Ausgabe der Zahlenwerte als normale Zahlen (Ausgabe gemäß Angabe \cm{Percent} oder \cm{Fraction}).
\end{itemize}

\item
\cmd{void setSeparator(final String separator)}:\\
Über diesen Befehl kann eingestellt werden, durch welches Zeichen die einzelnen Einträge
einer Verteilung getrennt werden soll, wenn diese über \cmd{Statistics.xml} ausgegeben wird.
Vorgabe ist die Trennung durch ein Semikolon.
Folgende Parameter können \cmd{Statistics.setSeparator} übergeben werden:
\begin{itemize}
\item
\cmd{Semicolon}:
Semikolons als Trenner
\item
\cmd{Line}:
Zeilenumbrüche als Trenner
\item
\cmd{Tabs}:
Tabulatoren als Trenner
\end{itemize}

\end{itemize}

\section{Zugriff auf die Statistik-XML-Daten}

\begin{itemize}

\item
\cmd{String xml(final String path)}:\\
Lädt das XML-Datenfeld, dessen Pfad als Parameter angegeben wurde und gibt dies gemäß den Vorgaben,
die per \cm{sim.getStatistics().setFormat} und \cm{sim.getStatistics().setSeparator} eingestellt wurden,
als formatierte Zeichenkette zurück.

Beispiel: \cm{String name=sim.getStatistics().xml("{}Modell->ModellName")}

\item
\cmd{Object xmlNumber(final String path)}:\\
Lädt das XML-Datenfeld, dessen Pfad als Parameter angegeben wurde und gibt den Inhalt als \cm{Double}-Zahl zurück.
Konnte das Feld nicht als Zahlenwert interpretiert werden, so wird eine Zeichenkette mit einer Fehlermeldung
zurückgegeben.  

\item
\cmd{Object xmlArray(final String path)}:\\
Lädt das XML-Datenfeld, dessen Pfad als Parameter angegeben wurde, interpretiert dieses als Verteilung
und gibt die Werte als Array aus Zahlenwerten (\cm{double[]}) zurück.
Konnte das Feld nicht als Verteilung interpretiert werden, so wird eine Zeichenkette mit einer Fehlermeldung
zurückgegeben.

Beispiel:\\
\cm{sim.getStatistics().xmlArray("{}StatistikBedienzeitenKunden->\\{}Kundentyp[Typ=$\backslash$"{}KundenA$\backslash$"]->[Verteilung]")}

\item
\cmd{Object xmlSum(final String path)}:\\
Lädt das XML-Datenfeld, dessen Pfad als Parameter angegeben wurde, interpretiert dieses als Verteilung,
summiert die Werte auf und liefert die Summe als \cm{Double}-Zahl zurück.
Konnte das Feld nicht als Verteilung interpretiert werden, so wird eine Zeichenkette mit einer Fehlermeldung
zurückgegeben.

Beispiel:\\
\cm{sim.getStatistics().xmlSum("{}StatistikBedienzeitenKunden->\\{}Kundentyp[Typ=$\backslash$"{}KundenA$\backslash$"]->[Verteilung]")}

\item
\cmd{Object xmlMean(final String path)}:\\
Lädt das XML-Datenfeld, dessen Pfad als Parameter angegeben wurde, interpretiert dieses als Verteilung,
bildet den Mittelwert der Werte und liefert diesen als \cm{Double}-Zahl zurück.
Konnte das Feld nicht als Verteilung interpretiert werden, so wird eine Zeichenkette mit einer Fehlermeldung
zurückgegeben.

Beispiel:\\
\cm{sim.getStatistics().xmlMean("{}StatistikBedienzeitenKunden->\\{}Kundentyp[Typ=$\backslash$"{}KundenA$\backslash$"]->[Verteilung]")}

\item
\cmd{Object xmlSD(final String path)}:\\
Lädt das XML-Datenfeld, dessen Pfad als Parameter angegeben wurde, interpretiert dieses als Verteilung,
bildet die Standardabweichung der Werte und liefert diesen als \cm{Double}-Zahl zurück.
Konnte das Feld nicht als Verteilung interpretiert werden, so wird eine Zeichenkette mit einer Fehlermeldung
zurückgegeben.

Beispiel:\\
\cm{sim.getStatistics().xmlSD("{}StatistikBedienzeitenKunden->\\{}Kundentyp[Typ=$\backslash$"{}KundenA$\backslash$"]->[Verteilung]")}

\item
\cmd{Object xmlCV(final String path)}:\\
Lädt das XML-Datenfeld, dessen Pfad als Parameter angegeben wurde, interpretiert dieses als Verteilung,
bildet den Variationskoeffizienten der Werte und liefert diesen als \cm{Double}-Zahl zurück.
Konnte das Feld nicht als Verteilung interpretiert werden, so wird eine Zeichenkette mit einer Fehlermeldung
zurückgegeben.

Beispiel:\\
\cm{sim.getStatistics().xmlCV("{}StatistikBedienzeitenKunden->\\{}Kundentyp[Typ=$\backslash$"{}KundenA$\backslash$"]->[Verteilung]")}

\item
\cmd{boolean translate(final String language)}:\\
Übersetzt die Statistikdaten ins Deutsche ("{}de") oder ins Englische ("{}en"), so dass jeweils die gewünschten
xml-Bezeichner verwendet werden können, auch wenn die Statistikdaten evtl.\ mit einer anderen Spracheinstellung
erstellt wurden.

\end{itemize}

\section{Speichern der Statistikdaten in Dateien}

\begin{itemize}

\item
\cmd{boolean save(final String fileName)}:\\
Speichert die kompletten Statistikdaten in der angegebenen Datei.\\
Diese Funktion steht nur in der Funktion zur Ausführung von Skripten zur Verfügung.

\item
\cmd{boolean saveNext(final String folderName)}:\\
Speichert die kompletten Statistikdaten unter dem nächsten verfügbaren Dateinamen in dem angegebenen Verzeichnis.\\
Diese Funktion steht nur in der Funktion zur Ausführung von Skripten zur Verfügung.

\item
\cmd{String filter(final String fileName)}:\\
Wendet das angegebene Skript auf die Statistikdaten an und gibt das Ergebnis zurück.\\
Diese Funktion steht nur in der Funktion zur Ausführung von Skripten zur Verfügung.

\item
\cmd{void cancel()}:\\
Setzt den Abbruch-Status. (Nach einem Abbruch werden Dateiausgaben nicht mehr ausgeführt.)

\end{itemize}

\section{Zugriff auf das Modell}

\begin{itemize}

\item
\cmd{int getStationID(final String name)}:\\
Liefert die ID einer Station basierend auf dem Namen der Station.
Existiert keine Station mit dem passenden Namen, so liefert die Funktion -1.

\end{itemize}

\section{Abfrage der zugehörigen Statistikdatei}

\begin{itemize}

\item
\cmd{String getStatisticsFile()}:\\
Liefert den vollständigen Pfad- und Dateinamen der Statistikdatei, aus der die Daten stammen.
Stammen die Statistikdaten nicht aus einer Datei, so wird eine leere Zeichenkette zurück geliefert.

\item
\cmd{String getStatisticsFileName()}:\\
Liefert den Dateinamen der Statistikdatei, aus der die Daten stammen.
Stammen die Statistikdaten nicht aus einer Datei, so wird eine leere Zeichenkette zurück geliefert.

\end{itemize}



\chapter{\texttt{RuntimeInterface} abrufbar über \texttt{sim.getRuntime}}

Das \texttt{RuntimeInterface}-Interface ermöglicht den Zugriff auf einige allgemeine Programmfunktionen.
Es ist immer verfügbar. Das Interface stellt folgende Methoden zur Verfügung:

\begin{itemize}

\item
\cmd{Object calc(final String expression)}:\\
Berechnet den als Zeichenkette übergebenen Ausdruck mit Hilfe der Termauswertungsfunktion, die
auch an verschiedenen anderen Stellen im Programm zur Anwendung kommt (siehe Teil \ref{part:Rechenbefehle}) und liefert das Ergebnis
als \cm{Double}-Zahl zurück. Konnte der Ausdruck nicht berechnet werden, so wird eine Fehlermeldung als
Zeichenkette zurückgeliefert. Die Termauswertung ermöglicht den Zugriff auf alle bekannten
Wahrscheinlichkeitsverteilungen, den Erlang-C-Rechner usw.

\item
\cmd{long getTime()}:\\
Liefert die aktuelle Systemzeit als Millisekunden-Wert zurück. Diese Funktion kann zur Messung der
Laufzeit des Skriptes verwendet werden.

\item
\cmd{double getInput(final String url, final double errorValue)}:\\
Lädt einen Zahlenwert über die angegebene Adresse und liefert diesen zurück.
Wenn kein Wert geladen werden konnte, wird der im zweiten Parameter angegebene
Fehlerwert zurückgeliefert.

\item
\cmd{boolean execute(final String commandLine)}:\\
Führt ein externes Programm aus und kehrt sofort zurück. Liefert true, wenn das Programm gestartet werden konnte.
Die Ausführung externer Programme durch Skripte ist standardmäßig deaktiviert
und muss zunächst im Einstellungen-Dialog aktiviert.

\item
\cmd{String executeAndReturnOutput(final String commandLine)}:\\
Führt ein externes Programm aus und liefert die Ausgaben des Programms zurück.
Die Ausführung externer Programme durch Skripte ist standardmäßig deaktiviert
und muss zunächst im Einstellungen-Dialog aktiviert.

\item
\cmd{int executeAndWait(final String commandLine)}:\\
Führt ein externes Programm aus und liefert den Rückgabecode des Programms zurück.
Im Fehlerfall wird -1 zurückgeliefert.
Die Ausführung externer Programme durch Skripte ist standardmäßig deaktiviert
und muss zunächst im Einstellungen-Dialog aktiviert.

\end{itemize}



\chapter{\texttt{SystemInterface} abrufbar über \texttt{sim.getSystem()}}

Das \texttt{SystemInterface}-Interface ermöglicht den Zugriff auf die aktuellen Simulationsdaten während
der Laufzeit der Simulation. Es ist nur verfügbar während die Simulation läuft und kann bei der
späteren Filterung der Ergebnisse nicht verwendet werden. Das Interface stellt folgende Methoden zur Verfügung:

\section{Basisfunktionen}

\begin{itemize}

\item
\cmd{double getTime()}:\\
Liefert die aktuelle Zeit in der Simulation als Sekunden-Zahlenwert.

\item
\cmd{Object calc(final String expression)}:\\
Berechnet den als Zeichenkette übergebenen Ausdruck mit Hilfe der Termauswertungsfunktion, die
auch an verschiedenen anderen Stellen im Programm zur Anwendung kommt (siehe Teil \ref{part:Rechenbefehle}) und liefert das Ergebnis
als \cm{Double}-Zahl zurück. Konnte der Ausdruck nicht berechnet werden, so wird eine Fehlermeldung als
Zeichenkette zurückgeliefert. Die Termauswertung ermöglicht den Zugriff auf alle bekannten
Wahrscheinlichkeitsverteilungen, den Erlang-C-Rechner usw.
  
\item
\cmd{boolean isWarmUp()}:\\
Liefert wahr oder falsch zurück in Abhängigkeit, ob sich die Simulation noch in der Einschwingphase befindet.

\item
\cmd{Map<String,Object> getMapLocal()}:\\
Liefert eine stations-lokale Zuordnung, in die Werte geschrieben und aus der Werte gelesen werden können.
Die hier gespeicherten Werte bleiben über die Ausführung des aktuellen Skriptes hinaus erhalten.

\item
\cmd{Map<String,Object> getMapGlobal()}:\\
Liefert eine modellweite Zuordnung, in die Werte geschrieben und aus der Werte gelesen werden können.
Die hier gespeicherten Werte bleiben über die Ausführung des aktuellen Skriptes hinaus erhalten.

\item
\cmd{void pauseAnimation()}:\\
Schaltet die Animation in den Einzelschrittmodus. Wird die Animation bereits im Einzelschrittmodus
ausgeführt oder aber wird das Modell als Simulation ausgeführt, so hat dieser Befehl keine Wirkung.

\item
\cmd{void terminateSimulation(final String message)}:\\
Terminates the simulation. If \cm{null} is passed as message, the simulation is terminated normally.
In case of a message, the simulation will be terminated with the corresponding error message.

\end{itemize}

\section{Zugriff auf Parameter des Simulationsmodells}

\begin{itemize}

\item
\cmd{void set(final String varName, final Object varValue)}:\\
Setzt die Simulationsvariable, deren Name im ersten Parameter angegeben wurde auf den im zweiten Parameter angegebenen Wert.
\cm{varValue} kann dabei eine Zahl oder eine Zeichenkette sein. Im Falle einer Zahl erfolgt eine direkte Zuweisung.
Zeichenketten werden gemäß \cm{calc(final String expression)} interpretiert und das Ergebnis an die Variable zugewiesen. Bei \cm{varName}
muss es sich um entweder eine an anderer Stelle definierte Simulationsvariable handeln oder um ein Kundendaten-Feld der Form
\cm{ClientData(index)} mit $index\ge0$. 
  
\item
\cmd{void setAnalogValue(final Object elementID, final Object value)}:\\
Stellt den Wert an dem ,,Analoger Wert''- oder ,,Tank''-Element mit der angegebenen Id ein.
  
\item
\cmd{void setAnalogRate(final Object elementID, final Object value)}:\\
Stellt die Änderungsrate (pro Sekunde) an dem ,,Analoger Wert''-Element mit der angegebenen Id ein.
  
\item
\cmd{void setAnalogValveMaxFlow(final Object elementID, final Object valveNr,\\final Object value)}:\\
Stellt den maximalen Durchfluss (pro Sekunde) an dem angegebenen Ventil (1-basierend) des ,,Tank''-Elements mit der angegebenen Id ein.
Der maximale Durchfluss muss dabei eine nichtnegative Zahl sein. 
  
\item
\cmd{int getWIP(final int id)}:\\
Liefert die aktuelle Anzahl an Kunden an der Station mit der angegebenen Id.
  
\item
\cmd{int getNQ(final int id)}:\\
Liefert die aktuelle Anzahl an Kunden in der Warteschlange an der Station mit der angegebenen Id.

\item
\cmd{int getWIP()}:\\
Liefert die aktuelle Anzahl an Kunden im System.
  
\item
\cmd{int getNQ()}:\\
Liefert die aktuelle Anzahl an im System wartenden Kunden.

\end{itemize}

\section{Anzahl an Bedienern in einer Ressource}

\begin{itemize}

\item
\cmd{int getAllResourceCount()}:\\
Liefert die aktuelle Anzahl an Bedienern in allen Ressourcen zusammen.
  
\item
\cmd{int getResourceCount(final int resourceId)}:\\
Liefert die aktuelle Anzahl an Bedienern in der Ressource mit der angegebenen Id.
  
\item
\cmd{boolean setResourceCount(final int resourceId, final int count)}:\\
Stellt die Anzahl an Bedienern an der Ressource mit der angegebenen Id ein.
Damit die Anzahl an Bedienern in der Ressource zur Laufzeit verändert werden kann,
muss initial eine feste Anzahl an Bedienern (nicht unendliche viele und nicht über einen Zeitplan)
in der Ressource definiert sein. Außerdem dürfen keine Ausfälle für die Ressource eingestellt sein.
Die Funktion liefert \cm{true} zurück, wenn die Anzahl an Bedienern erfolgreich geändert
werden konnte. Wenn der neue Wert geringer als der bisherige Wert ist, so ist der neue Wert
evtl.\ nicht sofort im Simulationssystem ersichtlich, da eigentlich nicht mehr vorhandene Bediener
zunächst aktuelle Bedienungen zu Ende führen, bevor diese entfernt werden.

\item
\cmd{int getAllResourceDown()}:\\
Liefert die aktuelle Anzahl an Bedienern in Ausfallzeit über alle Ressourcen.

\item
\cmd{int getResourceDown(final int resourceId)}:\\
Liefert die aktuelle Anzahl an Bedienern in Ausfallzeit in der Ressource mit der angegebenen Id.

\end{itemize}

\section{Signale auslösen}

\begin{itemize}

\item
\cmd{signal(final String signalName)}:\\
Löst das Signal mit dem angegebenen Namen aus.

\end{itemize}

\section{Externen Code aufrufen}

\begin{itemize}

\item
\cmd{Object runPlugin(final String className, final String functionName,\\final Object data)}:\\
Ruft in der angegebenen Klasse die angegebene Methode auf und übermittelt an diese den in
\cm{data} angegebenen optionalen Parameter. Der Rückgabewert der Methode wird von \cm{runPlugin}
zurückgegeben. Schlägt der Aufruf fehl, so liefert \cm{runPlugin} den Wert \cm{null} zurück.
\end{itemize}

\section{Meldung in Logging ausgeben}

\begin{itemize}

\item
\cmd{void log(final Object obj)}:\\
Gibt die übergebene Meldung im Logging-System aus (sofern eine Logaufzeichnung aktiviert ist).

\end{itemize}

\section{Kunden an Verzögerungen-Stationen freigeben}

Wurde an einer Verzögerung-Station eingestellt, dass eine Liste der aktuell dort befindlichen Kunden
mitgeführt werden soll, so kann diese Liste über die folgende Funktion abgefragt werden und es können
selektiv einzelne Kunden vor Ablauf ihrer angegebenen Verzögerungszeit freigegeben werden.

\begin{itemize}

\item
\cmd{ClientsInterface getDelayStationData(final int id)}:\\
Liefert ein Objekt, welches das Interface \cm{ClientsInterface} implementiert und die Liste
der Kunden an Verzögerungs-Station \cm{id} repräsentiert. Ist die ID ungültig
so wird \cm{null} zurückgeliefert. 
\end{itemize}



\chapter{\texttt{ClientInterface} abrufbar über \texttt{sim.getClient()}}

Das \texttt{ClientInterface}-Interface ermöglicht den Zugriff auf die Simulationsdaten des aktuellen Kunden während
der Laufzeit der Simulation. Es ist nur verfügbar während die Simulation läuft die Verarbeitung durch
einen Kunden ausgelöst wurde. Das Interface stellt folgende Methoden zur Verfügung:

\begin{itemize}

\item
\cmd{Object calc(final String expression)}:\\
Berechnet den als Zeichenkette übergebenen Ausdruck mit Hilfe der Termauswertungsfunktion, die
auch an verschiedenen anderen Stellen im Programm zur Anwendung kommt (siehe Teil \ref{part:Rechenbefehle}) und liefert das Ergebnis
als \cm{Double}-Zahl zurück. Konnte der Ausdruck nicht berechnet werden, so wird eine Fehlermeldung als
Zeichenkette zurückgeliefert. Die Termauswertung ermöglicht den Zugriff auf alle bekannten
Wahrscheinlichkeitsverteilungen, den Erlang-C-Rechner usw.

\item
\cmd{String getTypeName()}:\\
Liefert den Namen des Typs des Kunden, der die Verarbeitung des Skripts ausgelöst hat.
    
\item
\cmd{boolean isWarmUp()}:\\
Liefert wahr oder falsch zurück in Abhängigkeit, ob der Kunde während der Einschwingphase generiert wurde und
daher nicht in der Statistik erfasst werden soll.
  
\item
\cmd{boolean isInStatistics()}:\\
Liefert wahr oder falsch zurück in Abhängigkeit, ob der Kunde in der Statistik erfasst werden soll.
Diese Einstellung ist unabhängig von der Einschwingphase. Ein Kunde wird nur erfasst, wenn er außerhalb
der Einschwingphase generiert wurde und hier nicht falsch zurückgeliefert wird.
  
\item
\cmd{void setInStatistics(final boolean inStatistics)}:\\
Stellt ein, ob ein Kunde in der Statistik erfasst werden soll.
Diese Einstellung ist unabhängig von der Einschwingphase. Ein Kunde wird nur erfasst, wenn er außerhalb
der Einschwingphase generiert wurde und hier nicht falsch eingestellt wurde.
  
\item
\cmd{long getNumber()}:\\
Liefert die bei 1 beginnende, fortlaufende Nummer des aktuellen Kunden.
Werden mehrere Simulationsthreads verwendet, so ist dieser Wert Thread-lokal.
  
\item
\cmd{double getWaitingSeconds()}:\\
Liefert die bisherige Wartezeit des Kunden, der die Verarbeitung des Skripts ausgelöst hat, als Sekunden-Zahlenwert zurück.
  
\item
\cmd{String getWaitingTime()}:\\
Liefert die bisherige Wartezeit des Kunden, der die Verarbeitung des Skripts ausgelöst hat, als formatierte Zeitangabe als String zurück.
  
\item
\cmd{void setWaitingSeconds(final double seconds)}:\\
Stellt die bisherige Wartezeit des Kunden, der die Verarbeitung des Skripts ausgelöst hat, ein.
  
\item
\cmd{double getTransferSeconds()}:\\
Liefert die bisherige Transferzeit des Kunden, der die Verarbeitung des Skripts ausgelöst hat, als Sekunden-Zahlenwert zurück.
  
\item
\cmd{String getTransferTime()}:\\
Liefert die bisherige Transferzeit des Kunden, der die Verarbeitung des Skripts ausgelöst hat, als formatierte Zeitangabe als String zurück.
  
\item
\cmd{void setTransferSeconds(final double seconds)}:\\
Stellt die bisherige Transferzeit des Kunden, der die Verarbeitung des Skripts ausgelöst hat, ein.
  
\item
\cmd{double getProcessSeconds()}:\\
Liefert die bisherige Bedienzeit des Kunden, der die Verarbeitung des Skripts ausgelöst hat, als Sekunden-Zahlenwert zurück.
  
\item
\cmd{String getProcessTime()}:\\
Liefert die bisherige Bedienzeit des Kunden, der die Verarbeitung des Skripts ausgelöst hat, als formatierte Zeitangabe als String zurück.
  
\item
\cmd{void setProcessSeconds(final double seconds)}:\\
Stellt die bisherige Bedienzeit des Kunden, der die Verarbeitung des Skripts ausgelöst hat, ein.

\item
\cmd{double getResidenceSeconds()}:\\
Liefert die bisherige Verweilzeit des Kunden, der die Verarbeitung des Skripts ausgelöst hat, als Sekunden-Zahlenwert zurück.

\item
\cmd{String getResidenceTime()}:\\
Liefert die bisherige Verweilzeit des Kunden, der die Verarbeitung des Skripts ausgelöst hat, als formatierte Zeitangabe als String zurück.

\item
\cmd{void setResidenceSeconds(final double seconds)}:\\
Stellt die bisherige Verweilzeit des Kunden, der die Verarbeitung des Skripts ausgelöst hat, ein.

\item
\cmd{double getValue(final int index)}:\\
Liefert den zu dem \cm{index} für den aktuellen Kunden hinterlegten Zahlenwert.
  
\item
\cmd{void setValue(final int index, final int value)},\\
\cmd{void setValue(final int index, final double value)},\\
\cmd{void setValue(final int index, final String value)}:\\
Stellt für den aktuellen Kunden für \cm{index} den Wert \cm{value} ein.
Wird als \cm{value} eine Zeichenkette übergeben, so wird diese zunächst über die
\cm{calc(final String expression)}-Funktion ausgewertet.
  
\item
\cmd{String getText(final String key)}:\\
Liefert die zu \cm{key} für den aktuellen Kunden hinterlegte Zeichenkette.
  
\item
\cmd{void setText(final String key, final String value)}:\\
Stellt für den aktuellen Kunden für \cm{key} die Zeichenkette \cm{value} ein.
	
\end{itemize}

\section{Temporäre Batche}

Handelt es sich bei dem aktuellen Kunden um einen temporären Batch, so kann auf die
Eigenschaften der in ihm enthaltenen inneren Kunden lesend zugegriffen werden:

\begin{itemize}

\item
\cmd{int batchSize()}:\\
Liefert die Anzahl an Kunden, die sich in dem temporären Batch befinden.
Ist der aktuelle Kunden keine temporärer Batch, so liefert die Funktion 0.

\item
\cmd{String getBatchTypeName(final int batchIndex)}:\\
Liefert den Namen eines der Kunden in dem aktuellen Batch.
Der übergebene Index ist 0-basierend und muss im Bereich von 0 bis \cm{batchSize()-1} liegen.

\item
\cmd{double getBatchWaitingSeconds(final int batchIndex)}:\\
Liefert die bisherige Wartezeit eines der Kunden in dem aktuellen Batch in Sekunden als Zahlenwert.
Der übergebene Index ist 0-basierend und muss im Bereich von 0 bis \cm{batchSize()-1} liegen.
	

\item
\cmd{String getBatchWaitingTime(final int batchIndex)}:\\
Liefert die bisherige Wartezeit eines der Kunden in dem aktuellen Batch in formatierter Form als Zeichenkette.
Der übergebene Index ist 0-basierend und muss im Bereich von 0 bis \cm{batchSize()-1} liegen.

\item
\cmd{double getBatchTransferSeconds(final int batchIndex)}:\\
Liefert die bisherige Transferzeit eines der Kunden in dem aktuellen Batch in Sekunden als Zahlenwert.
Der übergebene Index ist 0-basierend und muss im Bereich von 0 bis \cm{batchSize()-1} liegen.

\item
\cmd{String getBatchTransferTime(final int batchIndex)}:\\
Liefert die bisherige Transferzeit eines der Kunden in dem aktuellen Batch in formatierter Form als Zeichenkette.
Der übergebene Index ist 0-basierend und muss im Bereich von 0 bis \cm{batchSize()-1} liegen.

\item
\cmd{double getBatchProcessSeconds(final int batchIndex)}:\\
Liefert die bisherige Bedienzeit eines der Kunden in dem aktuellen Batch in Sekunden als Zahlenwert.
Der übergebene Index ist 0-basierend und muss im Bereich von 0 bis \cm{batchSize()-1} liegen.

\item
\cmd{String getBatchProcessTime(final int batchIndex)}:\\
Liefert die bisherige Bedienzeit eines der Kunden in dem aktuellen Batch in formatierter Form als Zeichenkette.
Der übergebene Index ist 0-basierend und muss im Bereich von 0 bis \cm{batchSize()-1} liegen.
	 
\item
\cmd{double getBatchResidenceSeconds(final int batchIndex)}:\\
Liefert die bisherige Verweilzeit eines der Kunden in dem aktuellen Batch in Sekunden als Zahlenwert.
Der übergebene Index ist 0-basierend und muss im Bereich von 0 bis \cm{batchSize()-1} liegen.

\item
\cmd{String getBatchResidenceTime(final int batchIndex)}:\\
Liefert die bisherige Verweilzeit eines der Kunden in dem aktuellen Batch in formatierter Form als Zeichenkette.
Der übergebene Index ist 0-basierend und muss im Bereich von 0 bis \cm{batchSize()-1} liegen.

\item
\cmd{double getBatchValue(final int batchIndex, final int index)}:\\
Liefert einen zu einem der Kunden in dem aktuellen Batch einen gespeicherten Zahlenwert.
Der übergebene Batch-Index ist 0-basierend und muss im Bereich von 0 bis \cm{batchSize()-1} liegen.

\item
\cmd{String getBatchText(final int batchIndex, final String key)}:\\
Liefert zu einem der Kunden in dem aktuellen Batch einen gespeicherten Textwert.
Der übergebene Batch-Index ist 0-basierend und muss im Bereich von 0 bis \cm{batchSize()-1} liegen.
  
\end{itemize}



\chapter{\texttt{InputValueInterface} abrufbar über \texttt{sim.getInputValue()}}

Das \texttt{InputValueInterface}-Interface ermöglicht das Abrufen des nächsten Eingabewertes,
sofern die Skript-Verarbeitung innerhalb eines Eingabe (Skript)-Elements
angestoßen wurde. Das Interface stellt folgende Methode zur Verfügung:

\begin{itemize}
\item
\cm{double get()}:\\
Über diese Funktion kann der aktuelle Eingabewert abgerufen werden.
\end{itemize}



\chapter{\texttt{ClientsInterface} abrufbar über \texttt{sim.getClients()}}

Das \texttt{ClientsInterface}-Interface steht nur innerhalb des Skript-Bedingung-Elements zur Verfügung
und hält alle Informationen zu den wartenden Kunden vor. Des Weiteren ermöglicht es, einzelne Kunden freizugeben.

\begin{itemize}

\item
\cmd{int count()}:\\
Liefert die Anzahl an wartenden Kunden. Bei den anderen Methode kann
über den Index-Parameter (Wert 0 bis \cm{count()}-1)auf einen bestimmten
Kunden zugegriffen werden.

\item
\cmd{String clientTypeName(final int index)}:\\
Liefert den Namen des Typs des Kunden.

\item
\cmd{double clientWaitingSeconds(final int index)}:\\
Liefert die bisherige Wartezeit des Kunden als Sekunden-Zahlenwert zurück.

\item
\cmd{String clientWaitingTime(final int index)}:\\
Liefert die bisherige Wartezeit des Kunden als formatierte Zeitangabe als String zurück.

\item
\cmd{double clientTransferSeconds(final int index)}:\\
Liefert die bisherige Transferzeit des Kunden als Sekunden-Zahlenwert zurück.

\item
\cmd{String clientTransferTime(final int index)}:\\
Liefert die bisherige Transferzeit des Kunden als formatierte Zeitangabe als String zurück.

\item
\cmd{double clientProcessSeconds(final int index)}:\\
Liefert die bisherige Bedienzeit des Kunden als Sekunden-Zahlenwert zurück.

\item
\cmd{String clientProcessTime(final int index)}:\\
Liefert die bisherige Bedienzeit des Kunden als formatierte Zeitangabe als String zurück.

\item
\cmd{double clientResidenceSeconds(final int index)}:\\
Liefert die bisherige Verweilzeit des Kunden als Sekunden-Zahlenwert zurück.

\item
\cmd{String clientResidenceTime(final int index)}:\\
Liefert die bisherige Verweilzeit des Kunden als formatierte Zeitangabe als String zurück.

\item
\cmd{double clientData(final int index, final int data)}:\\
Liefert das über den zweiten Parameter adressierte Datum des angegebenen Kunden zurück.

\item
\cmd{void clientData(final int index, final int data, final double value)}:\\
Stellt den als dritten Parameter übergebenen Zahlenwert an der als zweiten Parameter adressierten Kundendatenposition des angegebenen Kunden ein.

\item
\cmd{String clientTextData(final int index, final String key)}:\\
Liefert den Wert es über den zweiten Parameter adressierte Schlüssels des angegebenen Kunden zurück.

\item
\cmd{String clientTextData(final int index, final String key, final String value)}:\\
Stellt den als dritten Parameter übergebenen Wert für den als zweiten Parameter adressierten Schlüssel des angegebenen Kunden ein.

\item
\cmd{void release(final int index)}:\\
Veranlasst die Weiterleitung des angegebenen Kunden.

\end{itemize}



\chapter{\texttt{OutputInterface} abrufbar über \texttt{sim.getOutput()}}

Das \texttt{OutputInterface}-Interface stellt Funktionen zur Ausgabe der gefilterten Ergebnisse zur Verfügung:

\begin{itemize}

\item
\cmd{void setFormat(final String format)}:\\
Über diesen Befehl kann das Format, in dem \cm{print} und \cm{println}
Zahlen formatieren, eingestellt werden. Es kann dabei für Zahlenwerte die lokaler Notation (im deutschsprachigen
Raum mit einem Dezimalkomma) oder die System-Notation mit einem Dezimalpunkt ausgegeben werden. Außerdem
kann angegeben werden, ob Zahlenwerte als Prozentangabe ausgegeben werden sollen. In diesem Fall wird
der Wert mit 100 multipliziert und ein "\%"-Zeichen an die Zahl angefügt. Voreingestellt ist stets die
Ausgabe in lokaler Notation und die Ausgabe als normale Fließkommazahl (also nicht als Prozentwert).

Folgende Parameter können \cm{setFormat} übergeben werden:	
\begin{itemize}
\item
\cmd{System}:\\
Wahl der System-Notation für Zahlen und Prozentwerte.
\item
\cmd{Local}:\\
Wahl der lokalen Notation für Zahlen und Prozentwerte.
\item
\cmd{Fraction}:\\
Wahl der Ausgabe als normale Zahl (z.B.\ $0{,}357$ oder $0.375$).
\item
\cmd{Percent}:\\
Wahl der Ausgabe als Prozentwert (z.B.\ $35,7\%$ oder $35.7\%$).
\item
\cmd{Number}:\\
Interpretation von Zahlenwerten als normale Zahlen (Dezimalwert oder Prozentwert).
\item
\cmd{Time}:\\
Interpretation von Zahlenwerten als Zeitangaben.

\end{itemize}

\item
\cmd{void setSeparator(final String separator)}:\\
Über diesen Befehl kann eingestellt werden, durch welches Zeichen die einzelnen Einträge
eines Arrays getrennt werden soll, wenn diese über \cm{print} oder
\cm{println} ausgegeben werden.
Vorgabe ist die Trennung durch ein Semikolon.

Folgende Parameter können \cm{setSeparator} übergeben werden:
\begin{itemize}
\item
\cmd{Semicolon}:\\
Semikolons als Trenner.
\item
\cmd{Line}:\\
Zeilenumbrüche als Trenner.
\item
\cmd{Tabs}:\\
Tabulatoren als Trenner.
\end{itemize}

\item
\cmd{void setDigits(final int digits)}:\\
Über diesen Befehl kann eingestellt werden, wie viele Nachkommastellen bei der
Ausgabe von Zahlen gemäß lokaler Notation ausgegeben werden sollen. Ein
negativer Wert bedeutet, dass alle verfügbaren Nachkommastellen ausgegeben.
(Bei Verwendung der System-Notation werden stets alle verfügbaren Nachkommastellen ausgegeben.)

\item
\cmd{void print(final Object obj)}:\\
Gibt den übergebenen Ausdruck aus.
Zeichenketten werden direkt ausgegeben. Zahlenwerte werden gemäß den per
\cm{setFormat} vorgenommenen Einstellungen formatiert.

\item
\cmd{void println(final Object obj)}:\\
Gibt den übergebenen Ausdruck aus und fügt dabei einen Zeilenumbruch an.
Zeichenketten werden direkt ausgegeben. Zahlenwerte werden gemäß den per
\cm{setFormat} vorgenommenen Einstellungen formatiert.

\item
\cmd{void newLine()}:\\
Gibt einen Zeilenumbruch aus. Diese Funktion ist gleichwertig zu dem Aufruf von
\cm{println("{}"{})}.

\item
\cmd{void tab()}:\\
Gibt einen Tabulator aus.

\item
\cmd{void cancel()}:\\
Setzt den Abbruch-Status. (Nach einem Abbruch werden Dateiausgaben nicht mehr ausgeführt.)

\item
\cmd{printlnDDE(final String workbook, final String table, final String cell,\\final Object obj)}:\\
Dieser  Befehl steht nur zur Verfügung, wenn DDE verfügbar ist, d.h. unter Windows.
Er gibt den übergebenen Ausdruck per DDE in der angegebenen Tabelle in Excel aus.
Zahlenwerte werden dabei gemäß den per \cm{setFormat} vorgenommenen Einstellungen formatiert.

\end{itemize}



\chapter{\texttt{FileOutputInterface} abrufbar über \texttt{sim.getFileOutput()}}

Das \texttt{FileOutputInterface}-Interface stellt alle Funktionen, die auch das
\texttt{OutputInterface}-Interface anbietet, zur Verfügung und ist nur während
der Parameterreihen-Skript-Ausführung verfügbar. Im Unterschied
zum \texttt{OutputInterface}-Interface werden die Ausgaben nicht auf die Standardausgabe
geleitet, sondern es muss zunächst per \texttt{sim.getFileOutput().setFile("{}Dateiname")}
eine Ausgabedatei definiert werden. Alle Ausgaben werden dann an diese
Datei angehängt.



\chapter{\texttt{ModelInterface} abrufbar über \texttt{sim.getModel()}}

Das \texttt{ModelInterface}-Interface steht nur während der Parameterreihen-Skript-Ausführung
zur verfügbar und bietet Funktionen, um auf Modelleigenschaften zuzugreifen und
Simulationen zu initiieren.

\begin{itemize}

\item
\cmd{void reset()}:\\
Stellt das Modell auf den Ausgangszustand zurück.

\item
\cmd{void  run()}:\\
Simuliert das aktuelle Modell.
Auf die Ergebnisse kann im Folgenden über das\\
<tt>StatisticsInterface</tt>-Interface zugegriffen werden.

\item
\cmd{boolean setDistributionParameter(final String xmlName, final int number,\\final double value)}:\\
Stellt einen Verteilungsparameter \cm{number} (zwischen 1 und 4) der über \cm{xmlName}
angegebenen Wahrscheinlichkeitsverteilung ein.

\item
\cmd{boolean setMean(final String xmlName, final double value)}:\\
Stellt den Mittelwert der über \cm{xmlName} angegebenen Wahrscheinlichkeitsverteilung auf den angegebenen Wert.

\item
\cmd{boolean setSD(final String xmlName, final double value)}:\\
Stellt die Standardabweichung der über \cm{xmlName} angegebenen Wahrscheinlichkeitsverteilung auf den angegebenen Wert.

\item
\cmd{boolean setString(final String xmlName, final String value)}:\\
Schreibt an die über \cm{xmlName} angegebene Stelle im Modell die angegebene Zeichenkette.

\item
\cmd{boolean setValue(final String xmlName, final double value)}:\\
Schreibt an die über \cm{xmlName} angegebene Stelle im Modell den angegebenen Wert.

\item
\cmd{String xml(final String xmlName)}:\\
Liefert den über \cm{xmlName} erreichbaren Wert.\\
Diese Funktion ist das Äquivalent zu \cm{sim.getStatistics().xml(xmlName)} für Modelldaten.

\item
\cmd{getResourceCount(final String resourceName)}:\\
Liefert die Anzahl an Bedienern in der Ressource mit Namen \cm{resourceName}.
Existiert die Ressource nicht oder ist in ihr die Bedieneranzahl nicht als Zahlenwert
hinterlegt, so liefert die Funktion -1. Ansonsten die Anzahl an Bedienern in der
Ressource.

\item
\cmd{boolean setResourceCount(final String resourceName, final int count)}:\\
Stellt die Anzahl an Bedienern in der Ressource mit Namen \cm{resourceName} ein.

\item
\cmd{String getGlobalVariableInitialValue(final String variableName)}:\\
Liefert den Ausdruck zur Bestimmung des initialen Wertes der globalen Variable
mit Namen\\\cm{variableName}. Existiert die globale Variable nicht, so wird
eine leere Zeichenkette geliefert.

\item
\cmd{boolean setGlobalVariableInitialValue(final String variableName,\\final String expression)}:\\
Stellt den Ausdruck zur Bestimmung des initialen Wertes der globalen Variable
mit Namen\\\cm{variableName} ein. 

\item
\cmd{String getGlobalMapInitialValue(final String variableName)}:\\
Liefert den initialen Wertes der Eintrags \cm{VariablenName} der globalen Zuordnung.
Existiert kein Eintrag mit diesem Namen, so wird \cm{null} geliefert.

\item
\cmd{boolean setGlobalMapInitialValue(final String variableName,\\final String expression)}:\\
Stellt den initialen Wert (vom Typ \cm{Integer}, \cm{Long}, \cm{Double} oder \cm{String}) des Schlüssels \cm{VariablenName}
in der globalen Zuordnung ein. 

\item
\cmd{void cancel()}:\\
Setzt den Abbruch-Status. (Nach einem Abbruch werden keine Simulationen mehr ausgeführt.)

\item
\cmd{int getStationID(final String name)}:\\
Liefert die ID einer Station basierend auf dem Namen der Station.
Existiert keine Station mit dem passenden Namen, so liefert die Funktion -1.

\end{itemize}



\chapter{XML-Auswahlbefehle}

Über die Parameter der Funktionen des \cmd{StatisticsInterface}-Interfaces kann der Inhalt eines XML-Elements oder der Wert eines
Attributes eines XML-Elements ausgelesen werden. Die Selektion eines XML-Elements erfolgt dabei mehrstufig
getrennt durch "\texttt{->}"-Zeichen. Zwischen den "\texttt{->}"-Zeichen stehen jeweils die Namen von XML-Elementen.
Zusätzlich können in eckigen Klammern Namen und Werte von Attributen angegeben werden, nach denen gefiltert werden soll.

Beispiele:

\begin{itemize}

\item
\cmd{sim.getStatistics().xml("{}Modell->ModellName")}:\\
Liefert den Inhalt des Elements \cm{ModellName}, welches ein Unterelement von \cm{Modell} ist.	

\item
\cmd{sim.getStatistics().xml("{}StatistikZwischenankunftszeitenKunden->\\{}Station[Typ=$\backslash$"{}Quelle id=1$\backslash$"]->[Mittelwert]")}:\\
Selektiert das \cm{Station}-Unterelement des \cm{StatistikZwischenankunftszeitenKunden}-Elements, bei
dem das \cm{Typ}-Attribut auf den Wert \cm{Quelle id=1} gesetzt ist. Und liefert dann den Inhalt des Attributs
\cm{Mittelwert}.

\end{itemize}