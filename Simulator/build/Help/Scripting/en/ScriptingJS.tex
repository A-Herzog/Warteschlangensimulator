\part{Javascript commands reference}

Scripts can be used at different points in the simulator.
The script language is \textbf{Javascript} or \textbf{Java}.

In this section the additional \textbf{Javascript} commands which
are available when using Javascript to access the
simulation or statistics data and to output filtered data are
presented.



\chapter{\texttt{Statistics} object}

The \cmd{Statistics} object offers read access to the xml elements which are the
base of the statistics data. The \cmd{Statistics} object is only available after the
simulation while filtering the results while and when running a parameter series script.
The following methods are in this object available:

\section{Definition of the output format}

\begin{itemize}

\item
\cmd{Statistics.setFormat("{}Format")}:\\
This command allows to setup the format that is used in \cmd{Statistics.xml} for outputing numbers
as strings. You can specify whether to use a floating point notation or percent notation or interpreting
the value as a time. As default floating point notation is used.
\begin{itemize}
\item
\cmd{System}:
Using floating point notation for numbers and percent values.
\item
\item
\cmd{Fraction}:
Using floating point notation for numbers ($0.375$ for example).
\item
\cmd{Percent}:
Using percent notation for numbers ($35.7\%$ for example).
\item
\cmd{Time}:
Interpreting numbers as times ($00{:}03{:}25{,}87$ for example).
\item
\cmd{Number}:
Interpreting time values as normal numbers (format defined by \cm{Percent} or \cm{Fraction}).
\end{itemize}
	
\item
\cmd{Statistics.setSeparator("{}Format")}:\\
This command allows to select the separator to be used when printing out distributions of measured values.
\begin{itemize}
\item
\cmd{Semicolon}:
Semicolons as separators
\item
\cmd{Line}:
Line break as separators
\item
\cmd{Tabs}:
Tabulators as separators
\end{itemize}

\end{itemize}

\section{Accessing statistics xml data}

\begin{itemize}
	
\item
\cmd{Statistics.xml("{}Path")}:\\
Loads the xml field which is specified by the parameter and returns the data in the format
defined by \cmd{Statistics.setFormat} and \cmd{Statistics.setSeparator} as a string.

Example:\\
\cm{var name=Statistics.xml("{}Model->ModelName")}
  
\item
\cmd{Statistics.xmlNumber("{}Path")}:\\
Loads the xml field which is specified by the parameter and returns the value as a number.
If the field cannot be interpreted as a number, a string containing an error message will be returned.  
	
\item
\cmd{Statistics.xmlArray("{}Path")}:\\
Loads the xml field which is specified by the parameter, interprets it as a distribution and
returns the values as an array of numbers.
If the field cannot be interpreted as a distribution, a string containing an error message will be returned.

Example:\\
\cm{Statistics.xmlArray("{}StatisticsProcessTimesClients->ClientType[Type=$\backslash$"{}ClientsA$\backslash$"]->\\{}[Distribution]")}

\item
\cmd{Statistics.xmlSum("{}Path")}:\\
Loads the xml field which is specified by the parameter, interprets it as a distribution and
returns the sum of all values as a number.
If the field cannot be interpreted as a distribution, a string containing an error message will be returned.	

Example:\\
\cm{Statistics.xmlSum("{}StatisticsProcessTimesClients->ClientType[Type=$\backslash$"{}ClientsA$\backslash$"]->\\{}[Distribution]")}

\item
\cmd{Statistics.xmlMean("{}Path")}:\\
Loads the xml field which is specified by the parameter, interprets it as a distribution and
returns the mean of values as a number.
If the field cannot be interpreted as a distribution, a string containing an error message will be returned.
	
Example:\\
\cm{Statistics.xmlMean("{}StatisticsProcessTimesClients->ClientType[Type=$\backslash$"{}ClientsA$\backslash$"]->\\{}[Distribution]")}
  
\item
\cmd{Statistics.xmlSD("{}Path")}:\\
Loads the xml field which is specified by the parameter, interprets it as a distribution and
returns the standard deviation of values as a number.
If the field cannot be interpreted as a distribution, a string containing an error message will be returned.

Example:\\
\cm{Statistics.xmlSD("{}StatisticsProcessTimesClients->ClientType[Type=$\backslash$"{}ClientsA$\backslash$"]->\\{}[Distribution]")}

\item
\cmd{Statistics.xmlCV("{}Path")}:\\
Loads the xml field which is specified by the parameter, interprets it as a distribution and
returns the coefficient of variation of values as a number.
If the field cannot be interpreted as a distribution, a string containing an error message will be returned.
	
Example:\\
\cm{Statistics.xmlCV("{}StatisticsProcessTimesClients->ClientType[Type=$\backslash$"{}ClientsA$\backslash$"]->\\{}[Distribution]")}

\item
\cmd{Statistics.xmlMedian("{}Path")}:\\
Loads the xml field which is specified by the parameter, interprets it as a distribution and
returns the median of values as a number.
If the field cannot be interpreted as a distribution, a string containing an error message will be returned.
	
Example:\\
\cm{Statistics.xmlMedian("{}StatisticsProcessTimesClients->ClientType[Type=$\backslash$"{}ClientsA$\backslash$"]->\\{}[Distribution]")}

\item
\cmd{Statistics.xmlMode("{}Path")}:\\
Loads the xml field which is specified by the parameter, interprets it as a distribution and
returns the mode value of values as a number.
If the field cannot be interpreted as a distribution, a string containing an error message will be returned.
	
Example:\\
\cm{Statistics.xmlMode("{}StatisticsProcessTimesClients->ClientType[Type=$\backslash$"{}ClientsA$\backslash$"]->\\{}[Distribution]")}

\item
\cmd{Statistics.translate("{}de")}:\\
Translates the statistics data to English ("{}en") or German ("{}de") so the preferred xml tag names can be
used independent of the language setting under which the statistics file was generated.

\end{itemize}

\section{Saving the statistics data to files}

\begin{itemize}

\item
\cmd{Statistics.save("{}FileName")}:\\
Saves the entry statistics data under the next available file name in the given folder.\\
This function is only available in the Run script panel.

\item
\cmd{Statistics.saveNext("{}FolderName")}:\\
Saves the entry statistics data under the next available file name in the given folder.\\
This function is only available in the Run script panel.

\item
\cmd{Statistics.filter("{}FileName")}:\\
Applies the selected script on the statistics data and returns the results.\\
This function is only available in the Run script panel.

\item
\cmd{Statistics.cancel()}:\\
Sets the cancel status. (When output is canceled to further file output will be performed.)

\end{itemize}

\section{Accessing station data}

\begin{itemize}

\item
\cmd{Statistics.getStationID("{}StationName")}:\\
Gets the ID of a station based on its name.
If there is no station with a matching name, the function will return -1.

\end{itemize}



\chapter{\texttt{System} object}

The \cmd{System} object allows to access some general program functions.
The \cmd{System} object is only available after the simulation while filtering the results
or when running parameter series scripts.
The following methods are available in this object:

\begin{itemize}

\item
\cmd{System.calc("{}Expression")}:\\
Calculates the expression passed as a string by means of the term evaluation function,
which is also used in various other places in the program (see part \ref{part:Rechenbefehle}),
and returns the result as a number. If the expression can not be calculated, an error message is returned as a string.
The term evaluation function allows access to all known probability distributions,
the Erlang C calculator, etc.

\item
\cmd{System.time()}:\\
Returns the current system time as a milliseconds value. This functions can be used to measure
the runtime of a script.

\item
\cmd{System.getInput("http://Adresse",-1)}:\\
Loads a numerical value via the specified address and returns it.
If no value could be loaded, the error value specified in the second parameter is returned.

\item
\cmd{System.execute("program.exe")}:\\
Executes an external command and returns immediately. Returns true, if the program could be started.
Executing external programs by scripts is disabled by default. If can be activated
in the program settings dialog.

\item
\cmd{System.executeAndReturnOutput("program.exe")}:\\
Executes an external command and returns the output.
Executing external programs by scripts is disabled by default. If can be activated
in the program settings dialog.

\item
\cmd{System.executeAndWait("program.exe")}:\\
Executes an external command, waits for completion and returns the return code of the program.
In case of an error -1 will be returned.
Executing external programs by scripts is disabled by default. If can be activated
in the program settings dialog.

\end{itemize}



\chapter{\texttt{Simulation} object}

The \cmd{Simulation} object allows to access the model data while simulation is running.
It is not available for filtering the results after simulation has terminated.
The following methods are available in this object:

\section{Base functions}

\begin{itemize}

\item
\cmd{Simulation.time()}:\\
Gets the current time in the simulation as a seconds numerical value.

\item
\cmd{Simulation.calc("{}Expression")}:\\
Calculates the expression passed as a string by means of the term evaluation function,
which is also used in various other places in the program (see part \ref{part:Rechenbefehle}),
and returns the result as a number. If the expression can not be calculated, an error message is returned as a string.
The term evaluation function allows access to all known probability distributions,
the Erlang C calculator, etc.

\item
\cmd{Simulation.getInput("http://Adresse",-1)}:\\
Loads a numerical value via the specified address and returns it.
If no value could be loaded, the error value specified in the second parameter is returned.

\item
\cmd{Simulation.execute("program.exe")}:\\
Executes an external command and returns immediately. Returns true, if the program could be started.
Executing external programs by scripts is disabled by default. If can be activated
in the program settings dialog.

\item
\cmd{Simulation.executeAndReturnOutput("program.exe")}:\\
Executes an external command and returns the output.
Executing external programs by scripts is disabled by default. If can be activated
in the program settings dialog.

\item
\cmd{Simulation.executeAndWait("program.exe")}:\\
Executes an external command, waits for completion and returns the return code of the program.
In case of an error -1 will be returned.
Executing external programs by scripts is disabled by default. If can be activated
in the program settings dialog.

\item
\cmd{Simulation.isWarmUp()}:\\
Gets true of false depending if the simulation is in the warm-up phase.

\item
\cmd{Simulation.getMapLocal()}:\\
Returns a station-local mapping into which values can be written and from which values can be read.
The values stored here are retained beyond the execution of the current script.  

\item
\cmd{Simulation.getMapGlobal()}:\\
Returns a model wide mapping into which values can be written and from which values can be read.
The values stored here are retained beyond the execution of the current script.

\item
\cmd{Simulation.pauseAnimation()}:\\
Switches the animation to single step mode. If the animation is already executed in
single step mode or if the model is executed as a simulation, this command has no effect.

\item
\cmd{Simulation.terminateSimulation(message)}:\\
Beendet die Simulation. Wird als Nachricht \cm{null} übergeben, so wird die Simulation normal
beendet. Im Falle einer Nachricht wird die Simulation mit der entsprechenden Fehlermeldung abgebrochen.

\end{itemize}

\section{Accessing client-specific data}

\begin{itemize}

\item
\cmd{Simulation.clientTypeName()}:\\
Gets the name of the type of the client who has triggered the processing of the script.\\
(If the event was triggered by a client.)

\item
\cmd{Simulation.isWarmUpClient()}:\\
Gets true of false depending if the current client was generated during the warm-up phase and
therefore will not be recorded in the statistics.\\
(If the event was triggered by a client.)

\item
\cmd{Simulation.isClientInStatistics()}:\\
Gets true of false depending if the current client is to be recorded in the statistics.
This value is independent of the warm-up phase. A client will only be recorded if he was
generated after the warm-up phase and this value is true.\\
(If the event was triggered by a client.)

\item
\cmd{Simulation.setClientInStatistics(inStatistics)}:\\
Sets if a client is to be recorded in the statistics.
This value is independent of the warm-up phase. A client will only be recorded if he was
generated after the warm-up phase and this value is not set to false.\\
(If the event was triggered by a client.)

\item
\cmd{Simulation.clientNumber()}:\\
Get the 1-based consecutive number of the current client.
When using multiple simulation threads this number is thread local.\\
(If the event was triggered by a client.)

\item
\cmd{Simulation.clientWaitingSeconds()}:\\
Gets the current waiting time of the client who has triggered the processing of the script as a seconds numerical value.\\
(If the event was triggered by a client.)

\item
\cmd{Simulation.clientWaitingTime()}:\\
Gets the current waiting time of the client who has triggered the processing of the script as a formated time value as a string.\\
(If the event was triggered by a client.)

\item
\cmd{Simulation.clientWaitingSecondsSet(seconds)}:\\
Sets the current waiting time of the client who has triggered the processing of the script.\\
(If the event was triggered by a client.)

\item
\cmd{Simulation.clientTransferSeconds()}:\\
Gets the current transfer time of the client who has triggered the processing of the script as a seconds numerical value.\\
(If the event was triggered by a client.)

\item
\cmd{Simulation.clientTransferTime()}:\\
Gets the current transfer time of the client who has triggered the processing of the script as a formated time value as a string.\\
(If the event was triggered by a client.)

\item
\cmd{Simulation.clientTransferSecondsSet(seconds)}:\\
Sets the current transder time of the client who has triggered the processing of the script.\\
(If the event was triggered by a client.)

\item
\cmd{Simulation.clientProcessSeconds()}:\\
Gets the current processing time of the client who has triggered the processing of the script as a seconds numerical value.\\
(If the event was triggered by a client.)

\item
\cmd{Simulation.clientProcessTime()}:\\
Gets the current processing time of the client who has triggered the processing of the script as a formated time value as a string.\\
(If the event was triggered by a client.)

\item
\cmd{Simulation.clientProcessSecondsSet(seconds)}:\\
Sets the current processing time of the client who has triggered the processing of the script.\\
(If the event was triggered by a client.)

\item
\cmd{Simulation.clientResidenceSeconds()}:\\
Gets the current residence time of the client who has triggered the processing of the script as a seconds numerical value.\\
(If the event was triggered by a client.)

\item
\cmd{Simulation.clientResidenceTime()}:\\
Gets the current residence time of the client who has triggered the processing of the script as a formated time value as a string.\\
(If the event was triggered by a client.)

\item
\cmd{Simulation.clientResidenceSecondsSet(seconds)}:\\
Sets the current residence time of the client who has triggered the processing of the script.\\
(If the event was triggered by a client.)

\item
\cmd{Simulation.getClientValue(index)}:\\
Gets for the current client the numerical value which is stored by the index \cm{index}.\\
(If the event was triggered by a client.)
  
\item
\cmd{Simulation.setClientValue(index,value)}:\\
Sets for the current client the numerical \cm{value} for the index \cm{index}.\\
(If the event was triggered by a client.)

\item
\cmd{Simulation.getClientText("{}key")}:\\
Gets for the current client the string which is stored by the key \cm{key}.\\
(If the event was triggered by a client.)
  
\item
\cmd{Simulation.setClientText("{}key","{}value")}:\\
Sets for the current client string \cm{value} for the key \cm{key}.\\
(If the event was triggered by a client.)

\item
\cmd{Simulation.getAllClientValues()}:\
Return all numerical values stored for the current client.
  
\item
\cmd{Simulation.getAllTexts()}:\
Return all text values stored for the current client.

\end{itemize}

\section{Temporary batches}

If the current client is a temporary batch, the properties of the inner clients
it contains can be accessed in read-only mode:

\begin{itemize}
\item
\cmd{Simulation.batchSize()}:\\
Returns the number of clients that are in the temporary batch.  
If the current client is not a temporary batch, the function returns 0.

\item
\cmd{Simulation.getBatchTypeName(batchIndex)}:\\
Returns the name of one of the clients in the current batch.
The passed index is 0-based and must be in the range from 0 to \cm{batchSize()-1}.

\item
\cmd{Simulation.getBatchWaitingSeconds(batchIndex)}:\\
Returns the previous waiting time of one of the clients in the current batch in seconds as a numerical value.
The passed index is 0-based and must be in the range from 0 to \cm{batchSize()-1}.


\item
md{Simulation.getBatchWaitingTime(batchIndex)}:\\
Returns the previous waiting time of one of the clients in the current batch in formatted form as a string.
The passed index is 0-based and must be in the range from 0 to \cm{batchSize()-1}.


tem
md{Simulation.getBatchTransferSeconds(batchIndex)}:\\
Returns the previous transfer time of one of the clients in the current batch in seconds as a numerical value.
The passed index is 0-based and must be in the range from 0 to \cm{batchSize()-1}.


\item
\cmd{Simulation.getBatchTransferTime(batchIndex)}:\\
Returns the previous transfer time of one of the clients in the current batch in formatted form as a string.
The passed index is 0-based and must be in the range from 0 to \cm{batchSize()-1}.

\item
\cmd{Simulation.getBatchProcessSeconds(batchIndex)}:\\
Returns the previous processing time of one of the clients in the current batch in seconds as a numerical value.
The passed index is 0-based and must be in the range from 0 to \cm{batchSize()-1}.

\item
\cmd{Simulation.getBatchProcessTime(batchIndex)}:\\
Returns the previous processing time of one of the clients in the current batch in formatted form as a string.
The passed index is 0-based and must be in the range from 0 to \cm{batchSize()-1}.
	 
\item
\cmd{Simulation.getBatchResidenceSeconds(batchIndex)}:\\
Returns the previous residence time of one of the clients in the current batch in seconds as a numerical value.
The passed index is 0-based and must be in the range from 0 to \cm{batchSize()-1}.

\item
\cmd{Simulation.getBatchResidenceTime(batchIndex)}:\\
Returns the previous residence time of one of the clients in the current batch in formatted form as a string.
The passed index is 0-based and must be in the range from 0 to \cm{batchSize()-1}.

\item
\cmd{Simulation.getBatchValue(batchIndex, index)}:\\
Returns a stored numerical value for one of the clients in the current batch.
The passed batch index is 0-based and must be in the range from 0 to \cm{batchSize()-1}.

\item
\cmd{Simulation.getBatchText(batchIndex, key)}:\\
Returns a stored text value for one of the clients in the current batch.
The passed batch index is 0-based and must be in the range from 0 to \cm{batchSize()-1}.

\end{itemize}

\section{Accessing parameters of the simulation model}

\begin{itemize}

\item
\cmd{Simulation.set("{}Name",Value)}:\\
Sets the simulation variable which is specified as the first parameter to the value specified as the second parameter.  
\cm{Value} can be a number or a string. The case of a number the value will be assigned directly.
Strings will be interpreted like \cm{Simulation.calc} does and the result will be assigned to the variable. \cm{Name}
can either be the name of an already defined simulation variable or of a client data field in the form
\cm{ClientData(index)} with $index\ge0$.

\item
\cmd{Simulation.setValue(id,Value)}:\\
Sets the value at the "Analog value" or "Tank" element with the specified \cm{id}.
  
\item
\cmd{Simulation.setRate(id,Value)}:\\
Sets the change rate (per second) at the "Analog value" element with the specified \cm{id}.
  
\item
\cmd{Simulation.setValveMaxFlow(id,ValveNr,Vale)}:\\
Sets the maximum flow (per second) at the specified valve (1 based) of the "Tank" element
with the specified \cm{id}. The maximum flow has to be a non-negative number.  

\item
\cmd{Simulation.getWIP(id)}:\\
Gets the current number of clients at the station with the specified id.
  
\item
\cmd{Simulation.getNQ(id)}:\\
Gets the current number of clients in the queue at the station with the specified id.

\item
\cmd{Simulation.getNS(id)}:\\
Gets the current number of clients in service process at the station with the specified id.

\item
\cmd{Simulation.getWIP()}:\\
Gets the current number of clients in the system.
  
\item
\cmd{Simulation.getNQ()}:\\
Gets the current number of waiting clients in the system.

\item
\cmd{Simulation.getNS()}:\\
Gets the current number of clients in service process in the system.

\end{itemize}

\section{Accessing the current input value}

\begin{itemize}

\item
\cmd{Simulation.getInput()}:\\
If the Javascript code is being executed from a Input (Script) element,
this function returns the current input value. Otherwise it will just return  0. 

\end{itemize}

\section{Number of operators in a resource}

\begin{itemize}
	
\item
\cmd{Simulation.getAllResourceCount()}:\\
Returns the current number of operators in all resources together.
  
\item
\cmd{Simulation.getResourceCount(id)}:\\
Returns the current number of operators in the resource with the specified id.  
  
\item
\cmd{Simulation.setResourceCount(id,count)}:\\
Sets the number of operators in the resource with the specified id.
To be able to set the number of operators in a resource at runtime,
the initial number of operators in the resource has to be a fixed number
(not infinite many and not by a time table). Additionally no down times
are allowed for this resource.
The function returns \cm{true} if the number of operators has successfully 
been changed. If the new number of operators is less than the previous number,
the new number may is not instantly visible in the simulation system because
removed but working operators will finish their current tasks before they are
actually removed.

\item
\cmd{Simulation.getAllResourceDown()}:\\
Returns the current number of operators in down time in all resources together.

\item
\cmd{Simulation.getResourceDown(id)}:\\
Returns the current number of operators in down time in the resource with the specified id.  
	
\end{itemize}

\section{Fire signal}

\begin{itemize}

\item
\cmd{Simulation.signal(name)}:\\
Fires the signal with the given name.

\end{itemize}

\section{Trigger script execution}

\begin{itemize}

\item
\cmd{Simulation.triggerScriptExecution(stationId,time)}:\\
Triggers the execution of the script at a script or a script hold station at a given time.

\end{itemize}

\section{Output message in logging}

\begin{itemize}

\item
\cmd{Simulation.log(message)}:\\
Outputs the passed message to the logging system (if logging is enabled).

\end{itemize}

\section{Release clients at delay stations}

If a list of clients at a delay station is recorded, this list can be queried using the
following function and individual clients can be selectively released before their
specified delay time has expired.

\begin{itemize}

\item
\cmd{getDelayStationData(id)}:\\
Returns an object which offers the methods described in \textbf{Accessing client-specific data}
for accessing the list of clients at the delay station \cm{id}. If the id is invalid, \cm{null}
will be returned.

\end{itemize}

\section{Clients in the queue of a process station}

\begin{itemize}

  \item
  \cmd{getProcessStationQueueData(id)}:\
  Returns an object which offers the methods described in \textbf{Accessing client-specific data}
  for accessing the list of clients waiting at the process station \cm{id}. If the id is invalid, \cm{null}
  will be returned. Only the waiting clients can be accessed, not the clients which are already in service process.
  Also clients cannot be released via the \cm{release} method here.
  
\end{itemize}



\chapter{\texttt{Clients} object}

The \cmd{Clients} object is only available within a hold by script condition element
and allows to access the waiting clients and to release them.

\begin{itemize}

\item
\cmd{Clients.count()}:\\
Returns the current number of waiting clients. For the other
methods a single client can be accessed via the index parameter
(valued from 0 to \cm{count()}-1).

\item
\cmd{Clients.clientTypeName(index)}:\\
Gets the name of the type of the client.

\item
\cmd{Clients.clientWaitingSeconds(index)}:\\
Gets the current waiting time of the client as a seconds numerical value.

\item
\cmd{Clients.clientWaitingTime(index)}:\\
Gets the current waiting time of the client as a formated time value as a string.

\item
\cmd{Clients.clientWaitingSecondsSet(index,value)}:\\
Sets the waiting time of the client as a seconds numerical value.

\item
\cmd{Clients.clientTransferSeconds(index)}:\\
Gets the current transfer time of the client as a seconds numerical value.

\item
\cmd{Clients.clientTransferTime(index)}:\\
Gets the current transfer time of the client as a formated time value as a string.

\item
\cmd{Clients.clientTransferSecondsSet(index,value)}:\\
Sets the transfer time of the client as a seconds numerical value.

\item
\cmd{Clients.clientProcessSeconds(index)}:\\
Gets the current processing time of the client as a seconds numerical value.

\item
\cmd{Clients.clientProcessTime(index)}:\\
Gets the current processing time of the client as a formated time value as a string.

\item
\cmd{Clients.clientProcessSecondsSet(index,value)}:\\
Sets the processing time of the client as a seconds numerical value.

\item
\cmd{Clients.clientResidenceSeconds(index)}:\\
Gets the current residence time of the client as a seconds numerical value.

\item
\cmd{Clients.clientResidenceTime(index)}:\\
Gets the current residence time of the client as a formated time value as a string.

\item
\cmd{Clients.clientResidenceSecondsSet(index,value)}:\\
Sets the residence time of the client as a seconds numerical value.

\item
\cmd{Clients.clientData(index,data)}:\\
Returns the data element which index is specified via the second parameter of the selected client.

\item
\cmd{Clients.clientData(index,data,value)}:\\
Set the numerical value specified by the third parameter for the data element which index is specified via the second parameter of the selected client.

\item
\cmd{Clients.clientTextData(index,key)}:\\
Returns the data element which key is specified via the second parameter of the selected client.

\item
\cmd{Clients.clientTextData(index,key,value)}:\\
Set the text value specified by the third parameter for the key which is specified via the second parameter of the selected client.

\item
\cmd{Clients.release(index)}:\\
Causes the forwarding of the specified client.
	
\end{itemize}



\chapter{\texttt{Output} object}

The \cmd{Output} object provides functions for output of filtered results:

\begin{itemize}

\item
\cmd{Output.setFormat("{}Format")}:\\
This command allows to setup the format that is used in \cmd{Output.print} and \cmd{Output.println}
for outputing numbers as strings. You can specify whether to use a floating point notation or percent notation or interpreting
the value as a time. As default floating point notation is used.

\begin{itemize}
\item
\cmd{Fraction}:\\
Using floating point notation for numbers ($0.375$ for example).
\item
\cmd{Percent}:\\
Using percent notation for numbers ($35.7\%$ for example).
\item
\cmd{Number}:\\
Interpreting numbers as normal number values (decimal or percent).
\item
\cmd{Time}:\\
Interpreting numbers as time values.
\end{itemize}

\item
\cmd{Output.setSeparator("{}Format")}:\\
This command allows to select the separator to be used when printing out arrays.
\begin{itemize}
\item
\cmd{Semicolon}:\\
Semicolons as separators.
\item
\cmd{Line}:\\
Line break as separators.
\item
\cmd{Tabs}:\\
Tabulators as separators.
\end{itemize}

\item
\cmd{Output.setDigits(digits)}:\\
This command allows to define the number of digits to be displayed when
printing a number in local notation. A negative value means that all
available digits are being printed.
(If the system notation is used, always all available digits are being printed.)

\item
\cmd{Output.print("{}Expression")}:\\
Outputs the passed expression.
Strings will be written directly. Numbers are formated according to the format
defined via \cm{Output.setFormat}.

\item
\cmd{Output.println("{}Expression")}:\\
Outputs the passed expression and adds a line break after the expression.
Strings will be written directly. Numbers are formated according to the format
defined via \cm{Output.setFormat}.

\item
\cmd{Output.newLine()}:\\
Outputs a line break. This functions is equivalent to calling\\
\cmd{Output.println("{}"{})}.

\item
\cmd{Output.tab()}:\\
Outputs a tabulator.

\item
\cmd{Output.cancel()}:\\
Sets the cancel status. (When output is canceled to further file output will be performed.)

\item
\cmd{Output.printlnDDE("{}Workbook","{}Table","{}Cell","{}Expression")}:\\
This command is only available if DDE is available, i.e. under Windows.
It outputs the passed expression via DDE in the specified table in Excel.
Numbers are formated according to the format defined via \cm{Output.setFormat}.

\end{itemize}



\chapter{\texttt{FileOutput} object}

The \cmd{FileOutput} object offers all function the \cmd{Output} has
but is only available when running a parameter series script. In opposite
to the \cmd{Output} object the output of the \cmd{FileOutput} object
is not written to the default output but is appended to a file which has
to be specified by \cmd{FileOutput.setFile("{}Filename")} before.



\chapter{\texttt{Model} object}

Th \cmd{Model} object is only available during parameter series
script execution and offers functions for accessing the model
properties and for starting simulations.

\begin{itemize}

\item
\cmd{Model.reset()}:\\
Resets the model to the initial state.

\item
\cmd{Model.run()}:\\
Simulates the current model.
The results can be accessed by the \cmd{Statistics} object after the simulation.

\item
\cmd{Model.setDistributionParameter("{}Path",Index,Value)}:\\
Sets the distribution parameter \cm{Index} (from 1 to 4) of the distribution referred
to by \cm{Path}.

\item
\cmd{Model.setMean("{}Path",Value)}:\\
Sets the mean of the distribution referred to by \cm{Path} to the specified value.

\item
\cmd{Model.setSD("{}Path",Value)}:\\
Sets the standard deviation of the distribution referred to by \cm{Path} to the specified value.  

\item
\cmd{Model.setString("{}Path","{}Text)"}:\\
Writes the string \cm{Text} to the location referred to by \cm{Path}.

\item
\cmd{Model.setValue("{}Path",Value)}:\\
Writes the number \cm{Value} to the location referred to by \cm{Path}.

\item
\cmd{Model.xml("{}Path")}:\\
Loads the xml field which is specified by the parameter and returns the data 
as String. This function is the equivalent of \cmd{Statistics.xml("{}Path")} for models.

\item
\cmd{Model.getResourceCount("{}ResourceName")}:\\
Gets the number of operations in the resource with name \cm{ResourceName}.
If the resource does not exist or is not defined by a fixed number of operators
the function will return -1.

\item
\cmd{Model.setResourceCount("{}ResourceName",Count)}:\\
Sets the number of operations in the resource with name \cm{ResourceName}.

\item
\cmd{Model.getGlobalVariableInitialValue("{}VariableName")}:\\
Gets the expression by which the initial value of the global variable with
name \cm{VariableName} is calculated. If the is no global variable with
this name the function will return an empty string.

\item
\cmd{Model.setGlobalVariableInitialValue("{}VariableName","{}Expression")}:\\
Sets the expression by which the initial value of the global variable with
name \cm{VariableName} is calculated.

\item
\cmd{Model.getGlobalMapInitialValue("{}VariableName")}:\\
Gets the initial value of the entry \cm{VariablenName} of the global map.
If there is no entry with this name, \cm{null} will be returned.

\item
\cmd{Model.setGlobalMapInitialValue("{}VariableName","{}Expression")}:\\
Sets the initial value (of type \cm{Integer}, \cm{Long}, \cm{Double} oder \cm{String}) of the key \cm{VariablenName}
in the global map.

\item
\cmd{Model.cancel()}:\\
Sets the cancel status. (When processing is canceled to further simulations will be performed.)

\end{itemize}

\section{Accessing station data}

\begin{itemize}

\item
\cmd{Model.getStationID("{}StationName")}:\\
Gets the ID of a station based on its name.
If there is no station with a matching name, the function will return -1.

\end{itemize}

\section{Retrieve the associated statistics file}

\begin{itemize}

\item
\cmd{Statistics.getStatisticsFile()}:\\
Returns the full path and file name of the statistics file from which the data was loaded.
If the statistic data was not loaded from a file, an empty string is returned.

\item
\cmd{Statistics.getStatisticsFileName()}:\\
Returns the file name of the statistics file from which the data was loaded.
If the statistic data was not loaded from a file, an empty string is returned.

\end{itemize}



\chapter{XML selection commands}

By the parameters of the functions of the \cmd{Statistics} object the content of the
value or of an attribute of an XML element can be read.
The selection of an XML element is done multistaged step by step divided by
"\texttt{->}" characters. Between the "\texttt{->}" characters the names of the individual XML nodes are
noted. In addition in square brackets names and values of attributes can be specified to filter by whom.

Examples:

\begin{itemize}

\item
\cmd{Statistics.xml("{}Model->ModellName")}:\\
Shows the content of the element \cm{ModelName}, which is a child element of \cm{Model}.

\item
\cmd{Statistics.xml("{}StatisticsInterArrivalTimesClients->\\{}Station[Type=$\backslash$"{}Source id=1$\backslash$"]->[Mean]")}:\\
Selects the \cm{Station} sub element of the \cm{StatisticsInterArrivalTimesClients} element, for
which the \cm{Type} attribute is set to \cm{Source id=1}. And returns the value of the
\cm{Mean} attribute.

\end{itemize}