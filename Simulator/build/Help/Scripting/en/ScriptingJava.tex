\part{Java commands reference}

Scripts can be used at different points in the simulator.
The script language is \textbf{Javascript} or \textbf{Java}.

In this section the additional \textbf{Java} commands which
are available when using Java to access the
simulation or statistics data and to output filtered data are
presented.

The \textbf{Java} code has to be embedded in a
\begin{verbatim}
void function(SimulationInterface sim) {

}
\end{verbatim}
method. In addition to the standard language commands you can access
the simulation or statistics data depending on the context in which the script is executed
via the \texttt{SimulationInterface} interface which is given as a parameter.
The \texttt{SimulationInterface} has some methods which allow to get sub-interfaces
which offer these data:



\chapter{\texttt{StatisticsInterface} accessible via \texttt{sim.getStatistics()}}

The \texttt{sim.getStatistics()} methods returns a {StatisticsInterface} interface
which offers read access to the xml elements which are the base of the statistics data.
The \texttt{StatisticsInterface} interface is only available after the
simulation while filtering the results while and when running a parameter series scripts.
The following methods are in this object available:

\section{Definition of the output format}

\begin{itemize}

\item
\cmd{void setFormat(final String format)}:\\
This command allows to setup the format that is used in \cmd{Statistics.xml} for outputing numbers
as strings. You can specify whether to use a floating point notation or percent notation or interpreting
the value as a time. As default floating point notation is used.
\begin{itemize}
\item
\cmd{System}:
Using floating point notation for numbers and percent values.
\item
\item
\cmd{Fraction}:
Using floating point notation for numbers ($0.375$ for example).
\item
\cmd{Percent}:
Using percent notation for numbers ($35.7\%$ for example).
\item
\cmd{Time}:
Interpreting numbers as times ($00{:}03{:}25{,}87$ for example).
\item
\cmd{Number}:
Interpreting time values as normal numbers (format defined by \cm{Percent} or \cm{Fraction}).
\end{itemize}
	
\item
\cmd{void setSeparator(final String separator)}:\\
This command allows to select the separator to be used when printing out distributions of measured values.
\begin{itemize}
\item
\cmd{Semicolon}:
Semicolons as separators
\item
\cmd{Line}:
Line break as separators
\item
\cmd{Tabs}:
Tabulators as separators
\end{itemize}

\end{itemize}

\section{Accessing statistics xml data}

\begin{itemize}

\item
\cmd{String xml(final String path)}:\\
Loads the xml field which is specified by the parameter and returns the data in the format
defined by \cm{sim.getStatistics().setFormat} and \cm{sim.getStatistics().setSeparator} as a string.

Example: \cm{String name=sim.getStatistics().xml("{}Model->ModelName")}

\item
\cmd{Object xmlNumber(final String path)}:\\
Loads the xml field which is specified by the parameter and returns the value as a \cm{Double} number.
If the field cannot be interpreted as a number, a string containing an error message will be returned.  

\item
\cmd{Object xmlArray(final String path)}:\\
Loads the xml field which is specified by the parameter, interprets it as a distribution and
returns the values as an array of numbers (\cm{double[]}).
If the field cannot be interpreted as a distribution, a string containing an error message will be returned.

Example:\\
\cm{sim.getStatistics().xmlArray("{}StatisticsProcessTimesClients->\\{}ClientType[Type=$\backslash$"{}ClientsA$\backslash$"]->[Distribution]")}

\item
\cmd{Object xmlSum(final String path)}:\\
Loads the xml field which is specified by the parameter, interprets it as a distribution and
returns the sum of all values as a \cm{Double} number.
If the field cannot be interpreted as a distribution, a string containing an error message will be returned.

Example:\\
\cm{sim.getStatistics().xmlSum("{}StatisticsProcessTimesClients->\\{}ClientType[Type=$\backslash$"{}ClientsA$\backslash$"]->[Distribution]")}

\item
\cmd{Object xmlMean(final String path)}:\\
Loads the xml field which is specified by the parameter, interprets it as a distribution and
returns the mean of values as a \cm{Double} number.
If the field cannot be interpreted as a distribution, a string containing an error message will be returned.

Example:\\
\cm{sim.getStatistics().xmlMean("{}StatisticsProcessTimesClients->\\{}ClientType[Type=$\backslash$"{}ClientsA$\backslash$"]->[Distribution]")}

\item
\cmd{Object xmlSD(final String path)}:\\
Loads the xml field which is specified by the parameter, interprets it as a distribution and
returns the standard deviation of values as a \cm{Double} number.
If the field cannot be interpreted as a distribution, a string containing an error message will be returned.

Example:
\cm{sim.getStatistics().xmlSD("{}StatisticsProcessTimesClients->\\{}ClientType[Type=$\backslash$"{}ClientsA$\backslash$"]->[Distribution]")}

\item
\cmd{Object xmlCV(final String path)}:\\
Loads the xml field which is specified by the parameter, interprets it as a distribution and
returns the coefficient of variation of values as a \cm{Double} number.
If the field cannot be interpreted as a distribution, a string containing an error message will be returned.

Example:\\
\cm{sim.getStatistics().xmlCV("{}StatisticsProcessTimesClients->\\{}ClientType[Type=$\backslash$"{}ClientsA$\backslash$"]->[Distribution]")}

\item
\cmd{boolean translate(final String language)}:\\
Translates the statistics data to English ("{}en") or German ("{}de") so the preferred xml tag names can be
used independent of the language setting under which the statistics file was generated.

\end{itemize}

\section{Saving the statistics data to files}

\begin{itemize}

\item
\cmd{boolean save(final String fileName)}:\\
Saves the entry statistics data under the next available file name in the given folder.\\
This function is only available in the Run script panel.

\item
\cmd{boolean saveNext(final String folderName)}:\\
Saves the entry statistics data under the next available file name in the given folder.\\
This function is only available in the Run script panel.

\item
\cmd{String filter(final String fileName)}:\\
Applies the selected script on the statistics data and returns the results.\\
This function is only available in the Run script panel.

\item
\cmd{void cancel()}:\\
Sets the cancel status. (When output is canceled to further file output will be performed.)

\end{itemize}

\section{Accessing station data}

\begin{itemize}

\item
\cmd{int getStationID(final String name)}:\\
Gets the ID of a station based on its name.
If there is no station with a matching name, the function will return -1.

\end{itemize}

\section{Retrieve the associated statistics file}

\begin{itemize}

\item
\cmd{String getStatisticsFile()}:\\
Returns the full path and file name of the statistics file from which the data was loaded.
If the statistic data was not loaded from a file, an empty string is returned.

\item
\cmd{String getStatisticsFileName()}:\\
Returns the file name of the statistics file from which the data was loaded.
If the statistic data was not loaded from a file, an empty string is returned.

\end{itemize}



\chapter{\texttt{RuntimeInterface} accessible via \texttt{sim.getRuntime()}}

The \texttt{RuntimeInterface} interface allows to access some general program functions.\\
The \texttt{RuntimeInterface} is always available. The following methods are available in this interface:

\begin{itemize}

\item
\cmd{Object calc(final String expression)}:\\
Calculates the expression passed as a string by means of the term evaluation function,
which is also used in various other places in the program (see part \ref{part:Rechenbefehle}), and returns the result as a \cm{Double} number.
If the expression can not be calculated, an error message is returned as a string.
The term evaluation function allows access to all known probability distributions,
the Erlang C calculator, etc.

\item
\cmd{long getTime()}:\\
Returns the current system time as a milliseconds value. This functions can be used to measure
the runtime of a script.

\item
\cmd{double getInput(final String url, final double errorValue)}:\\
Loads a numerical value via the specified address and returns it.
If no value could be loaded, the error value specified in the second parameter is returned.

\item
\cmd{boolean execute(final String commandLine)}:\\
Executes an external command and returns immediately. Returns true, if the program could be started.
Executing external programs by scripts is disabled by default. If can be activated
in the program settings dialog.

\item
\cmd{String executeAndReturnOutput(final String commandLine)}:\\
Executes an external command and returns the output.
Executing external programs by scripts is disabled by default. If can be activated
in the program settings dialog.

\item
\cmd{int executeAndWait(final String commandLine)}:\\
Executes an external command, waits for completion and returns the return code of the program.
In case of an error -1 will be returned.
Executing external programs by scripts is disabled by default. If can be activated
in the program settings dialog.

\end{itemize}



\chapter{\texttt{SystemInterface} accessible via \texttt{sim.getSystem()}}

The \texttt{SystemInterface} interface allows to access the model data while simulation is running.
It is not available for filtering the results after simulation has terminated.
The following methods are available in this interface:

\section{Base functions}

\begin{itemize}

\item
\cmd{double getTime()}:\\
Gets the current time in the simulation as a seconds numerical value.
  
\item
\cmd{Object calc(final String expression)}:\\
Calculates the expression passed as a string by means of the term evaluation function,
which is also used in various other places in the program (see part \ref{part:Rechenbefehle}), and returns the result as a \cm{Double} number.
If the expression can not be calculated, an error message is returned as a string.
The term evaluation function allows access to all known probability distributions,
the Erlang C calculator, etc.
  
\item
\cmd{boolean isWarmUp()}:\\
Gets true of false depending if the simulation is in the warm-up phase.
	
\end{itemize}

\section{Accessing parameters of the simulation model}

\begin{itemize}

\item
\cmd{void set(final String varName, final Object varValue)}:\\
Sets the simulation variable which is specified as the first parameter to the value specified as the second parameter.  
\cm{varValue} can be a number or a string. The case of a number the value will be assigned directly.
Strings will be interpreted like \cm{calc(final String expression)} does and the result will be assigned to the variable. \cm{varName}
can either be the name of an already defined simulation variable or of a client data field in the form
\cm{ClientData(index)} with $index\ge0$. 
  
\item
\cmd{void setAnalogValue(final Object elementID, final Object value)}:\\
Sets the value at the "Analog value" or "Tank" element with the specified id.
  
\item
\cmd{void setAnalogRate(final Object elementID, final Object value)}:\\
Sets the change rate (per second) at the "Analog value" element with the specified id.
  
\item
\cmd{void setAnalogValveMaxFlow(final Object elementID, final Object valveNr,\\final Object value)}:\\
Sets the maximum flow (per second) at the specified valve (1 based) of the "Tank" element
with the specified id. The maximum flow has to be a non-negative number.  
  
\item
\cmd{int getWIP(final int id)}:\\
Gets the current number of clients at the station with the specified id.
  
\item
\cmd{int getNQ(final int id)}:\\
Gets the current number of clients in the queue at the station with the specified id.

\item
\cmd{int getWIP()}:\\
Gets the current number of clients in the system.
  
\item
\cmd{int getNQ()}:\\
Gets the current number of waiting clients in the system.

\end{itemize}

\section{Number of operators in a resource}

\begin{itemize}

\item
\cmd{int getAllResourceCount()}:\\
Returns the current number of operators in all resources together.
  
\item
\cmd{int getResourceCount(final int resourceId)}:\\
Returns the current number of operators in the resource with the specified id.  
  
\item
\cmd{boolean setResourceCount(final int resourceId, final int count)}:\\
Sets the number of operators in the resource with the specified id.
To be able to set the number of operators in a resource at runtime,
the initial number of operators in the resource has to be a fixed number
(not infinite many and not by a time table). Additionally no down times
are allowed for this resource.
The function returns \cm{true} if the number of operators has successfully 
been changed. If the new number of operators is less than the previous number,
the new number may is not instantly visible in the simulation system because
removed but working operators will finish their current tasks before they are
actually removed.

\item
\cmd{int getAllResourceDown()}:\\
Returns the current number of operators in down time in all resources together.

\item
\cmd{int getResourceDown(final int resourceId)}:\\
Returns the current number of operators in down time in the resource with the specified id.

\end{itemize}

\section{Trigger signal}

\begin{itemize}

\item
\cmd{signal(final String signalName)}:\\
Fires the signal with the given name.

\end{itemize}

\section{Run external code}

\begin{itemize}

\item
\cmd{Object runPlugin(final String className, final String functionName,\\final Object data)}:\\
Runs the specified method in the specified class and passes the optional parameter \cm{data} to the method.
The return value of the method will be returned by \cm{runPlugin}. If calling the external method fails,
\cm{runPlugin} will return \cm{null}.

\end{itemize}



\chapter{\texttt{ClientInterface} accessible via \texttt{sim.getClient()}}

The \texttt{ClientInterface} interface allows to access the data of the current
client while the simulation is running. It is only available if the execution
was triggered by a client. The following methods are available in this interface:

\begin{itemize}

\item
\cmd{Object calc(final String expression)}:\\
Calculates the expression passed as a string by means of the term evaluation function,
which is also used in various other places in the program (see part \ref{part:Rechenbefehle}), and returns the result as a \cm{Double} number.
If the expression can not be calculated, an error message is returned as a string.
The term evaluation function allows access to all known probability distributions,
the Erlang C calculator, etc.

\item
\cmd{String getTypeName()}:\\
Gets the name of the type of the client who has triggered the processing of the script.
  
\item
\cmd{boolean isWarmUp()}:\\
Gets \cm{true} of \cm{false} depending if the current client was generated during the warm-up phase and
therefore will not be recorded in the statistics.
  
\item
\cmd{boolean isInStatistics()}:\\
Gets true of false depending if the current client is to be recorded in the statistics.
This value is independent of the warm-up phase. A client will only be recorded if he was
generated after the warm-up phase and this value is true.
  
\item
\cmd{void setInStatistics(final boolean inStatistics)}:\\
Sets if a client is to be recorded in the statistics.
This value is independent of the warm-up phase. A client will only be recorded if he was
generated after the warm-up phase and this value is not set to false.
  
\item
\cmd{long getNumber()}:\\
Get the 1 based consecutive number of the current client.
When using multiple simulation threads this number is thread local.
  
\item
\cmd{double getWaitingSeconds()}:\\
Gets the current waiting time of the client who has triggered the processing of the script as a seconds numerical value.
  
\item
\cmd{String getWaitingTime()}:\\
Gets the current waiting time of the client who has triggered the processing of the script as a formated time value as a string.
    
\item
\cmd{void setWaitingSeconds(final double seconds)}:\\
Sets the current waiting time of the client who has triggered the processing of the script.
  
\item
\cmd{double getTransferSeconds()}:\\
Gets the current transfer time of the client who has triggered the processing of the script as a seconds numerical value.
  
\item
\cmd{String getTransferTime()}:\\
Gets the current transfer time of the client who has triggered the processing of the script as a formated time value as a string.
  
\item
\cmd{void setTransferSeconds(final double seconds)}:\\
Sets the current transfer time of the client who has triggered the processing of the script.
  
\item
\cmd{double getProcessSeconds()}:\\
Gets the current processing time of the client who has triggered the processing of the script as a seconds numerical value.
  
\item
\cmd{String getProcessTime()}:\\
Gets the current processing time of the client who has triggered the processing of the script as a formated time value as a string.
  
\item
\cmd{void setProcessSeconds(final double seconds)}:\\
Sets the current processing time of the client who has triggered the processing of the script.
  
\item
\cmd{double getResidenceSeconds()}:\\
Gets the current residence time of the client who has triggered the processing of the script as a seconds numerical value.

\item
\cmd{String getResidenceTime()}:\\
Gets the current residence time of the client who has triggered the processing of the script as a formated time value as a string.

\item
\cmd{void setResidenceSeconds(final double seconds)}:\\
Sets the current residence time of the client who has triggered the processing of the script.
  
\item
\cmd{double getValue(final int index)}:\\
Gets for the current client the numerical value which is stored by the index \cm{index}.
  
\item
\cmd{void setValue(final int index, final int value)},\\
\cmd{void setValue(final int index, final double value)},\\
\cmd{void setValue(final int index, final String value)}:\\
Sets for the current client the \cm{value} for the index \cm{index}.
If \cm{value} is a string, the string is interpreted by
\cm{calc(final String expression)} before assigning the result.
  
\item
\cmd{String getText(final String key)}:\\
Gets for the current client the string which is stored by the key \cm{key}.
  
\item
\cmd{void setText(final String key, final String value)}:\\
Sets for the current client string \cm{value} for the key \cm{key}.
	
\end{itemize}



\chapter{\texttt{InputValueInterface} accessible via \texttt{sim.getInputValue()}}

The \texttt{InputValueInterface} interface allows to access the next input value
if the processing was triggered by a Input (Script) element.
The following methods are available in this interface:

\begin{itemize}
\item
\cmd{double get()}:\\
This function returns the current input value.
\end{itemize}



\chapter{\texttt{ClientsInterface} accessible via \texttt{sim.getClients()}}

The \texttt{ClientsInterface} object is only available within a hold by script condition element
and allows to access the waiting clients and to release them.

\begin{itemize}

\item
\cmd{int count()}:\\
Returns the current number of waiting clients. For the other
methods a single client can be accessed via the index parameter
(valued from 0 to \cm{count()}-1).

\item
\cmd{String clientTypeName(final int index)}:\\
Gets the name of the type of the client.

\item
\cmd{double clientWaitingSeconds(final int index)}:\\
Gets the current waiting time of the client as a seconds numerical value.

\item
\cmd{String clientWaitingTime(final int index)}:\\
Gets the current waiting time of the client as a formated time value as a string.

\item
\cmd{double clientTransferSeconds(final int index)}:\\
Gets the current transfer time of the client as a seconds numerical value.

\item
\cmd{String clientTransferTime(final int index)}:\\
Gets the current transfer time of the client as a formated time value as a string.

\item
\cmd{double clientProcessSeconds(final int index)}:\\
Gets the current processing time of the client as a seconds numerical value.

\item
\cmd{String clientProcessTime(final int index)}:\\
Gets the current processing time of the client as a formated time value as a string.

\item
\cmd{double clientResidenceSeconds(final int index)}:\\
Gets the current residence time of the client as a seconds numerical value.

\item
\cmd{String clientResidenceTime(final int index)}:\\
Gets the current residence time of the client as a formated time value as a string.

\item
\cmd{void clientData(final int index, final int data, final double value)}:\\
Set the numerical value specified by the third parameter for the data element which index is specified via the second parameter of the selected client.

\item
\cmd{double clientData(final int index, final int data)}:\\
Returns the data element which index is specified via the second parameter of the selected client.

\item
\cmd{String clientTextData(final int index, final String key)}:\\
Returns the data element which key is specified via the second parameter of the selected client.

\item
\cmd{String clientTextData(final int index, final String key, final String value)}:\\
Set the text value specified by the third parameter for the key which is specified via the second parameter of the selected client.

\item
\cmd{void release(final int index)}:\\
Causes the forwarding of the specified client.

\end{itemize}


\chapter{\texttt{OutputInterface} accessible via \texttt{sim.getOutput()}}

The \texttt{OutputInterface} interface provides functions for output of filtered results:

\begin{itemize}

\item
\cmd{void setFormat(final String format)}:\\
This command allows to setup the format that is used in \cm{print} and \cm{println}
for outputing numbers as strings. You can specify whether to use a floating point notation or percent notation or interpreting
the value as a time. As default floating point notation is used.

\begin{itemize}
\item
\cmd{Fraction}:\\
Using floating point notation for numbers ($0.375$ for example).
\item
\cmd{Percent}:\\
Using percent notation for numbers ($35.7\%$ for example).
\item
\cmd{Number}:\\
Interpreting numbers as normal number values (decimal or percent).
\item
\cmd{Time}:\\
Interpreting numbers as time values.
\end{itemize}

\item
\cmd{void setSeparator(final String separator)}:\\
This command allows to select the separator to be used when printing out arrays.
\begin{itemize}
\item
\cmd{Semicolon}:\\
Semicolons as separators.
\item
\cmd{Line}:\\
Line break as separators.
\item
\cmd{Tabs}:\\
Tabulators as separators.
\end{itemize}

\item
\cmd{void setDigits(final int digits)}:\\
This command allows to define the number of digits to be displayed when
printing a number in local notation. A negative value means that all
available digits are being printed.
(If the system notation is used, always all available digits are being printed.)

\item
\cmd{void print(final Object obj)}:\\
Outputs the passed expression.
Strings will be written directly. Numbers are formated according to the format
defined via \cm{setFormat}.

\item
\cmd{void println(final Object obj)}:\\
Outputs the passed expression and adds a line break after the expression.
Strings will be written directly. Numbers are formated according to the format
defined via \cm{setFormat}.

\item
\cmd{void newLine()}:\\
Outputs a line break. This functions is equivalent to calling
\cm{println("{}"{})}.

\item
\cmd{void tab()}:\\
Outputs a tabulator.

\item
\cmd{void cancel()}:\\
Sets the cancel status. (When output is canceled to further file output will be performed.)

\end{itemize}



\chapter{\texttt{FileOutputInterface} accessible via \texttt{sim.getFileOutput()}}

The \texttt{FileOutputInterface} interface offers all function the \texttt{OutputInterface} interface has
but is only available when running a parameter series script. In opposite
to the \texttt{OutputInterface} interface the output of the \texttt{FileOutputInterface} interface
is not written to the default output but is appended to a file which has
to be specified by \texttt{sim.getFileOutput().setFile("{}Filename")} before.



\chapter{\texttt{ModelInterface} accessible via \texttt{sim.getModel()}}

The \texttt{ModelInterface} interface is only available during parameter series
script execution and offers functions for accessing the model
properties and for starting simulations.

\begin{itemize}

\item
\cmd{void reset()}:\\
Resets the model to the initial state.

\item
\cmd{void run()}:\\
Simulates the current model.
The results can be accessed by the \cm{StatisticsInterface} interface after the simulation.

\item
\cmd{boolean setDistributionParameter(final String xmlName, final int number,\\final double value)}:\\
Sets the distribution parameter \cm{number} (from 1 to 4) of the distribution referred
to by \cm{xmlName}.

\item
\cmd{boolean setMean(final String xmlName, final double value)}:\\
Sets the mean of the distribution referred to by \cm{xmlName} to the specified value.

\item
\cmd{boolean setSD(final String xmlName, final double value)}:\\
Sets the standard deviation of the distribution referred to by \cm{xmlName} to the specified value.  

\item
\cmd{boolean setString(final String xmlName, final String value)}:\\
Writes the string \cm{value} to the location referred to by \cm{xmlName}.

\item
\cmd{boolean setValue(final String xmlName, final double value)}:\\
Writes the number \cm{value} to the location referred to by \cm{xmlName}.  

\item
\cmd{String xml(final String xmlName)}:\\
Loads the xml field which is specified by the parameter and returns the data 
as String. This function is the equivalent of \cm{sim.getStatistics().xml(xmlName)}
for models.

\item
\cmd{getResourceCount(final String resourceName)}:\\
Gets the number of operations in the resource with name \cm{resourceName}.
If the resource does not exist or is not defined by a fixed number of operators
the function will return -1.

\item
\cmd{boolean setResourceCount(final String resourceName, final int count)}:\\
Sets the number of operations in the resource with name \cm{resourceName}.

\item
\cmd{String getGlobalVariableInitialValue(final String variableName)}:\\
Gets the expression by which the initial value of the global variable with
name \cm{variableName} is calculated. If the is no global variable with
this name the function will return an empty string.

\item
\cmd{boolean setGlobalVariableInitialValue(final String variableName,\\final String expression)}:\\
Sets the expression by which the initial value of the global variable with
name \cm{variableName} is calculated.

\item
\cmd{void cancel()}:\\
Sets the cancel status. (When processing is canceled to further simulations will be performed.)

\end{itemize}

\section{Accessing station data}

\begin{itemize}

\item
\cmd{int getStationID(final String name)}:\\
Gets the ID of a station based on its name.
If there is no station with a matching name, the function will return -1.

\end{itemize}



\chapter{XML selection commands}

By the parameters of the functions of the \cmd{StatisticsInterface} interface the content of the
value or of an attribute of an XML element can be read.
The selection of an XML element is done multistaged step by step divided by
"\texttt{->}" characters. Between the "\texttt{->}" characters the names of the individual XML nodes are
noted. In addition in square brackets names and values of attributes can be specified to filter by whom.

Examples:

\begin{itemize}

\item
\cmd{sim.getStatistics().xml("{}Model->ModellName")}:\\
Shows the content of the element \cm{ModelName}, which is a child element of \cm{Model}.

\item
\cmd{sim.getStatistics().xml("{}StatisticsInterArrivalTimesClients->\\{}Station[Type=$\backslash$"{}Source id=1$\backslash$"]->[Mean]")}:\\
Selects the \cm{Station} sub element of the \cm{StatisticsInterArrivalTimesClients} element, for
which the \cm{Type} attribute is set to \cm{Source id=1}. And returns the value of the
\cm{Mean} attribute.

\end{itemize}