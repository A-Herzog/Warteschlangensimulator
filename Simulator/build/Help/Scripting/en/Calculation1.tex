\part{Calculation commands reference}\label{part:Rechenbefehle}

When using calculation commands in Warteschlangensimulator,
it is distinguished between \textbf{expressions} and \textbf{comparisons}.
Expressions are used to calculate a numerical values, which are e.g.\ used as periods of time.
Comparisons provide a yes/no decision (for example, whether a client should be directed
in a particular direction). Unlike expressions, comparisons always contain at least one comparison operator.

All commands presented below are each recognized in \textbf{any case}.
There is no distinction between different case types.



\chapter{Constants}

The following constants are available in all calculation commands:

\begin{itemize}

\item
\cmd{e}: Returns the basis of the exponential function $e^x$. It is
$e\approx 2.718281828459$.

\item
\cmd{pi}: Returns the value of the circle constant $\pi$. It is
$\pi\approx 3.1415926535898$.

\end{itemize}



\chapter{Variables}

When calculating values in the context of a concrete client, the variables
\begin{itemize}
\item
\cmd{w} for the \textbf{previous waiting time} of the client,
\item
\cmd{t} for the \textbf{previous transfer time} of the client ant
\item
\cmd{p} for the \textbf{previous operating time} of the client are always available.
\end{itemize}

If the calculation is done for getting a clients score, the variable \cmd{w} does not contain
the total waiting time of the current client but the waiting time of the current client at the
current station.

Furthermore, all variables that are defined by an assignment element are always available.
Before the first assignment of a value to a variable, it has the value 0.



\chapter{Basic arithmetic operations}

Supported instructions for the basic arithmetic operations:
\begin{itemize}
\item Addition: \cmd{$+$}
\item Subtraction: \cmd{$-$}
\item Multiplication: \cmd{$*$}
\item Division: \cmd{$/$}
\item Potentiate: \cmd{$\hat~$}
\end{itemize}

The rule \textbf{point before line calculation} is taken into account.
To enforce deviating evaluations, \textbf{brackets} can be set.



\chapter{Trailing instructions}

The following expressions can be written directly behind a number:

\begin{itemize}
\item
\cmd{\%}:
The numerical value left to this symbol is interpreted as a percent value,
for example $30\%=0.3$.

\item
\cmd{$^2$}:
Exponentiate number by 2.

\item
\cmd{$^3$}:
Exponentiate number by 3.

\item
\cmd{!}:
Calculate factorial of the number, for example\ $4!=1\cdot2\cdot3\cdot4=24$.

\item
\cmd{$^{\circ}$}:
Converts the value left to this symbol from grad to radian, for example $180^{\circ}=3.1415\ldots$.\\
(See also section \ref{sec:Winkelfunktionen} in which the supported
trigonometric functions are presented.)
\end{itemize}



\chapter{General functions}

\begin{itemize}

\item
\cmd{abs(x)}:
Absolute value, for example \cm{abs(-5)=5}.

\item
\cmd{ceil(x)}:
Round to next bigger integer number, for example \cm{ceil(2.1)=3}

\item
\cmd{exp(x)}:
Exponential function $e^x$.

\item
\cmd{factorial(x)}:
Factorial, for example $4!=1\cdot2\cdot3\cdot4=24$.

\item
\cmd{floor(x)}:
Round to next smaller integer number, for example \cm{floor(2.9)=2}

\item
\cmd{frac(x)}:
Fraction part, for example \cm{frac(1.3)=0,3}

\item
\cmd{gamma(x)}:
Gamma funkcion, for example \cm{gamma(5)=4!=24}

\item
\cmd{int(x)}:
Integer part, for example \cm{int(2.9)=2}

\item
\cmd{log(x)}:
Logarithm to the base $e$.

\item
\cmd{log(x;b)}:
Logarithm to the base $b$.

\item
\cmd{ld(x)}:
Logarithm to the base $2$, for example \cm{ld(256)=8}.

\item
\cmd{lg(x)}:
Logarithm to the base $10$, for example \cm{lg(100)=2}.

\item
\cmd{ln(x)}:
Logarithm to the base $e$.

\item
\cmd{modulo(a;b)} oder \cmd{mod(a;b)}:
Division reminder when dividing a/b

\item
\cmd{pow(x;y)}:
Exponentiate $x^y$.

\item
\cmd{round(x)}:
Round, for example \cm{round(4.4)=4} and \cm{round(4.5)=5}.

\item
\cmd{sign(x)}:
Sign of a number, for example \cm{sign(3)=1} and \cm{sign(-3)=-1}.

\item
\cmd{sqrt(x)}:
Square root, for example \cm{sqrt{81}=9}.

\item
\cmd{sqr(x)}:
Square the number, for example \cm{sqr(4)=16}.

\end{itemize}



\section{Random numbers}

The following commands can be used to generate random numbers
that are \textbf{equally distributed} in a certain area.
Section \ref{sec:Wahrscheinlichkeitsverteilungen} introduces
additional functions for generating random numbers according
to certain distribution functions.

\begin{itemize}

\item
\cmd{random()}:
Random number between 0 (inclusive) and 1 (exclusive).

\item
\cmd{random(x)}:
Random number between 0 (inclusive) and x (exclusive).

\end{itemize}





\chapter{Trigonometric functions}\label{sec:Winkelfunktionen}

The trigonometric functions always refer to $2\pi$ as a full circle (radians).
If angles in degrees ($360^\circ$ for the full circle) are to be specified
in the elementary trigonometric functions,
these have to be converted to radians using the angle functions, for example
\cm{sin($90^\circ$)=1}.



\section{Elementary trigonometric functions}

\begin{itemize}

\item
\cmd{sin(x)}:
Sine

\item
\cmd{cos(x)}:
Cosine

\item
\cmd{tan(x)}:
Tangent

\item
\cmd{cot(x)}:
Cotangent

\end{itemize}



\section{Hyperbolic trigonometric functions}

\begin{itemize}

\item
\cmd{sinh(x)}:
Sine hyperbolicus

\item
\cmd{cosh(x)}:
Cosine hyperbolicus

\item
\cmd{tanh(x)}:
Tangent hyperbolicus

\item
\cmd{coth(x)}:
Cotangent hyperbolicus

\end{itemize}



\section{Inverse of the elementary trigonometric functions}

\begin{itemize}

\item
\cmd{arcsin(x)}:
Arcus sine

\item
\cmd{arccos(x)}:
Arcus cosine

\item
\cmd{arctan(x)}:
Arcus tangent

\item
\cmd{arccot(x)}:
Arcus cotangent

\end{itemize}



\section{Inverse of the hyperbolic trigonometric functions}

\begin{itemize}

\item
\cmd{arcsinh(x)}:
Arcus sine hyperbolicus

\item
\cmd{arccosh(x)}:
Arcus cosine hyperbolicus

\item
\cmd{arctanh(x)}:
Arcus-Tangent hyperbolicus

\item
\cmd{arccoth(x)}:
Arcus-Cotangent hyperbolicus

\end{itemize}



\chapter{Funktionen mit mehreren Parametern}

The following functions can accept any number of parameters.
The individual parameters have to be specified separately by semicolon ";".

\begin{itemize}

\item
\cmd{Min(a;b;c;...)}:
Calculates the minimum of the given numbers.

\item
\cmd{Max(a;b;c;...)}:
Calculates the maximum of the given numbers.

\item
\cmd{Sum(a;b;c;...)}:
Calculates the sum of the given numbers.

\item
\cmd{Mean(a;b;c;...)}:
Calculates the mean value of the given numbers.

\item
\cmd{Median(a;b;c;...)}:
Calculates the median of the given numbers.

\item
\cmd{Var(a;b;c;...)}:
Calculates the sample variance of the given numbers.

\item
\cmd{SD(a;b;c;...)}:
Calculates the sample standard deviation of the given numbers.

\item
\cmd{SCV(a;b;c;...)}:
Calculates the squared coefficient of variation of the given numbers.

\item
\cmd{CV(a;b;c;...)}:
Calculates the coefficient of variation of the given numbers.

\end{itemize}



\chapter{Probability distributions}\label{sec:Wahrscheinlichkeitsverteilungen}

By using the following commands both values of the density and the cumulative distribution function
of the following probability distributions can be calculated as well as random numbers are based on
one of these probability distributions:



\section{Exponential distribution with mean \texorpdfstring{$a$}{a}}

\begin{itemize}

\item
\cmd{ExpDist(x;a;0)}:
Calculates the probability density at $x$.

\item
\cmd{ExpDist(x;a;1)}:
Calculates the cumulative distribution function at $x$.

\item
\cmd{ExpDist(a)}:
Generates a random number based on this distribution.

\end{itemize}



\section{Uniform distribution in the interval \texorpdfstring{$[a;b]$}{[a;b]}}

\begin{itemize}

\item
\cmd{UniformDist(x;a;b;0)}:
Calculates the probability density at $x$.

\item
\cmd{UniformDist(x;a;b;1)}:
Calculates the cumulative distribution function at $x$.

\item
\cmd{UniformDist(a;b)}:
Generates a random number based on this distribution.

\end{itemize}



\section{Normal distribution with mean \texorpdfstring{$a$}{a} and standard deviation \texorpdfstring{$b$}{b}}

\begin{itemize}

\item
\cmd{NormalDist(x;a;b;0)}:
Calculates the probability density at $x$.

\item
\cmd{NormalDist(x;a;b;1)}:
Calculates the cumulative distribution function at $x$.

\item
\cmd{NormalDist(a;b)}:
Generates a random number based on this distribution.

\end{itemize}



\section{Log-normal distribution with mean \texorpdfstring{$a$}{a} and standard deviation \texorpdfstring{$b$}{b}}

\begin{itemize}

\item
\cmd{LogNormalDist(x;a;b;0)}:
Calculates the probability density at $x$.

\item
\cmd{LogNormalDist(x;a;b;1)}:
Calculates the cumulative distribution function at $x$.

\item
\cmd{LogNormalDist(a;b)}:
Generates a random number based on this distribution.

\end{itemize}



\section{Gamma distribution with parameters \texorpdfstring{$\alpha=a$}{a} and \texorpdfstring{$\beta=b$}{b}}

\begin{itemize}

\item
\cmd{GammaDist(x;a;b;0)}:
Calculates the probability density at $x$.

\item
\cmd{GammaDist(x;a;b;1)}:
Calculates the cumulative distribution function at $x$.

\item
\cmd{GammaDist(a;b)}:
Generates a random number based on this distribution.
\end{itemize}



\section{Erlang distribution with parameters \texorpdfstring{$n$}{n} and \texorpdfstring{$\lambda=l$}{l}}

\begin{itemize}

\item
\cmd{ErlangDist(x;n;l;0)}:
Calculates the probability density at $x$.

\item
\cmd{ErlangDist(x;n;l;1)}:
Calculates the cumulative distribution function at $x$.

\item
\cmd{ErlangDist(n;b)}:
Generates a random number based on this distribution.

\end{itemize}



\section{Beta distribution in the interval \texorpdfstring{$[a;b]$}{[a;b]} and with parameters \texorpdfstring{$\alpha=c$}{c} and \texorpdfstring{$\beta=d$}{d}}

\begin{itemize}

\item
\cmd{BetaDist(x;a;b;c;d;0)}:
Calculates the probability density at $x$.

\item
\cmd{BetaDist(x;a;b;c;d;1)}:
Calculates the cumulative distribution function at $x$.

\item
\cmd{BetaDist(a;b;c;d)}:
Generates a random number based on this distribution.

\end{itemize}



\section{Weibull distribution with parameters Scale=\texorpdfstring{$a$}{a} and Form=\texorpdfstring{$b$}{b}}

\begin{itemize}

\item
\cmd{WeibullDist(x;a;b;0)}:
Calculates the probability density at $x$.

\item
\cmd{WeibullDist(x;a;b;1)}:
Calculates the cumulative distribution function at $x$.

\item
\cmd{WeibullDist(a;b)}:
Generates a random number based on this distribution.

\end{itemize}



\section{Cauchy distribution with mean \texorpdfstring{$a$}{a} and Scale=\texorpdfstring{$b$}{b}}

\begin{itemize}

\item
\cmd{CauchyDist(x;a;b;0)}:
Calculates the probability density at $x$.

\item
\cmd{CauchyDist(x;a;b;1)}:
Calculates the cumulative distribution function at $x$.

\item
\cmd{CauchyDist(a;b)}:
Generates a random number based on this distribution.

\end{itemize}



\section{\texorpdfstring{Chi$^2$}{Chi2} distribution with \texorpdfstring{$n$}{n} degrees of freedom}

\begin{itemize}

\item
\cmd{ChiSquareDist(x;n;0)}:
Calculates the probability density at $x$.

\item
\cmd{ChiSquareDist(x;n;1)}:
Chi$^2$ distribution with $n$ degrees of freedom.

\item
\cmd{ChiSquareDist(n)}:
Generates a random number based on this distribution.

\end{itemize}



\section{Chi distribution with \texorpdfstring{$n$}{n} degrees of freedom}

\begin{itemize}

\item
\cmd{ChiDist(x;n;0)}:
Calculates the probability density at $x$.

\item
\cmd{ChiDist(x;n;1)}:
Chi distribution with $n$ degrees of freedom.

\item
\cmd{ChiDist(n)}:
Generates a random number based on this distribution.

\end{itemize}



\section{F distribution with \texorpdfstring{$a$}{a} degrees of freedom for the numerator and \texorpdfstring{$b$}{b} degrees of freedom for the denominator}

\begin{itemize}

\item
\cmd{FDist(x;a;b;0)}:
Calculates the probability density at $x$.

\item
\cmd{FDist(x;a;b;1)}:
Calculates the cumulative distribution function at $x$.

\item
\cmd{FDist(a;b)}:
Generates a random number based on this distribution.

\end{itemize}



\section{Johnson SU distribution with parameters \texorpdfstring{$\gamma=a$}{a}, \texorpdfstring{$\xi=b$}{b}, \texorpdfstring{$\delta=c$}{c} and \texorpdfstring{$\lambda=d$}{d}}

\begin{itemize}

\item
\cmd{JohnsonSUDist(x;a;b;c;d;0)}:
Calculates the probability density at $x$.

\item
\cmd{JohnsonSUDist(x;a;b;c;d;1)}:
Calculates the cumulative distribution function at $x$.

\item
\cmd{JohnsonSUDist(a;b;c;d)}:
Generates a random number based on this distribution.  

\end{itemize}



\section{Triangular distribution over \texorpdfstring{$[a;c]$}{[a;c]} with most likely value \texorpdfstring{$b$}{b}}

\begin{itemize}

\item
\cmd{TriangularDist(x;a;b;c;0)}:
Calculates the probability density at $x$.

\item
\cmd{TriangularDist(x;a;b;c;1)}:
Calculates the cumulative distribution function at $x$.

\item
\cmd{TriangularDist(a;b;c)}:
Generates a random number based on this distribution.

\end{itemize}



\section{Laplace distribution with mean \texorpdfstring{$mu$}{mu} and scale factor \texorpdfstring{$b$}{b}}

\begin{itemize}

\item
\cmd{LaplaceDist(x;mu;b;0)}:
Calculates the probability density at $x$.

\item
\cmd{LaplaceDist(x;mu;b;1)}:
Calculates the cumulative distribution function at $x$.

\item
\cmd{LaplaceDist(mu;b)}:
Generates a random number based on this distribution.

\end{itemize}



\section{Pareto distribution with scale parameter \texorpdfstring{$x_{\rm min}=xmin$}{xmin} and shape parameter \texorpdfstring{$\alpha=a$}{a}}

\begin{itemize}

\item
\cmd{ParetoDist(x;xmin;a;0)}:
Calculates the probability density at $x$.

\item
\cmd{ParetoDist(x;xmin;a;1)}:
Calculates the cumulative distribution function at $x$.

\item
\cmd{ParetoDist(xmin;a)}:
Generates a random number based on this distribution.

\end{itemize}



\section{Logistic distribution with mean \texorpdfstring{$\mu=mu$}{mu} and scale parameter \texorpdfstring{$s$}{s}}

\begin{itemize}

\item
\cmd{LogisticDist(x;mu;s;0)}:
Calculates the probability density at $x$.

\item
\cmd{LogisticDist(x;mu;s;1)}:
Calculates the cumulative distribution function at $x$.

\item
\cmd{LogisticDist(mu;s)}:
Generates a random number based on this distribution.

\end{itemize}


	
\section{Inverse gaussian distribution with \texorpdfstring{$\lambda=l$}{l} and mean \texorpdfstring{$mu$}{mu}}

\begin{itemize}

\item
\cmd{InverseGaussianDist(x;l;mu;0)}:
Calculates the probability density at $x$.

\item
\cmd{InverseGaussianDist(x;l;mu;1)}:
Calculates the cumulative distribution function at $x$.

\item
\cmd{InverseGaussianDist(l;mu)}:
Generates a random number based on this distribution.

\end{itemize}



\section{Rayleigh distribution with mean \texorpdfstring{$mu$}{mu}}

\begin{itemize}

\item
\cmd{RayleighDist(x;mu;0)}:
Calculates the probability density at $x$.

\item
\cmd{RayleighDist(x;mu;1)}:
Calculates the cumulative distribution function at $x$.

\item
\cmd{RayleighDist(mu)}:
Generates a random number based on this distribution.

\end{itemize}



\section{Log-Logistic distribution with \texorpdfstring{$\alpha$}{alpha} and mean \texorpdfstring{$\beta$}{beta}}

\begin{itemize}

\item
\cmd{LogLogisticDist(x;alpha;beta;0)}:
Calculates the probability density at $x$.

\item
\cmd{LogLogisticDist(x;alpha;beta;1)}:
Calculates the cumulative distribution function at $x$.

\item
\cmd{LogLogisticDist(alpha;beta)}:
Generates a random number based on this distribution.

\end{itemize}



\section{Power distribution on \texorpdfstring{$[a;b]$}{[a;b]} with exponent \texorpdfstring{$c$}{c}}

\begin{itemize}

\item
\cmd{PowerDist(x;a;b;c;0)}:
Calculates the probability density at $x$.

\item
\cmd{PowerDist(x;a;b;c;1)}:
Calculates the cumulative distribution function at $x$.

\item
\cmd{PowerDist(a;b;c)}:
Generates a random number based on this distribution.

\end{itemize}



\section{Gumbel distribution with expected value \texorpdfstring{$a$}{a} and standard deviation \texorpdfstring{$b$}{b}}

\begin{itemize}

\item
\cmd{GumbelDist(x;a;b;0)}:
Calculates the probability density at $x$.

\item
\cmd{GumbelDist(x;a;b;1)}:
Calculates the cumulative distribution function at $x$.

\item
\cmd{GumbelDist(a;b)}:
Generates a random number based on this distribution.

\end{itemize}



\section{Fatigue life distribution with location parameter \texorpdfstring{$\mu$}{mu}, scale parameter \texorpdfstring{$\beta$}{beta} and form parameter \texorpdfstring{$\gamma$}{gamma}}

\begin{itemize}

\item
\cmd{FatigueLifeDist(x;mu;beta;gamma;0)}:
Calculates the probability density at $x$.

\item
\cmd{FatigueLifeDist(x;mu;beta;gamma;1)}:
Calculates the cumulative distribution function at $x$.

\item
\cmd{FatigueLifeDist(mu;beta;gamma)}:
Generates a random number based on this distribution.

\end{itemize}



\section{Frechet distribution with location parameter \texorpdfstring{$\delta$}{delta}, scale parameter \texorpdfstring{$\beta$}{beta} and form parameter \texorpdfstring{$\alpha$}{alpha}}

\begin{itemize}

\item
\cmd{FrechetDist(x;delta;beta;alpha;0)}:
Calculates the probability density at $x$.

\item
\cmd{FrechetDist(x;delta;beta;alpha;1)}:
Calculates the cumulative distribution function at $x$.

\item
\cmd{FrechetDist(delta;beta;alpha)}:
Generates a random number based on this distribution.

\end{itemize}



\section{Hyperbolic secant distribution with mean \texorpdfstring{$a$}{a} and standard deviation \texorpdfstring{$b$}{b}}

\begin{itemize}

\item
\cmd{HyperbolicSecantDist(x;a;b;0)}:
Calculates the probability density at $x$.

\item
\cmd{HyperbolicSecantDist(x;a;b;1)}:
Calculates the cumulative distribution function at $x$.

\item
\cmd{HyperbolicSecantDist(a;b)}:
Generates a random number based on this distribution.

\end{itemize}



\section{Distribution based on empirical values}

\begin{itemize}

\item
\cmd{EmpiricalDensity(x;value1;value2;value3;...;max)}:\\
Calculates the probability density at $x$.
The specified values will be used for the density in the range from 0 to $max$.

\item
\cmd{EmpiricalDistribution(x;value1;value2;value3;...;max)}:\\
Calculates the cumulative distribution function at $x$.
The specified values will be used for the density in the range from 0 to $max$.

\item
\cmd{EmpiricalRandom(value1;value2;value3;...;max)}:\\
Generates a random number based on this distribution.
The specified values will be used for the density in the range from 0 to $max$.

\item
\cmd{EmpiricalDistributionMean(value1;value2;value3;...;max)}:\\
Calculates the expected value of the distribution.

\item
\cmd{EmpiricalDistributionMedian(value1;value2;value3;...;max)}:\\
Calculates the median of the distribution.

\item
\cmd{EmpiricalDistributionQuantil(value1;value2;value3;...;max;p)}:\\
Calculates the quantil for the probability p of the distribution.

\item
\cmd{EmpiricalDistributionSD(value1;value2;value3;...;max)}:\\
Calculates the standard deviation of the distribution.

\item
\cmd{EmpiricalDistributionVar(value1;value2;value3;...;max)}:\\
Calculates the variance of the distribution.

\item
\cmd{EmpiricalDistributionCV(value1;value2;value3;...;max)}:\\
Calculates the coefficient of variation of the distribution.

\end{itemize}



\chapter{Erlang C calculator}

By using the following command some performance indicators can be calculated
using the extended Erlang C formula:

\begin{itemize}

\item
\cmd{ErlangC(lambda;mu;nu;c;K;-1)}:\\
Calculates the average queue length $\E[N_Q]$. 

\item
\cmd{ErlangC(lambda;mu;nu;c;K;-2)}:\\
Calculates the average number of clients in the system $\E[N]$.

\item
\cmd{ErlangC(lambda;mu;nu;c;K;-3)}:\\
Calculates the average waiting time $\E[W]$.

\item
\cmd{ErlangC(lambda;mu;nu;c;K;-4)}:\\
Calculates the average residence time $\E[V]$.

\item
\cmd{ErlangC(lambda;mu;nu;c;K;-5)}:\\
Calculates the average accessibility $1-P(A)$.

\item
\cmd{ErlangC(lambda;mu;nu;c;K;t)}:\\
Calculates the the probability for the service level at the $t$ seconds threshold $P(W\le t)$.

\end{itemize}

The parameters have the following meanings:
\begin{itemize}
\item
\cm{lambda}:\\
Arrival rate $\lambda$ (in clients per time unit), i.e.\ inverse of the mean inter-arrival time.
\item
\cm{mu}:\\
Service rate $\mu$ (in clients per time unit), i.e.\ inverse of the mean service time.
\item
\cm{nu}:\\
Cancelation rate $\nu$ (in clients per time unit), i.e.\ inverse of the mean waiting time tolerance.
\item
\cm{c}:\\
Number of available parallel operating servers.
\item
\cm{K}:\\
Number of available places in the system (waiting and processing places together, i.e.\ it is $K\ge c$).
\end{itemize}



\chapter{Allen-Cunneen approximation formula}

By using the following command some performance indicators can be calculated
using the Allen-Cunneen approximation formula:

\begin{itemize}

\item
\cmd{AllenCunneen(lambda;mu;cvI;cvS;c;-1)}:\\
Calculates the average queue length $E[N_Q]$. 

\item
\cmd{AllenCunneen(lambda;mu;cvI;cvS;c;-2)}:\\
Calculates the average number of clients in the system $\E[N]$.

\item
\cmd{AllenCunneen(lambda;mu;cvI;cvS;c;-3)}:\\
Calculates the average waiting time $\E[W]$.

\item
\cmd{AllenCunneen(lambda;mu;cvI;cvS;c;-4)}:\\
Calculates the average residence time $\E[V]$.
\end{itemize}

The parameters have the following meanings:
\begin{itemize}
\item
\cm{lambda}:\\
Arrival rate $\lambda$ (in clients per time unit), i.e.\ inverse of the mean inter-arrival time.
\item
\cm{mu}:\\
Service rate $\mu$ (in clients per time unit), i.e.\ inverse of the mean service time.
\item
\cm{cvI}:\\
Coefficient of variation of the inter-arrival times $\CV[I]$ (small values mean that the inter-arrival times are very homogeneous).
\item
\cm{cvS}:\\
Coefficient of variation of the service times $\CV[S]$ (small values mean that the operations are very homogeneous).
\item
\cm{c}:\\
Number of available parallel operating servers.
\end{itemize}